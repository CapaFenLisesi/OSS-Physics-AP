\documentclass[12pt,aspectratio=169]{beamer}

\mode<presentation>
{
  \usetheme{Singapore}
 %\setbeamersize{text margin left=.6cm,text margin right=.6cm}
%  \setbeamertemplate{navigation symbols}{} % suppress nav bar
%  \setbeamercovered{transparent}
}
\usefonttheme{professionalfonts}
\usepackage{graphicx}
\usepackage{tikz}
\usepackage{amsmath}
\usepackage{mathpazo}
\usepackage[scaled]{helvet}
\usepackage{xcolor,colortbl}
\usepackage{siunitx}
\usepackage{hyperref}

\sisetup{detect-all}

\title{Classes 19: Fluid Mechanics}
\subtitle{AP Physics}
\author[TML]{Dr.\ Timothy Leung}
\institute{Olympiads School}
\date{March 2018}

\newcommand{\pic}[2]{\includegraphics[width=#1\textwidth]{#2}}
\newcommand{\mb}[1]{\mathbf{#1}}
\newcommand{\eq}[2]{\vspace{#1}{\Large\begin{displaymath}#2\end{displaymath}}}
%\newcommand{\protip}[1]{
%  \begin{center}
%    \fbox{
%      \begin{minipage}{.95\textwidth}
%        {\footnotesize
%          \textbf{Protip: }#1
%        }
%      \end{minipage}
%    }
%  \end{center}
%}

\begin{document}

\begin{frame}
  \maketitle
\end{frame}


%\section[Intro]{Introduction}

\begin{frame}
  \frametitle{Files for You to Download}
  Download from the school website:
  \begin{enumerate}
  \item\texttt{19-fluidMechanics.pdf}---This
    presentation. If you want to print the slides on paper, I recommend
    printing 4 slides per page.
  \item\texttt{20-Homework.pdf}---Homework assignment for Classes 19 and 20,
    which cover Fluid Mechanics and Thermodynamics
  \end{enumerate}

  \vspace{.2in}Please download/print the PDF file before each class. When you
  are taking notes, pay particular attention to things I say that aren't
  necessarily on the slides.
\end{frame}


\begin{frame}
  \frametitle{Disclaimer}
  \framesubtitle{Use of Calculus}
  Fluid mechanics is part of the AP Physics 2 Exam, which does not require
  calculus. However, in the interest in completeness, \emph{some} calculus will
  still be used when deriving equations.
\end{frame}


\begin{frame}
  \frametitle{What is a Fluid}

  \begin{itemize}
  \item\textbf{The simplistic explanstion:} anything that flows

  \item\textbf{The scientific explanation:} Any substancs that deform
    \emph{continuously} under oblique stress
  \end{itemize}
\end{frame}

\begin{frame}
  \frametitle{Properties of Fluids}
  \framesubtitle{Density}
\end{frame}


\begin{frame}
  \frametitle{Continuity}
  A fluid is considered to be continuous in space.
\end{frame}


\begin{frame}
  \frametitle{Properties of Fluids}
  \framesubtitle{Viscosity}
\end{frame}


\begin{frame}
  \frametitle{Hydrostatics}
\end{frame}


\begin{frame}
  \frametitle{Buoyancy}
  \framesubtitle{Everything Floats a Little}
  When an object is submerged inside a fluid (e.g.\ water, air, etc), the fluid
  exerts a pressure at the surface of the object. We can integrate the pressure
  over the entire surface area and find the total force the fluid exerts on the
  object.
\end{frame}



\begin{frame}
  \frametitle{Derivation of Buoyance Force}
  We can integrate the pressure over the entire surface to find the total force,
  or take some knowledge of vector calculus (divergence theorem):

  \eq{-.1in}{
    \mb{B}
    =-\oint_S p\mb{n}dS
    =-\iiint \nabla p dV
  }
  
  Since pressure is given by $p=\rho gz$---a function in $z$ only---the
  gradient easy to compute: $\nabla p=\rho g\hat{\mb{k}}$, giving us

  \eq{-.1in}{
    \mb{B}
    =\rho_\mathrm{fluid}g\hat{\mb{k}}\iiint dV = \rho_\mathrm{fluid}gV\hat{\mb{k}}
  }
\end{frame}

\begin{frame}
  \frametitle{Derivation of Buoyance Force}
  Although the derivation required a lot of calculus, the expression of
  buoyance force is \emph{very} straightforward:
  
  \eq{-.2in}{
    \boxed{\mb{B} = \rho_\mathrm{fluid}gV\hat{\mb{k}}=
      m_\mathrm{fluid}g\hat{\mb{k}}}
  }
  
  where $\rho_\mathrm{fluid}$ is the density of the displaced fluid, and
  $V$ is the volume displaced. This equation is known as
  \textbf{Archimedes' principle}.
  
  \vspace{.25in}\textbf{Buoyance force has a magnitude that equals to the
    weight of the fluid displaced by the submerged object, pointing upward.}
\end{frame}

\begin{frame}
  \frametitle{An Easier Explanation of Buoyancy}
  \framesubtitle{Not Much Calculus}
  \begin{columns}
    \column{.7\textwidth}
    There is a simpler way to find the buoyance force, by taking an
    infinitesimal ``tube'' of the object, and finding the pressure difference
    between the top and bottom of the tube:

    \vspace{-.5in}{\Large
      \begin{align*}
        \mb{B}&=\int (p_2-p_1)dA\\
        &= \rho g\int(z_2-z_1)dA\\
        &=\rho g V
      \end{align*}
    }

    \vspace{-.2in}which is the same expression that we got with calculus.

    \column{.3\textwidth}
    \pic{1}{buoyancy.jpg}
  \end{columns}
%  The pressure difference difference between the top and bottom . If we
%  integrate over the entire surface, we get the buoyance force $\mb{B}$:
%
%  \eq{-.2in}{
%    \boxed{\mb{B}=\rho_{\textrm{fluid}}gV_{\textrm{disp}}}
%  }
%
%  where $\rho_{\textrm{fluid}}$ is the density of the displaced fluid, and
%  $V_{\textrm{disp}}$ is the volume displaced. This equation is known as
%  \textbf{Archimedes' principle}.
\end{frame}


\begin{frame}
  \frametitle{Buoyancy}

%  Buoyancy depends on:
%  \begin{itemize}
%  \item the density of the (displaced) fluid $\rho_\mathrm{fluid}$
%  \item the volume of the fluid displaced $V$, and
%  \item the local acceleration due to gravity $g$
%  \end{itemize}
  Note that buoyancy does not depend on:
  \begin{itemize}
  \item the mass of the immersed object, or
  \item the density of the immersed object
  \end{itemize}
%\end{frame}
%
%\begin{frame}
%  \frametitle{Buoyancy}
  Objects immersed in a fluid have an ``apparent weight'' that is reduced by the
  buoyance force:

  \vspace{-.45in}{\Large
    \begin{align*}
      \mb{W}′ &= \mb{W}-\mb{B}\\
      \mb{W}′ &= (\rho_{\textrm{obj}}-\rho_{\textrm{fluid}})\mb{g}V
    \end{align*}
  }
  $\mb{W}'$ is proportional to the relative density
  ($\rho'=\rho_{\textrm{obj}}-\rho_{\textrm{fluid}}$)
\end{frame}

\begin{frame}
  \frametitle{Buoyancy}
  For a submerged object:
  \begin{center}
    \begin{tabular}{c|c|c|c}
      \rowcolor{pink}
      Densities	&
      $B>W_{\textrm{obj}}$ &
      $B=W_{\textrm{obj}}$ &
      $B<W_{\textrm{obj}}$ \\\hline
      $\rho_{\textrm{obj}}<\rho_{\textrm{fluid}}$ & object rises & float on surface & \\
      $\rho_{\textrm{obj}}=\rho_{\textrm{fluid}}$ & & neutral buoyancy & \\
      $\rho_{\textrm{obj}}>\rho_{\textrm{fluid}}$ & & & object sinks
    \end{tabular}
  \end{center}
\end{frame}


\begin{frame}
  \frametitle{How Submarines Work}
  \framesubtitle{Like this?}
  \begin{center}
    \pic{.7}{EbHMOXk.jpg}
  \end{center}
\end{frame}


\begin{frame}
  \frametitle{How Submarines Work}
  Like most ships, a submarine does not naturally sink because of the buoyance
  force. When a submarine submerges, water needed to be pumped inside
  ``ballast tanks'' to make the ship heavier.
  \begin{center}
    \pic{1}{risinglemur.jpg}
  \end{center}
\end{frame}


\begin{frame}
  \frametitle{Stable? Or unstable?}
  \begin{itemize}
  \item Buoyance force $\mb{B}$ acts at the \emph{center of buoyancy} (CB) of
    the submerged object
    \begin{itemize}
    \item The CB is the CG \emph{if the object has constant density}
    \item The actual CG of the object may be at a different position
    \item Sometimes the object is not fully submerged
    \end{itemize}
  \item Therefore $\mb{F}_g$ and $\mb{B}$ may act at different points, creating
    a torque/moment on the object
  \end{itemize}
  \begin{center}
    \vspace{-.15in}\pic{.6}{stable-unstable.jpg}
  \end{center}
\end{frame}



\begin{frame}
  \frametitle{Fluid Flow}
  Flow of fluid out of a surface requires us to look at the flux function again:
  Volume flux is defined as:
  
  \eq{-.15in}{
    \boxed{\Phi_\mathrm{V}=\int\mb{V}\cdot d\mb{A}}
  }
    
  \vspace{-.1in}where $\mb{V}$ is the velocity (vector field) at the surface,
  and $d\mb{A}$ is the infinitesimal area pointing \textbf{outwards}. We can
  also expressed volume flux using the outward normal unit vector
  $\hat{\mb{n}}$:

  \eq{-.15in}{
    \boxed{\Phi_\mathrm{V}=\int\mb{V}\cdot\hat{\mb{n}}dA}
  }
\end{frame}

\begin{frame}
  \frametitle{Bernoulli Equation}

  \eq{-.01in}{\boxed{
      p_1+\frac{1}{2}\rho v_1^2 + \rho gz_1=
      p_2+\frac{1}{2}\rho v_2^2 + \rho gz_2
  }}
  The term $\displaystyle\frac{1}{2}\rho v^2 $ is called ``dynamic pressure''
\end{frame}


\begin{frame}
  \frametitle{Bernoulli Equation}

  \eq{-.01in}{\boxed{
      p_1+\frac{1}{2}\rho v_1^2 + \rho gz_1=
      p_2+\frac{1}{2}\rho v_2^2 + \rho gz_2
  }}

  Bernoulli's equation is valid when
  \begin{itemize}
  \item the flow is \textbf{steady} (independent of time)
  \item the flow is \textbf{incompressible}--compressibility (i.e. changes in
    density of the fluid) effects are negligible for Mach number $M<0.30$
  \item the flow \textbf{along a single streamline}
  \item there is \textbf{no shaft work} done along the streamline between 1 and
    2
  \item there is \textbf{no heat transfer} along the streamline between 1 and 2
  \end{itemize}
\end{frame}



\begin{frame}
  \frametitle{Bernoulli Equation}

  Regions where Bernoulli equation is valid:
  \begin{center}
    \pic{.8}{bernoulli.jpg}
  \end{center}
\end{frame}



\begin{frame}
  \frametitle{How Does A Wing Work?}

  When air flows past a wing, a force is generated
\end{frame}

\end{document}
