\documentclass[12pt,aspectratio=169]{beamer}

\mode<presentation>
{
  \usetheme{Singapore}
  \setbeamersize{text margin left=.6cm,text margin right=.6cm}
%  \setbeamertemplate{navigation symbols}{} % suppress nav bar
%  \setbeamercovered{transparent}
}
\usefonttheme{professionalfonts}
\usepackage{graphicx}
\usepackage{tikz}
\usepackage{amsmath}
\usepackage{mathpazo}
\usepackage[scaled]{helvet}
\usepackage{xcolor,colortbl}
\usepackage{siunitx}

\sisetup{number-math-rm=\mathnormal}

\title{Class 10: Electrostatics}
\subtitle{AP Physics}
\author[TML]{Dr.\ Timothy Leung}
\institute{Olympiads School}
\date{Januay 2018}

\newcommand{\pic}[2]{\includegraphics[width=#1\textwidth]{#2}}
\newcommand{\mb}[1]{\mathbf{#1}}
%\newcommand{\magdir}[2]{$#1\;[\mathrm{#2}]$}

\begin{document}

\begin{frame}
  \maketitle
\end{frame}


\section[Intro]{Introduction}

\begin{frame}
  \frametitle{Files for You to Download}
  Download from the school website:
  \begin{enumerate}
  \item\texttt{10-electrostatics.pdf}---This presentation. If you want to print
    on paper, I recommend printing 4 pages per side.
  \item\texttt{10-Homework.pdf}---Homework assignment for this class and next
    week.
    The assignment is similar in format compared to the actual AP exams, with
    most questions being multiple-choice questions. The assignment has more
    questions though.
  \end{enumerate}
  
  Please download/print the PDF file before each class. There is no point
  copying notes that are already printed out for you. Instead, take notes on
  things I say that aren't necessarily on the slides.
\end{frame}


\section[$\mb{F}_q$]{Electric Force}

\begin{frame}
  \frametitle{The Charges Are}
  \framesubtitle{Let's Review Some Basics}
  We already know a bit about charge particles:
  \begin{itemize}
  \item A \textbf{proton} carries a \textbf{positive} charge
  \item An \textbf{electron} carries a \textbf{negative} charge
  \item A \emph{net charge} of an object means an excess of protons or electrons
  \item Similar charges are repel, opposite charges attract
  \end{itemize}

  \vspace{0.2in}
  We will start with electrostatics:
  \begin{itemize}
  \item Charges that are not moving relative to one another
  \end{itemize}
\end{frame}

\begin{frame}
  \frametitle{Coulomb's Law for Electrostatic Force}
  The magnitude of the \textbf{electrostatic force} between two point charges
  is given by:
 
  \vspace{-.1in}{\Large
    \begin{displaymath}
      \boxed{F_q=\frac{k\left|q_1q_2\right|}{r^2}}
    \end{displaymath}
  }
  
  \vspace{-.1in}
  \begin{center}
    \begin{tabular}{l|c|l}
      \rowcolor{pink}
      \textbf{Quantity} & \textbf{Symbol} & \textbf{SI Unit} \\ \hline
      Electrostatic force            & $F_q$ &  \si{\newton} (newtons)\\
      Coulomb's constant (electrostatic constant) & $k$   & \si{N.m^2/C^2} \\
      Point charges 1 and 2 (occupies no space) & $q_1$, $q_2$ & \si{C} (coulombs)\\
      Distance between point charges & $r$   & \si{\metre} (metres)\\
    \end{tabular}
  \end{center}

  \vspace{-.1in}
  $\displaystyle k=\frac{1}{4\pi\epsilon_0}=\SI{8.99e9}{N.m^2/C^2}$ where
  $\epsilon_0=\SI{8.85e-12}{C^2/N.m^2}$ is called the
  ``permittivity of free space''
\end{frame}


%\begin{frame}
%  \frametitle{Similarities Between $\mb{F}_g$ to $\mb{F}_q$}
%  \vspace{-0.2in}
%  \begin{columns}
%    \column{0.5\textwidth}
%    \begin{center}
%      Electric Force (point charge):
%      {\LARGE
%        \begin{displaymath}
%          \boxed{F_q=\frac{k\left|q_1q_2\right|}{r^2}}
%        \end{displaymath}
%      }
%    \end{center}
%    \column{0.5\textwidth}
%    \begin{center}
%      Gravitational Force (point mass):
%      {\LARGE
%        \begin{displaymath}
%          \boxed{F_g=\frac{Gm_1m_2}{r^2}}
%        \end{displaymath}
%      }
%    \end{center}
%  \end{columns}
%  \begin{itemize}
%  \item Similarities:
%    \begin{itemize}
%    \item Both inversely proportional to $r^2$
%    \item Both are scaled by a constant
%      \end{itemize}
%  \item Difference:
%    \begin{itemize}
%    \item For gravity only positive mass
%    \item Electric charge can be either positive or negative
%    \end{itemize}
%  \end{itemize}
%\end{frame}

\section[$\mb{E}$]{Electric Field}

\begin{frame}
  \frametitle{Think Electric Field}
  We can get \textbf{electric field} by repeating the same procedure as with
  gravitational field. Again, let's group the variables in Coulomb's equation:

  \vspace{-0.2in}{\Large
    \begin{displaymath}
      F_q=\underbrace{\left[\frac{kq_1}{r^2}\right]}_{=E}q_2
    \end{displaymath}
  }

  \vspace{-0.1in}
  We can say that charge $q_1$ creates an ``electric field'' ($E$) with an
  intensity

  \vspace{-0.2in}{\Large
    \begin{displaymath}
      E=\frac{kq_1}{r^2}
    \end{displaymath}
  }

  \vspace{-0.1in}
  This electric field $\mb{E}$ created by $q_1$ is a function (``vector field'')
  that shows how it influences other charged particles around it
\end{frame}



\begin{frame}
  \frametitle{Electric Field Intensity Near a Point Charge}
  The electric field intensity a distance $r$ away from a point charge is
  the product of Coulomb's constant and the charge, divided by the square
  of the distance from the charge.
  \textbf{The direction of the field is radially outward from a positive
    point charge and radially inward towards a negative charge.}

  \vspace{-0.2in}
  {\Large
    \begin{displaymath}
      \boxed{E=\frac{kq_s}{r^2}}
    \end{displaymath}
  }
  \begin{center}
    \begin{tabular}{l|c|l}
      \rowcolor{pink}
      \textbf{Quantity} & \textbf{Symbol} & \textbf{SI Unit} \\ \hline
      Electric field intensity    & $E$   & \si{N/C} (newtons per coulomb)\\
      Coulomb's constant          & $k$   & \si{N.m^2/C^2} \\
      Source charge               & $q_s$ & \si{C} (coulombs) \\
      Distance from source charge & $r$   & \si{m} (metres)  \\
    \end{tabular}
  \end{center}
\end{frame}



\begin{frame}
  \frametitle{Think Electric Field}

  $\mb{E}$ \emph{doesn't do anything} until another charge interacts with it.
  And when there is a charge $q$, the electric force $\mb{F}_q$ that it
  experiences in the presence of $\mb{E}$ is:

  \vspace{-.2in}{\Large
    \begin{displaymath}
      \boxed{\mb{F}_q=\mb{E}q}
    \end{displaymath}
  }

  \vspace{-.1in}$\mb{F}_q$ and $\mb{E}$ are vectors, and following the
  principle of superposition, i.e.

  \vspace{-.45in}{\Large
    \begin{align*}
      \mb{F}_q&=\mb{F}_1+\mb{F}_2+\mb{F}_3+\mb{F}_4\ldots\\
      \mb{E}&=\mb{E}_1+\mb{E}_2+\mb{E}_3+\mb{E}_4\ldots
    \end{align*}
  }
  
  \vspace{-.3in}This understanding is especially important when we want to find
  $\mb{F}_q$ and $\mb{E}$ some distance from a continuous distribution of
  charges
\end{frame}


\begin{frame}
  \frametitle{Electric Field Lines}
  If you place a positive charge in an electric field, the force on the charge
  will be in the direction of the electric field.
  \begin{columns}
    \column{0.25\textwidth}
    \pic{.95}{pos_charge.png}\\
    \pic{.95}{neg_charge.png}
    \column{0.75\textwidth}
    \pic{.95}{2charges.png}
  \end{columns}
\end{frame}




%\begin{frame}
%  \frametitle{Example Problem}
%  \textbf{Example 2:} What is the electric field intensity at a point
%  \SI{30.0}{cm} from the centre of a small sphere that has a positive charge of
%  \SI{2.0e-6}{C}? ($k=\SI{8.99e9}{N.m^2/C^2}$)
%\end{frame}

%\begin{frame}
%  \frametitle{Example Problem}
%  \textbf{Example 3:} Three charges, A(\SI{+6.0}{\micro\coulomb}),
%  B(\SI{-5.0}{\micro\coulomb}), and C(\SI{+6.0}{\micro\coulomb}), are located at
%  the corners of a square with sides that are \SI{5.0}{cm} long. What is the
%  electric field intensity at point D?
%  \begin{center}
%    \pic{0.55}{graphics/charges-square.png}
%  \end{center}
%\end{frame}
%
%\begin{frame}
%  \frametitle{Example Problem (Cont.)}
%  \begin{center}
%    \pic{0.5}{graphics/charges-square-forces.png}
%  \end{center}
%\end{frame}



\section[$U_q$ and $V$]{Electric Potential \& Potential Energy}


\begin{frame}
  \frametitle{Electrical Potential Energy}
  \framesubtitle{(Follow the Same Work on Gravitational Potential Energy)}

  If we move a charged particle against the electric force, work must be done
  (either positive or negative, depending on which way the particle moves):
  
  \vspace{-.3in}{\Large
    \begin{displaymath}
      W=\int\mb{F}_q\cdot d\mb{s}
      =kq_1q_2\int_{r_1}^{r_2}\frac{dr}{r^2}
      =-\frac{kq_1q_2}{r}=-\Delta U_q
    \end{displaymath}
  }

  \vspace{-.1in}\textbf{Electrical potential energy} is defined as:
    
  \vspace{-.2in}{\Large
    \begin{displaymath}
      \boxed{U_q=\frac{kq_1q_2}{r}}
    \end{displaymath}
  }

  \vspace{-.1in}$U_q$ can be positive or negative, because charged particles
  can be either positive or negative
\end{frame}




\begin{frame}
  \frametitle{How it Differs from Gravitational Potential}
  \begin{columns}
    \column{0.33\textwidth}
    \begin{center}
      Two positive charges:

      \vspace{-0.3in}{\Large
        \begin{displaymath}
          U_q>0
        \end{displaymath}
      }
    \end{center}
    \column{0.33\textwidth}
    \begin{center}
      Two negative charges:

      \vspace{-0.3in}{\Large
        \begin{displaymath}
          U_q>0
        \end{displaymath}
      }
    \end{center}
    \column{0.34\textwidth}
    \begin{center}
      One positive and one negative charge:

      \vspace{-0.5in}{\Large
        \begin{displaymath}
          U_q<0
        \end{displaymath}
      }
    \end{center}
  \end{columns}

  \vspace{.4in}
  \begin{itemize}
  \item $U_q>0$ means positive work is done to bring two charges together from
   $r=\infty$ to $r$ (both charges of the same sign)
  \item $U_q<0$ means negative work (the charges are opposite signs)
  \item For gravitational potential $U_g$ is always $<0$
  \end{itemize}
\end{frame}

\begin{frame}
  \frametitle{Electric Potential}
  \begin{itemize}
  \item When I move an object of mass $m$ against a gravitational force from
    one point to another, the work that I do is directly proportional to $m$
  \item i.e.\ there is a ``constant'' in that scales with \emph{any} mass, as
    long as they move between those same two points
  \item In the trivial case (small changes in height, no change in $g$), it is
    just

    \vspace{-.2in}{\Large
      \begin{displaymath}
        \frac{\Delta U_q}{m}=g\Delta h
      \end{displaymath}
    }
  \end{itemize}
  (We have actually looked at this briefly in our discussion on universal
  gravitation.)
\end{frame}


\begin{frame}
  \frametitle{Electric Potential}

  This is also true for moving a charged particle against an electric force,
  and the constant is called the \textbf{electric potential}. For a point
  charge, it is defined as

  \vspace{-.2in}{\Large
    \begin{displaymath}
      \boxed{V=\frac{U_q}{q}=\frac{kq}{r}}
    \end{displaymath}
  }

  The unit for electric potential is a \emph{volt} which is
  \emph{one joule per coulomb}:

  \vspace{-.25in}{\Large
    \begin{displaymath}
      \SI{1}{V}=\SI{1}{J/C}
    \end{displaymath}
  }
\end{frame}


\begin{frame}
  \frametitle{Electrical Potential}
  
  We can easily that there is also a relationship between electrical potential
  $V$ and electrical potential energy:
  
  \vspace{-.2in}{\Large
    \begin{displaymath}
      \boxed{\Delta V=-\int\mb{E}\cdot d\mb{s}}
    \end{displaymath}
  }
\end{frame}



\begin{frame}
  \frametitle{Potential Difference (Voltage)}

  The change in electric potential is called the
  \textbf{potential difference} or \textbf{voltage}:

  \vspace{-.2in}{\Large
    \begin{displaymath}
      \boxed{\Delta V=\frac{\Delta U_q}{q}}\quad\textsf{\normalsize and}\quad
      \boxed{dV=\frac{dU_q}{q}}
    \end{displaymath}
  }

  Here, we can relate $\Delta V$ to an equation that we knew from Physics 11,
  which related to the energy dissipated in a resistor in a circuit
  $\Delta U$ to the voltage drop $\Delta V$:
    
  \vspace{-.2in}{\Large\[\boxed{\Delta U=q\Delta V}\]}

  Electric potential difference also has the unit \emph{volts} (\si{V})
\end{frame}



\begin{frame}
  \frametitle{Getting Those Names Right}
  Remember that these three quantities are all scalars, as opposed to electric
  force $\mb{F}_q$ and electric field $\mb{E}$ which are vectors

  \vspace{.1in}
  \begin{itemize}
  \item Electric potential energy:
    \begin{displaymath}
      U=-\frac{kq_1q_2}{r}
    \end{displaymath}
  \item Electric potential:
    \begin{displaymath}
      V=\frac{kq}{r}
    \end{displaymath}
  \item Electric potential difference (voltage):
    \begin{displaymath}
      \Delta V=\frac{\Delta U_q}{q}
    \end{displaymath}
  \end{itemize}
\end{frame}
%\begin{frame}
%  \frametitle{Potential Difference and Potential Energy}
%  \begin{columns}
%    \column{0.48\textwidth}
%    \begin{center}
%      Electric Potential Energy
%    \end{center}
%    {\Large
%      \begin{displaymath}
%        \boxed{U_q=\frac{kq_1q_2}{r}}
%      \end{displaymath}
%    }
%    \column{0.48\textwidth}
%    \begin{center}
%      Electric Potential Difference
%    \end{center}
%    {\Large
%      \begin{displaymath}
%        \boxed{V=\frac{kq}{r}}
%      \end{displaymath}
%    }
%  \end{columns}
%  \begin{itemize}
%  \item Two very similar expressions
%  \item Analogy: Carrying a bucket of water up a mountain
%    \begin{itemize}
%    \item $U_q$ is the work that I have to do
%    \item $V$ is the height of the mountain
%    \end{itemize}
%  \end{itemize}
%\end{frame}

\begin{frame}
  \frametitle{Relating $U_q$, $\mb{F}_q$ and $\mb{E}$}
  \framesubtitle{Our Integrals In Reverse}
  \begin{itemize}
  \item Using vector calculus, we can relate electric force ($\mb{F}_q$) to
    electrical potential energy ($U_q$), and electric field ($\mb{E}$) to the
    electric potential ($V$):

    \vspace{-.1in}{\Large
      \begin{displaymath}
        \mb{F}_q=-\nabla U_q=-\frac{\partial U_q}{\partial r}\hat{\mb{r}}
        \quad\;\;
        \mb{E}=-\nabla V=-\frac{\partial V}{\partial r}\hat{\mb{r}}
      \end{displaymath}
    }
  
  \item\vspace{-.2in}Electric force $\mb{F}_q$ always points from high
    potential to low potential energy
  \item Notice that electric field can also be expressed as the change of
    electric potential per unit distance, which has the unit
    
    \vspace{-.1in}\begin{displaymath}
      \SI{1}{N/C}=\SI{1}{V/m}
    \end{displaymath}
    Electric field is sometimes also called ``potential gradient''
  \end{itemize}
\end{frame}



%\begin{frame}
%  \frametitle{Example Problem}
%  \textbf{Example 4:} A small sphere with a charge of \SI{-3.0}{\micro\coulomb}
%  creates an electric field. Calculate the electric potential difference at
%  point A, located \SI{2.0}{cm} from the source charge, and at point B, located
%  \SI{5.0}{cm} from the same source charge. Which point is at higher potential?
%
%  \begin{center}
%    \pic{0.5}{graphics/neg-potential.png}
%  \end{center}
%\end{frame}
%
%\begin{frame}
%  \frametitle{Example Problem}
%  \textbf{Example 5:} The diagram shows three charges, A(\SI{5.0}{\micro C}),
%  B(\SI{-7.0}{\micro C}), and C(\SI{2.0}{\micro C}), placed at three corners of
%  a rectangle.
%  Point D is the fourth corner. What is the electric potential difference at
%  point D?
%  \begin{center}
%    \pic{0.6}{graphics/multi-potential.png}
%  \end{center}
%\end{frame}



%\section{Field Structure}
%
%\begin{frame}
%  \frametitle{Field Structure}
%  How should we draw the field lines?
%  \begin{center}
%    \pic{.3}{graphics/plate1.png}
%    \pic{.3}{graphics/plate2.png}
%  \end{center}
%\end{frame}
%
\begin{frame}
  \frametitle{Equipotential Lines}
  \begin{center}
    \pic{0.65}{plate3.png}
  \end{center}

  \vspace{-.2in}The dotted blue lines are called \textbf{equipotential lines}.
  They are always \emph{perpendicular} to the electric field lines. Charges
  moving in the direction of the equipotential lines have constant electric
  potential
\end{frame}

\begin{frame}
  \frametitle{Electric Field between Two Parallel Plates}
  \begin{center}
    \pic{.6}{elfield-600x205.jpg}
  \end{center}
  
  \vspace{-.2in}
  \begin{itemize}
  \item $\mb{E}$ is uniform at all points between the
    parallel plates, independent of position
  \item $E$ is proportional to the charge density (charge per unit
    area) on the plates:

    \vspace{-.2in}{\Large
      \begin{displaymath}
        E\propto\sigma\quad\textsf{\normalsize where}\quad
        \sigma=\frac{\sigma}{A}
      \end{displaymath}
    }
  \item $E$ outside the plates is very low (close to zero), except for
    the fringe effects at the edges of the plates. 
  \end{itemize}
\end{frame}

%\begin{frame}
%  \frametitle{Example Problem}
%  \textbf{Example 6}: An identical pair of metal plates is mounted parallel on
%  insulating stands \SI{20}{cm} apart and equal amounts of opposite charges are
%  placed on the plates. The electric field intensity at the midpoint between the
%  plates is \SI{400}{N/C}.
%  \begin{enumerate}
%  \item What is the electric field intensity at a point \SI{5.0}{cm} from the
%    positive plate?
%  \item If the same amount of charge was placed on plates that have twice the
%    area and are \SI{20}{cm} apart, what would be the electric field intensity
%    at the point \SI{5.0}{cm} from the positive plate?
%  \item What would be the electric field intensity of the original plates if
%    the distance of separation of the plates was doubled?
%  \end{enumerate}
%\end{frame}

\begin{frame}
  \frametitle{Electric Field and Potential Difference}
  
  The relationship between electric field ($\mb{E}$) and electric potential
  difference ($V$):
    
  \vspace{-.15in}{\Large
    \begin{displaymath}
      \mb{E}=-\frac{\partial V}{\partial r}
    \end{displaymath}
  }

  In a uniform electric field (e.g.\ parallel plate) it simplifies to a very
  simple equation:

  \vspace{-.2in}{\Large
    \begin{displaymath}
      \boxed{E=\frac{\Delta V}{d}}
    \end{displaymath}
  }

  \vspace{-.1in}
  \begin{center}
    \begin{tabular}{l|c|l}
      \rowcolor{pink}
      \textbf{Quantity} & \textbf{Symbol} & \textbf{SI Unit} \\ \hline
      Electric field intensity & $E$ & \si{N/C} (newtons per coulomb) \\
      Potential difference between plates & $\Delta V$ & \si{V} (volts) \\
      Distance between plates       & $d$ & \si{m} (metres)\\
    \end{tabular}
  \end{center}
\end{frame}

%\begin{frame}
%  \frametitle{Example Problem}
%  \textbf{Example 7:} Two parallel plates \SI{5.0}{cm} apart are oppositely
%  charged. The electric potential difference across the plates is \SI{80.0}{V}.
%  \begin{itemize}
%  \item What is the electric field intensity between the plates?
%  \item What is the potential difference at point A?
%  \item What is the potential difference at point B?
%  \item What is the potential difference between points A and B?
%  \item What force would be experienced by a small \SI{2.0}{\micro C} charge
%    placed at point A?
%  \end{itemize}
%\end{frame}


\end{document}
