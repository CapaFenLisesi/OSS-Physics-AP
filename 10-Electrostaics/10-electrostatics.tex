\documentclass[12pt,aspectratio=169]{beamer}

\mode<presentation>
{
  \usetheme{Singapore}
  \setbeamersize{text margin left=.6cm,text margin right=.6cm}
%  \setbeamertemplate{navigation symbols}{} % suppress nav bar
%  \setbeamercovered{transparent}
}
\usefonttheme{professionalfonts}
\usepackage{graphicx}
\usepackage{tikz}
\usepackage{amsmath}
\usepackage{mathpazo}
\usepackage[scaled]{helvet}
\usepackage{xcolor,colortbl}
\usepackage{siunitx}

\sisetup{number-math-rm=\mathnormal}

\title{Class 10: Electrostatics}
\subtitle{AP Physics}
\author[TML]{Dr.\ Timothy Leung}
\institute{Olympiads School}
\date{Januay 2018}

\newcommand{\pic}[2]{\includegraphics[width=#1\textwidth]{#2}}
\newcommand{\mb}[1]{\mathbf{#1}}
\newcommand{\eq}[2]{\vspace{#1}{\Large\begin{displaymath}#2\end{displaymath}}}


\begin{document}

\begin{frame}
  \maketitle
\end{frame}


\section[Intro]{Introduction}

\begin{frame}
  \frametitle{Files for You to Download}
  Download from the school website:
  \begin{enumerate}
  \item\texttt{10-electrostatics.pdf}---This presentation. If you want to print
    on paper, I recommend printing 4 pages per side.
%  \item\texttt{10-Homework.pdf}---Homework assignment for this class and next
%    week.
%    The assignment is similar in format compared to the actual AP exams, with
%    most questions being multiple-choice questions. The assignment has more
    %    questions though.

  \item There is no assignment this week. It will be combined into a larger one
    for next week, when we deal with distributed charges and capacitors.
  \end{enumerate}
  
  Please download/print the PDF file before each class. There is no point
  copying notes that are already printed out for you. Instead, take notes on
  things I say that aren't necessarily on the slides.
\end{frame}


\section[$\mb{F}_q$]{Electric Force}

\begin{frame}
  \frametitle{The Charges Are}
  \framesubtitle{Let's Review Some Basics}
  We already know a bit about charge particles:
  \begin{itemize}
  \item A \textbf{proton} carries a \textbf{positive} charge
  \item An \textbf{electron} carries a \textbf{negative} charge
  \item A \emph{net charge} of an object means an excess of protons or electrons
  \item Similar charges are repel, opposite charges attract
  \end{itemize}

  \vspace{0.2in}
  We will start with electrostatics:
  \begin{itemize}
  \item Charges that are not moving relative to one another
  \end{itemize}
\end{frame}

\begin{frame}
  \frametitle{Coulomb's Law for Electrostatic Force}
  The magnitude of the \textbf{electrostatic force} between two point charges
  is given by:

  \eq{-.1in}{
    \boxed{F_q=\frac{k\left|q_1q_2\right|}{r^2}}
  }

  \vspace{-.1in}
  \begin{center}
    \begin{tabular}{l|c|l}
      \rowcolor{pink}
      \textbf{Quantity} & \textbf{Symbol} & \textbf{SI Unit} \\ \hline
      Electrostatic force            & $F_q$ &  \si{\newton} (newtons)\\
      Coulomb's constant (electrostatic constant) & $k$   & \si{N.m^2/C^2} \\
      Point charges 1 and 2 (occupies no space) & $q_1$, $q_2$ & \si{C} (coulombs)\\
      Distance between point charges & $r$   & \si{\metre} (metres)\\
    \end{tabular}
  \end{center}

  \vspace{-.1in}
  $\displaystyle k=\frac{1}{4\pi\epsilon_0}=\SI{8.99e9}{N.m^2/C^2}$ where
  $\epsilon_0=\SI{8.85e-12}{C^2/N.m^2}$ is called the
  ``permittivity of free space''
\end{frame}


\section[$\mb{E}$]{Electric Field}

\begin{frame}
  \frametitle{Think Electric Field}
  We can get \textbf{electric field} by repeating the same procedure as with
  gravitational field. Again, let's group the variables in Coulomb's equation:

  \eq{-.2in}{
    F_q=\underbrace{\left[\frac{kq_1}{r^2}\right]}_{=E}q_2
  }

  \vspace{-0.1in}
  We can say that charge $q_1$ creates an ``electric field'' ($E$) with an
  intensity

  \eq{-0.2in}{
    E=\frac{kq_1}{r^2}
  }

  \vspace{-0.1in}
  This electric field $\mb{E}$ created by $q_1$ is a function (``vector field'')
  that shows how it influences other charged particles around it
\end{frame}



\begin{frame}
  \frametitle{Electric Field Intensity Near a Point Charge}
  The electric field intensity a distance $r$ away from a point charge is
  the product of Coulomb's constant and the charge, divided by the square
  of the distance from the charge.
  \textbf{The direction of the field is radially outward from a positive
    point charge and radially inward towards a negative charge.}

  \eq{-0.2in}{
    \boxed{E=\frac{kq_s}{r^2}}
  }
  \begin{center}
    \begin{tabular}{l|c|l}
      \rowcolor{pink}
      \textbf{Quantity} & \textbf{Symbol} & \textbf{SI Unit} \\ \hline
      Electric field intensity    & $E$   & \si{N/C} (newtons per coulomb)\\
      Coulomb's constant          & $k$   & \si{N.m^2/C^2} \\
      Source charge               & $q_s$ & \si{C} (coulombs) \\
      Distance from source charge & $r$   & \si{m} (metres)  \\
    \end{tabular}
  \end{center}
\end{frame}



\begin{frame}
  \frametitle{Think Electric Field}

  $\mb{E}$ \emph{doesn't do anything} until another charge interacts with it.
  And when there is a charge $q$, the electric force $\mb{F}_q$ that it
  experiences in the presence of $\mb{E}$ is:

  \eq{-.2in}{
    \boxed{\mb{F}_q=\mb{E}q}
  }

  \vspace{-.1in}$\mb{F}_q$ and $\mb{E}$ are vectors, and following the
  principle of superposition, i.e.

  \vspace{-.45in}{\Large
    \begin{align*}
      \mb{F}_q&=\mb{F}_1+\mb{F}_2+\mb{F}_3+\mb{F}_4\ldots\\
      \mb{E}&=\mb{E}_1+\mb{E}_2+\mb{E}_3+\mb{E}_4\ldots
    \end{align*}
  }
  
  \vspace{-.3in}This understanding is especially important when we want to find
  $\mb{F}_q$ and $\mb{E}$ some distance from a continuous distribution of
  charges
\end{frame}


\begin{frame}
  \frametitle{Electric Field Lines}
  If you place a positive charge in an electric field, the force on the charge
  will be in the direction of the electric field.
  \begin{columns}
    \column{0.25\textwidth}
    \pic{.95}{pos_charge.png}\\
    \pic{.95}{neg_charge.png}
    \column{0.75\textwidth}
    \pic{.95}{2charges.png}
  \end{columns}
\end{frame}


\section[$U_q$ and $V$]{Electric Potential \& Potential Energy}

\begin{frame}
  \frametitle{Electrical Potential Energy}
  \framesubtitle{(Follow the Same Work on Gravitational Potential Energy)}

  If we move a charged particle against the electric force, work must be done
  (either positive or negative, depending on which way the particle moves):
  
  \eq{-.3in}{
    W=\int\mb{F}_q\cdot d\mb{s}
    =kq_1q_2\int_{r_1}^{r_2}\frac{dr}{r^2}
    =-\frac{kq_1q_2}{r}\Big|^{r_2}_{r_1}=-\Delta U_q
  }

  \vspace{-.1in}\textbf{Electrical potential energy} is defined as:
    
  \eq{-.2in}{
    \boxed{U_q=\frac{kq_1q_2}{r}}
  }

  \vspace{-.1in}$U_q$ can be (+) or (-), because charged particles can be
  either (+) or (-)
\end{frame}




\begin{frame}
  \frametitle{How it Differs from Gravitational Potential}
  \begin{columns}
    \column{0.33\textwidth}
    \begin{center}
      Two positive charges:

      \eq{-0.3in}{U_q>0}
    \end{center}
    \column{0.33\textwidth}
    \begin{center}
      Two negative charges:

      \eq{-0.3in}{U_q>0}
    \end{center}
    \column{0.34\textwidth}
    \begin{center}
      One positive and one negative charge:

      \eq{-0.5in}{U_q<0}
    \end{center}
  \end{columns}

  \vspace{.2in}
  \begin{itemize}
  \item $U_q>0$ means positive work is done to bring two charges together from
   $r=\infty$ to $r$ (both charges of the same sign)
  \item $U_q<0$ means negative work (the charges are opposite signs)
  \item For gravitational potential $U_g$ is always $<0$
  \end{itemize}
\end{frame}

\begin{frame}
  \frametitle{Electric Potential}
  \framesubtitle{Start with an Analogy}

  When I move an object of mass $m$ against a gravitational force from one
  point to another, the work that I do is directly proportional to $m$, i.e.\
  there is a ``constant'' in that scales with \emph{any} mass, as long as they
  move between those same two points:

  \eq{-.25in}{W=Km}

  \vspace{-.15in}In the trivial case (small changes in height, no change in
  $g$), this constant is just

  \eq{-.15in}{
    \frac{W}{m}=g\Delta h
  }

  \vspace{-.1in}(We have actually looked at this briefly in our discussion on
  universal gravitation.)
\end{frame}


\begin{frame}
  \frametitle{Electric Potential}

  This is also true for moving a charged particle against an electric force,
  and the constant is called the \textbf{electric potential}. For a point
  charge, it is defined as

  \eq{-.2in}{
    \boxed{V=\frac{U_q}{q}=\frac{kq}{r}}
  }

  The unit for electric potential is a \emph{volt} which is
  \emph{one joule per coulomb}:

  \eq{-.25in}{
    \SI{1}{V}=\SI{1}{J/C}
  }

  \vspace{-.15in}We can easily that there is also a relationship between
  electrical potential $V$ and electrical potential energy:
  
  \eq{-.18in}{
    \boxed{\Delta V=-\int\mb{E}\cdot d\mb{s}}
  }
\end{frame}



\begin{frame}
  \frametitle{Potential Difference (Voltage)}

  The change in electric potential is called the
  \textbf{potential difference} or \textbf{voltage}:

  \eq{-.2in}{
    \boxed{\Delta V=\frac{\Delta U_q}{q}}\quad\textsf{\normalsize and}\quad
    \boxed{dV=\frac{dU_q}{q}}
  }

  Here, we can relate $\Delta V$ to an equation that we knew from Physics 11,
  which related to the energy dissipated in a resistor in a circuit
  $\Delta U$ to the voltage drop $\Delta V$:
    
  \eq{-.2in}{
    \boxed{\Delta U=q\Delta V}
  }

  Electric potential difference also has the unit \emph{volts} (\si{V})
\end{frame}



\begin{frame}
  \frametitle{Getting Those Names Right}
  Remember that these three quantities are all scalars, as opposed to electric
  force $\mb{F}_q$ and electric field $\mb{E}$ which are vectors

  \vspace{.1in}
  \begin{itemize}
  \item Electric potential energy:
    \begin{displaymath}
      U=-\frac{kq_1q_2}{r}
    \end{displaymath}
  \item Electric potential:
    \begin{displaymath}
      V=\frac{kq}{r}
    \end{displaymath}
  \item Electric potential difference (voltage):
    \begin{displaymath}
      \Delta V=\frac{\Delta U_q}{q}
    \end{displaymath}
  \end{itemize}
\end{frame}


\begin{frame}
  \frametitle{Relating $U_q$, $\mb{F}_q$ and $\mb{E}$}
  \framesubtitle{Our Integrals In Reverse}
  Using vector calculus, we can relate electric force ($\mb{F}_q$) to
  electric potential energy ($U_q$), and electric field ($\mb{E}$) to the
  electric potential ($V$):

  \eq{-.2in}{
    \mb{F}_q=-\nabla U_q=-\frac{\partial U_q}{\partial r}\hat{\mb{r}}
    \quad\;\;
    \mb{E}=-\nabla V=-\frac{\partial V}{\partial r}\hat{\mb{r}}
  }

  \begin{itemize}  
  \item\vspace{-.2in}Electric force $\mb{F}_q$ always points from high
    potential to low potential energy
  \item Electric field can also be expressed as the change of electric
    potential per unit distance, which has the unit
    
    \eq{-.25in}{
      \SI{1}{N/C}=\SI{1}{V/m}
    }
  \item Electric field is also called ``potential gradient''
  \end{itemize}
\end{frame}



\begin{frame}
  \frametitle{Equipotential Lines}
  \begin{center}
    \pic{0.65}{plate3.png}
  \end{center}

  \vspace{-.2in}The dotted blue lines are called \textbf{equipotential lines}.
  They are always \emph{perpendicular} to the electric field lines. Charges
  moving in the direction of the equipotential lines have constant electric
  potential
\end{frame}

\begin{frame}
  \frametitle{Electric Field between Two Parallel Plates}
  \begin{center}
    \pic{.6}{elfield-600x205.jpg}
  \end{center}
  
  \vspace{-.2in}
  \begin{itemize}
  \item $\mb{E}$ is uniform at all points between the
    parallel plates, independent of position
  \item $E$ is proportional to the charge density (charge per unit
    area) on the plates:

    \eq{-.2in}{
      E\propto\sigma\quad\textsf{\normalsize where}\quad
      \sigma=\frac{\sigma}{A}
    }
  \item $E$ outside the plates is very low (close to zero), except for
    the fringe effects at the edges of the plates. 
  \end{itemize}
\end{frame}



\begin{frame}
  \frametitle{Electric Field and Potential Difference}
  
  The relationship between electric field ($\mb{E}$) and electric potential
  difference ($V$):
    
  \eq{-.15in}{
    \mb{E}=-\frac{\partial V}{\partial r}
  }

  In a uniform electric field (e.g.\ parallel plate) it simplifies to a very
  simple equation:

  \eq{-.2in}{
    \boxed{E=\frac{\Delta V}{d}}
  }

  \vspace{-.1in}
  \begin{center}
    \begin{tabular}{l|c|l}
      \rowcolor{pink}
      \textbf{Quantity} & \textbf{Symbol} & \textbf{SI Unit} \\ \hline
      Electric field intensity & $E$ & \si{N/C} (newtons per coulomb) \\
      Potential difference between plates & $\Delta V$ & \si{V} (volts) \\
      Distance between plates       & $d$ & \si{m} (metres)\\
    \end{tabular}
  \end{center}
\end{frame}

\end{document}
