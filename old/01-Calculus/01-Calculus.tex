\documentclass[12pt,compress,aspectratio=169]{beamer}

\mode<presentation>
{
  \usetheme{Singapore}
  \setbeamersize{text margin left=.5cm,text margin right=.5cm}
%  \setbeamertemplate{navigation symbols}{} % suppress nav bar
%  \setbeamercovered{transparent}
}
\usefonttheme{professionalfonts}
\usepackage{amsmath}
\usepackage{siunitx}
%\usepackage{graphicx}
%\usepackage{tikz}
\usepackage{mathpazo}
\usepackage[scaled]{helvet}
%\usepackage{xcolor,colortbl}
%\usepackage{hyperref}

\title{1.\ Calculus in Physics}
\subtitle{AP Physics}
\author[TML]{Dr.\ Timothy Leung}
\institute{Olympiads School}
\date{Fall 2017}

\newcommand{\pic}[2]{\includegraphics[width=#1\textwidth]{#2}}
%\newcommand{\e}[1]{\times 10^{#1}}
%%\renewcommand\vec{\mathbf}
\newcommand{\mb}[1]{\ensuremath\mathbf{#1}}

\begin{document}

\begin{frame}
  \frametitle{}
  {\LARGE
    \begin{center}
      \textbf{WELCOME TO AP PHYSICS}
    \end{center}
  }
\end{frame}

\section[Intro]{Introduction}

\subsection[Tim]{About Tim}
\begin{frame}
  \frametitle{Hi, My Name is Tim}
  \begin{columns}
    \column{0.3\textwidth}
    \pic{1}{tim.jpg}
    \column{0.7\textwidth}
    \begin{itemize}
    \item B.A.Sc.\ in Engineering Physics (UBC)
      \begin{itemize}
      \item Won the Roy Nodwell Prize for my design of a solar car
      \end{itemize}
    \item M.A.Sc.\ and Ph.D.\ in Aerospace Engineering (UTIAS)
      \begin{itemize}
      \item ``Computational Fluid Dynamics'' (CFD)
      \item ``Aerodynamic shape optimization''
      \item Aircraft design
      \end{itemize}
    \item Also spent a year in Vancouver as a professional violinist\ldots
    \end{itemize}
  \end{columns}
\end{frame}

\begin{frame}
  \frametitle{Tim's Past Research Work}
  \begin{center}
    \pic{0.25}{view-grid-original-1024x899.jpg}\hspace{0.01in}
    \pic{0.25}{view-grid-final-1024x899.jpg}\hspace{0.01in}
    \pic{0.225}{flowsolution-original-1024x1000.jpg}\hspace{0.01in}
    \pic{0.225}{flowsolution-deformed-1024x1000.jpg}
  \end{center}
\end{frame}

%\begin{frame}
%  \frametitle{What This Course Looks Like}
%  \begin{itemize}
%  \item Sixteen $2\frac{1}{2}$-hour classes % each week, 16 classes total
%  \item During class time we go over the theory and concepts, and we solve
%    example problems
%    \begin{itemize}
%    \item It's very important that you bring your calculator and follow along
%%    \item Sit back, relax\ldots, take notes and \emph{think}
%    \end{itemize}
%  \item A lot of what you learn will be through problem solving (i.e.\ homework)
%  \item Homework assignments downloadable from school website at the start of
%    each unit; due 1 class after we complete the unit % later %2 weeks later
%    \begin{itemize}
%    \item We will also go over \emph{some} homework questions in class
%    \end{itemize}
%  \end{itemize}
%\end{frame}

\begin{frame}
  \frametitle{Classroom Rules}
  \begin{itemize}
  \item Treat me and each other with respect, and I'll treat you like an adult
  \item If you need to go to the bathroom, you don't need my permission
  \item Raise your hands if you have a question. Don't wait too long
  \item \emph{``There is no such thing as a stupid question''}
  \item E-mail me at \textbf{\texttt{tim@timleungjr.ca}} for any questions
    related to physics and math and engineering
  \item Do \textbf{\emph{not}} try to find me on social media
  \end{itemize}
\end{frame}

\begin{frame}
  \titlepage
\end{frame}

\begin{frame}
  \frametitle{Files for You to Download}
  \begin{itemize}
  %\item\texttt{00-courseOutline.pdf}--The course outline
  \item\texttt{01-Calculus-2x2.pdf}--The slides that I am using right now
  \item\texttt{01-Homework.pdf}--This week's homework assignment
  \end{itemize}
  Please download/print the PDF file for the class slides before each class.
  There is no point copying notes that are already printed out for you.
  Instead, take notes on things I say that aren't necessarily on the slides.
\end{frame}

\begin{frame}
  \frametitle{Invented for Physics}
  \framesubtitle{Thanks, Issac Newton}
  \begin{itemize}
  \item We cannot learn physics properly without calculus (you have got
    away with it long enough in Grade 11 and 12 Physics classes\ldots)
  \item Differential calculus was ``invented'' so that we can understand
    motion, especially on non-constant velocity and acceleration.
  \item If you are taking calculus, you may have noticed that a lot of the
    word problems are really physics problems
  \end{itemize}
\end{frame}

\begin{frame}
  \frametitle{Differentiation and Integration}
  %There are two parts to calculus, differentiation and integration:
  \begin{itemize}
  \item\textbf{Differential Calculus}
    \begin{itemize}
    \item Finding how quickly something is changing (``rate of change'' of a
      quantity)
    \item Math: slopes of functions
    \item Physics: how quickly a physical quantity is changing in time and/or
      space
    \item Examples: velocity (how quickly position changes), acceleration
      (how quickly velocity changes), power (how quickly work is done)
    \end{itemize}
  \item\textbf{Integral Calculus}
    \begin{itemize}
    \item The opposite of differentiation
    \item We use it to compute the area under a curve, or
    \item Summation of many small terms
    \item Examples: area under the $\mb{v}$-$t$ graph (to calculate
      displacement), area under the $\mb{F}$-$t$ graph (to calculate impulse),
      area under the $F$-$d$ graph (to calculate work)
    \end{itemize}
  \end{itemize}
\end{frame}


\section{Differentiation}

\begin{frame}
  \frametitle{}
  \begin{center}
    {\Large\textbf{FIRST, WE LOOK AT DIFFERENTIATION}
    }
  \end{center}
\end{frame}

\subsection{Kinematics}

\begin{frame}
  \frametitle{Velocity, Time Derivative of Displacement}
  \begin{columns}
    \column{.3\textwidth}
    \pic{1}{speedometer.jpg}
    \column{0.62\textwidth}
    Suppose the motion of a car is governed by the equation:

      \vspace{-.4in}{\Large
        \begin{displaymath}
          s(t)=3t^2
        \end{displaymath}
      }
      
      \vspace{-0.2in}where $s$ is the car's position along a straight path at
      time $t$. \textbf{What is its velocity at $t=2$?}
  \end{columns}
  
  \vspace{.2in}(At the moment it's not important what \emph{units} we use. May
  be $s$ is in metres and $t$ in seconds, or $s$ in kilometres and $t$ in hours.
  The principle still holds regardless.)
\end{frame}


\begin{frame}
  \frametitle{Instantaneous Velocity}
  \begin{itemize}
  \item We can use $s(t)$ to find the average velocity between $t=2$ and $3$:
    \begin{displaymath}
      v_\mathrm{ave}=\frac{s(3)-s(2)}{3-2}=\frac{27-12}{1}=15
    \end{displaymath}
  \item Or the average velocity between $t=2$ and $t=2.5$
  \item Or the average velocity between $t=2$ and $t=2.1$
  \item But I cannot just plug in $t=2$ into $s(t)$ and expect to get the
    instantaneous velocity, because average velocity needs two specific time
    values.
  \item Perhaps I can find the average velocity between $t=\num{2}$ and\ldots
    \uncover<2>{
      \num{2.000000000001}?
    }
  \end{itemize}
\end{frame}


\begin{frame}
  \frametitle{Solution: Differentiate!}
  \begin{itemize}
  \item The premise of differential calculus (as applied to our example) is
    that if we can find the average velocity between $t=2$ and
    $t=2+h$, where $h$ is a \emph{very} small positive number, we have
    actually found the \emph{instantaneous} velocity at $t=2$

    \begin{align*}
      v&=\frac{s(2+h)-s(2)}{(2+h)-2}=\frac{3(2+h)^2-3(2)^2}{h}\\
      &=\frac{3(4+4h+h^2)-12}{h}=\frac{12h+3h^2}{h}=12+3h\\
    \end{align*}
  \item Since we know that $h$ is a very very small number, we have $v=12$!
  \end{itemize}
\end{frame}


\begin{frame}
  \frametitle{Instantaneous Velocity}
  \framesubtitle{Time Derivative of Displacement}
  \begin{itemize}
  \item In fact, this is the very \emph{definition} of a derivative.
  \item For any arbitrary function $f(x)$, the derivative with respect to $x$
    is defined by:

    \vspace{-.2in}{\LARGE
      \begin{displaymath}
        \boxed{f'(x)=\lim_{h\rightarrow 0}\frac{f(x+h)-f(x)}{h}}
      \end{displaymath}
    }
  \item The ``limit as $h$ approaches $0$'' is the mathematical way of making
    $h$ a very small number
  \end{itemize}
\end{frame}

\begin{frame}
  \frametitle{Instantaneous Velocity}
  \framesubtitle{Time Derivative of Displacement}
  \begin{itemize}
  \item So what we have now is that the instantaneous velocity of an object is
    the time derivative of its position:
    
    \vspace{-0.2in}{\LARGE
      \begin{displaymath}
        \boxed{v(t)=s'(t)=\frac{ds}{dt}}
      \end{displaymath}
    }
  
  \item In physics, we \emph{usually} use the prime notation (e.g.\ $v'$) to
    indicate the derivative is the rate of change with respect to time, and use
    the $d/dx$ notation to indicate rate of change with respect to spatial
    coordinates ($x$, $y$ or $z$). But it's not always the case.
  \end{itemize}
\end{frame}

\begin{frame}
  \frametitle{You Don't Have to Apply The Definition All The Time}
  \framesubtitle{Ways To Save Time}
  It's tedious to use the definition of the derivative every time I want
  to compute the velocity of an object. Thankfully mathematicians have
  recognized some patterns:
  \begin{itemize}
  \item The derivative of a constant (``$C$'') is zero:

    \vspace{-0.1in}{\large
      \begin{displaymath}
        \frac{d}{dt}C=0
      \end{displaymath}
    }
    
    \vspace{-0.1in}This shouldn't be surprising, as the slope of the
    function $f(x)=C$ is always zero %. (We have seen some of these functions in motion
    %graphics in kinematics.)

  \item A constant multiple of any function can be factored outside the
    derivative:

    \vspace{-0.1in}{\large
      \begin{displaymath}
        \frac{d}{dt}\left[af(x)\right]=a\frac{d}{dx}f(x)
      \end{displaymath}
      }
  \end{itemize}
\end{frame}

\begin{frame}
  \frametitle{Time-Saving Rules for Differentiation}
  \begin{itemize}
  \item The derivative of a sum is the sum of a derivative:
    {\large
      \begin{displaymath}
        \frac{d}{dt}\left(f(t)+g(t)\right) = \frac{df}{dt}+\frac{dg}{dt}
      \end{displaymath}
    }
  \item Power Rule:
    {\large
      \begin{displaymath}
        \frac{d}{dt}\left(t^n\right) = nt^{n-1}\quad
        \textsf{for all}\quad n\neq 0
      \end{displaymath}
    }

    FYI: if $n=0$ we really just have a constant.
  \item Try these examples:
    \begin{displaymath}
      \frac{d}{dt}\left(3t^2\right)=\quad\quad\quad
      \frac{d}{dt}\left(t^3 + t + 4\right)=\quad\quad\quad
      \frac{d}{dt}\left(\frac{1}{t}\right)=
    \end{displaymath}
  \end{itemize}
\end{frame}

\begin{frame}
  \frametitle{Time-Saving Rules for Differentiation}
  \begin{itemize}
  \item Sines and cosines:
    {\large
      \begin{displaymath}
      \frac{d}{dt}\sin t = \cos t\quad\quad\quad
      \frac{d}{dt}\cos t = -\sin t
      \end{displaymath}
    }
  \item For AP-level physics, you will not need to be an expert in all things
    differential, but helps to have a lot of experience before tackling
    difficult problems
  \end{itemize}
\end{frame}

\begin{frame}
  \frametitle{Instantaneous Acceleration}
  In the same way that velocity is the time derivative of displacement,
  \textbf{acceleration is the time derivative of velocity}, i.e.:

  \vspace{-0.2in}{\LARGE
    \begin{displaymath}
      \boxed{a(t)= v'(t)=s''(t)}
    \end{displaymath}
  }
  \begin{itemize}
  \item Acceleration is the second derivative of position, i.e.
    \begin{enumerate}
    \item Take derivative of $s(t)$ to get $v(t)=s'(t)$
    \item Take derivative again of $v(t)$ to get $a(t)=v'(t)$
    \end{enumerate}
  \item<2->\textbf{Example:} If the position of an object is given by
    $\displaystyle s(t)=3t^5$, what is
    \begin{itemize}
    \item the velocity at $t=1$ and
    \item the acceleration at $t=1$?
    \end{itemize}
  \end{itemize}
\end{frame}

\subsection{Dynamics}
\begin{frame}
  \frametitle{Newton's Second Law of Motion}
  \begin{itemize}
  \item You may be familiar with Newton's second law written as:

    \vspace{-.1in}{\large
      \begin{displaymath}
        \mb{F}=m\mb{a}=m\frac{d\mb{v}}{dt}=m\frac{d^2\mb{s}}{dt^2}
      \end{displaymath}
    }
  \item That's a \emph{special case} where mass $m$ remains constant, and
    $\mb{F}$ is only related to acceleration
  \item In Grade 11 and 12 physics (no calculus!), we only deal with
    cases where $\mb{F}$ is a constant (acceleration $\mb{a}$ is constant)
  \item With differential calculus, however, we can relate an acceleration
    that is time depending (i.e.\ changes with time) with a time-depend force
  \end{itemize}
\end{frame}

\begin{frame}
  \frametitle{Newton's Second Law of Motion}
  \framesubtitle{General Form}
  \begin{itemize}
  \item The general form of Newton's second law is actually this:
    {\large 
      \begin{displaymath}
        \mb{F}=m\mb{a}=m\frac{d\mb{v}}{dt}=\frac{d(m\mb{v})}{dt}=
        \frac{d\mb{p}}{dt}
      \end{displaymath}
    }
    the vector quantity $\mb{p}=m\mb{v}$ is an object's momentum
  \item In this case, we do not require either the mass or velocity to be
    constant; both can vary with time.
  \end{itemize}
\end{frame}


\begin{frame}
  \frametitle{Let's Try To Do an Example}
  Suppose there is a small cart moving along an icy road with no friction. The
  cart has mass \SI{5}{kg} and a constant velocity \SI{5}{m/s}.
  Suddenly it begins to rain and rain water is collected in the bed of the
  cart. Now the cart's mass changes as
  \begin{displaymath}
    m(t)=5+0.01t\;\si{\kg}
  \end{displaymath}
  That is, every second the mass of the cart increases by \SI{0.01}{\kg}.
  \textbf{If the cart wants to have the same velocity, what would be the force
    needed?}
\end{frame}

\begin{frame}
  \frametitle{Solving the Example Without Calculus}
  \framesubtitle{Not Recommended}
  \begin{itemize}
  \item Before the rain started, the net force on the cart is zero. Once the
    rain started though,
  \item After \SI{1}{\s}, the cart gains \SI{0.01}{\kg} of water in the bed
  \item That means we need to accelerate this water from rest to
    \SI{5}{m/s} in \SI{1}{\s}, i.e.\ $a=5\si{m/s^2}$
  \item This \SI{1}{\s} worth of water will require a force of
    \begin{displaymath}
      F=ma=0.05\si{N}
    \end{displaymath}
  \end{itemize}
\end{frame}


\begin{frame}
  \frametitle{Try it With Calculus}
  \frametitle{Much Easier}
  \begin{itemize}
  \item Apply Newton's second law of motion:
    \begin{displaymath}
      F=\frac{d(mv)}{dt}=\frac{d}{dt}(5+0.01t)(5)=5\frac{d}{dt}(5+0.01t)=
      0.05\;\si{N}
    \end{displaymath}
  \item We can see that in this case, although $v$ is constant, because mass
    $m$ changes with time, there is a net force applied to the cart. 
  \end{itemize}

\end{frame}

\begin{frame}
  \frametitle{Let's Make This a Challenge}
  \frametitle{A Familiar Problem}
  Suppose there is a small cart moving along an icy road with no friction. The
  cart has mass \SI{5}{kg} and a constant velocity \SI{5}{m/s}. Suddenly it
  begins to snow and the wet snow is collected in the bed of the cart. Now the
  cart's mass changes as
  \begin{displaymath}
    m(t)=5+0.01t\;\si{\kg}
  \end{displaymath}

  If the cart wants to have the velocity of
  \begin{displaymath}
    v(t)=5+0.1t
  \end{displaymath}
  what would be the force needed?
\end{frame}

\begin{frame}
  \frametitle{Newton's Second Law}
  \begin{itemize}
  \item We can apply Newton's second law again:
    \begin{align*}
      F(t)&=\frac{d(mv)}{dt}=\frac{d}{dt}\left[(5+0.01t)(5+0.1t)\right]\\
      &=\frac{d}{dt}\left(25+0.55t+0.001t^2 \right)=0.55+0.002t
    \end{align*}
  \item Since both mass and velocity are changing with time, force is not a
    constant, but it's also a function of time
  \end{itemize}
\end{frame}

\begin{frame}
  \frametitle{Another Way of Looking At the Problem}
  \begin{itemize}
  \item Think of this as a two part problem:
    \begin{itemize}
    \item Force $F_1$ is used to provide the acceleration for the existing mass
      $(5+0.01t)\times 0.1$
    \item Force $F_2$ is used to accelerate new water into the speed
      $0.01\times (5+0.1t)$
    \item In all, $F=F_1+F_2$
    \end{itemize}
  \item In fact, this is actually the ``product rule'' in calculus:
    {\large
      \begin{displaymath}
        \frac{d}{dx}\left(f(x)g(x)\right)=
        f'(x)g(x)+f(x)g'(x)
      \end{displaymath}
    }
  \item Applying this to our example, we really have
    \begin{align*}
      F(t)&=\frac{d(mv)}{dt}=m(t)v'(t)+m'(t)v(t)\\
      &=(5+0.01t)\frac{d}{dt}(5+0.1t)+(5+0.1t)\frac{d}{dt}(5+0.01t)\\
      &=(5+0.01t)(0.1)+ (5+0.1t)(0.01)=0.55+0.002t
    \end{align*}
  \end{itemize}
\end{frame}


\section{Integration}

\begin{frame}
  \frametitle{}
  \begin{center}
    {\LARGE\textbf{NOW ON TO INTEGRATION}}
  \end{center}
\end{frame}

\begin{frame}
  \frametitle{Integration: Area Under the Curve}

  \begin{itemize}
  \item Let's to an example: A car is moving with speed $v(t)=5t$. What is its
    displacement at $t=5$?
  \item We know that if on a $v$-$t$ graph, and the area under that curve is
    the displacement. So how do we find the area?
  \item If we divide $5$ into many small time intervals:
    \begin{displaymath}
      \Delta t_1,\;\Delta t_2,\;\Delta t_3,\;\Delta t_4,\ldots,\;\Delta t_n
    \end{displaymath}
    We can find the displacement in teach of these $\Delta t_i$, and
  \item In this example, the total displacement would be  
    \begin{displaymath}
      d(5)=\lim_{n\rightarrow\infty}\sum_{i=1}^{n}v(t_i)\Delta t_i=\int_{t_1}^{t_2}v(t)dt=
      \int_{t=0}^{5}5t\;dt=\frac{5}{2}t^2\Big|^5_0=\frac{125}{2}
    \end{displaymath}
  \end{itemize}
\end{frame}

\begin{frame}
  \frametitle{Integration: Differentiation in Reverse}
  \begin{displaymath}
    \frac{d}{dt}\left(t^2\right)=\frac{1}{2}t
    \quad\quad\longrightarrow\quad\quad
    \int\frac{1}{2}tdt=t^2
  \end{displaymath}
\end{frame}

\begin{frame}
  \frametitle{Commonly Used Integrals in Physics}
  Calculating an integral can be a very daunting task. But these few rules
  should help:

  \begin{align*}
    \int x^ndx&=\frac{1}{n+1}x^{n+1}+C\\
    \int \frac{1}{x}&=\ln x+C\\
    \int\cos xdx&=\sin x+C\\
    \int\sin xdx&=-\cos x+C
  \end{align*}
\end{frame}

\begin{frame}
  \frametitle{Area Under A Curve}
  What is the area under the curve
  \begin{displaymath}
    f(x)=2x^2+3x+1\quad\textsf{between}\quad x=1\;\textsf{and}\;x=5
  \end{displaymath}
  Our integration works like this:
  \begin{align*}
    A&=\int_1^5\left(2x^2+3x+1\right)dt\\
    &=\left(\frac{2}{3}x^3+\frac{3}{2}x^2+x\right)|^5_3\\
    &=24+\frac{196}{3}
  \end{align*}
\end{frame}

\begin{frame}
  \frametitle{Kinematic Equations}
  \begin{itemize}
  \item Remember this equation:

    \vspace{-.2in}{\large
      \begin{displaymath}
        s(t)=s_0+v_0t+\frac{1}{2}at^2
      \end{displaymath}
    }
    
    \vspace{-0.1in}(the notation that you used may be a little bit different,
    but it's the same equation)
  \item We actually obtained this by integrating a constant acceleration
  \end{itemize}
\end{frame}

\begin{frame}
  \frametitle{Integration to Find Volume}
  \begin{itemize}
  \item Interested in finding the volume when we rotate \emph{any} function
    about the $x$ axis
  \item Many applications in physics, e.g.\ finding the centre of
    mass or centroid of shapes
  \end{itemize}
  \begin{columns}
    \column{.33\textwidth}
    \pic{1}{cone.png}
    \column{.64\textwidth}
    \begin{itemize}
    \item Each circular disk the yellow has a volume of $\pi r^2dx$,
      where $r=f(x)$, so the volume of each disk is in fact:
      
      \vspace{-0.3in}{\Large
        \begin{displaymath}
          dV=\pi f(x)^{2} dx
        \end{displaymath}
      }    
    \item ``summing'' them together gives us the integral:
      
      \vspace{-0.2in}{\Large
        \begin{displaymath}
          \boxed{V=\int_{x_1}^{x_2} dV=\int_{x_1}^{x_2} \pi f(x)^{2} dx}
        \end{displaymath}
      }
    \end{itemize}
  \end{columns}
\end{frame}

\begin{frame}
  \frametitle{Integration to Find Volume}
  \textbf{Example:} Find the volume of the following shape:
  \begin{itemize}
  \item In this question, $f(x)=3x$, and we are integrating from $x_1=0$ to
    $x_2=1$
  \end{itemize}
  \vspace{.1in}
  \begin{columns}
    \column{.37\textwidth}
    \pic{1}{cone.png}
    \column{.6\textwidth}
    We use the formula from before:
    \begin{align*}
      V&=\int_{x_1}^{x_2} \pi f(x)^{2} dx\\
      &=\int_{0}^{1} \pi 9x^2dx\\
      &=9\pi\int_{0}^{1} x^2dx\\
      &=3\pi x^3\Big|^1_0\\
      &=3\pi
    \end{align*}
  \end{columns}
\end{frame}

\begin{frame}
  \frametitle{One Last Example}

  \textbf{Using Integration to calculate work done by non-constant force}

  A force of $F(t)=5t\si{\N}$ is applied on an object $m=\SI{1}{\kg}$ at
  rest, there is no friction force. What would be the displacement and work
  done on this object at $t=\SI{3}{\s}$?

  \begin{enumerate}
  \item<2-> Apply Newton's second law to find acceleration:
    $\displaystyle a(t)=\frac{F}{m}=5t$
  \item<3-> Then we integrate to get velocity:
    $\displaystyle v(t)=\int a(t)=\frac{5}{2}t^2$
  \item<4-> And finally, displacement:
    $\displaystyle s(t)=\int v(t)=\frac{5}{6}t^3\quad\longrightarrow\quad
    \textsf{at}\;t=3, \boxed{d=\frac{45}{2}\si{m}}$
  \item<5-> Integrate force with velocity to find work done:
    $\displaystyle W=\int F(t)v(t)dt =\int\frac{25}{2}t^3dt=\frac{25}{8}t^4
    \quad\longrightarrow\quad
    \textsf{at}\;t=3,\;\boxed{W=\frac{2025}{8}\si{J}}$
  \end{enumerate}
\end{frame}

\end{document}
