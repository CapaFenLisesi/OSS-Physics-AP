\documentclass[12pt,aspectratio=169]{beamer}

\usetheme{metropolis}
\setbeamersize{text margin left=.6cm,text margin right=.6cm}
%  \setbeamertemplate{navigation symbols}{} % suppress nav bar
%  \setbeamercovered{transparent}

\usefonttheme{professionalfonts}
\usepackage{graphicx}
\usepackage{tikz}
\usepackage{amsmath}
\usepackage{mathpazo}
\usepackage{xcolor,colortbl}
\usepackage{siunitx}
\usepackage{hyperref}
%\usepackage[siunitx]{circuitikz} % to draw circuits!

\setmonofont{Ubuntu Mono}
\setlength{\parskip}{0pt}
\renewcommand{\baselinestretch}{1}

\sisetup{
  number-math-rm=\mathnormal,
  per-mode=symbol
}

\title{Topic 14: Maxwell's Equations}
\subtitle{Advanced Placement Physics}
\author{Dr.\ Timothy Leung}
\institute{Olympiads School, Toronto, ON, Canada}
\date{March 2020}

\newcommand{\pic}[2]{\includegraphics[width=#1\textwidth]{#2}}\newcommand{\mb}[1]{\mathbf{#1}}
\newcommand{\eq}[2]{\vspace{#1}{\Large\begin{displaymath}#2\end{displaymath}}}
%\newcommand{\protip}[1]{
%  \begin{center}
%    \fbox{
%      \begin{minipage}{.95\textwidth}
%        {\footnotesize
%          \textbf{Protip: }#1
%        }
%      \end{minipage}
%    }
%  \end{center}
%}

\begin{document}

\begin{frame}
  \maketitle
\end{frame}


\section[Intro]{Introduction}

\begin{frame}{Files for You to Download}
  Download from the school website:
  \begin{enumerate}
  \item\texttt{16-maxwellsEquations.pdf}---This
    presentation. If you want to print the slides on paper, I recommend
    printing 4 slides per page.
%  \item\texttt{14-Homework.pdf}---Homework assignment for Class 14.
%    Please note the new formatting style
  \end{enumerate}

  \vspace{.2in}Please download/print the PDF file before each class. When you
  are taking notes, pay particular attention to things I say that aren't
  necessarily on the slides.
\end{frame}



\begin{frame}{Making Amp\`{e}re's Law Better}
  Amp\`{e}re's law, as we know it, only applied to \emph{steady} currents:

  \eq{-.1in}{
    \oint_C \mb{B}\cdot d\boldsymbol{\ell}=\mu_0 I_c
  }
  However,
  \begin{itemize}
  \item Current are usually not steady in $RC$ circuits
  \item Applying Amp\`{e}re's law at a charging/discharging capacitor gives an
    ambiguous answer
  \end{itemize}
\end{frame}



\begin{frame}{Modifying Amp\`{e}re's Law for Unsteady Current}
  \begin{center}
    \pic{.3}{mag_displacement_fig3.png}
  \end{center}
  Four surfaces bounded by the same circular Amperian loop
  (think blowing a soap bubble). Surfaces \numlist{1;2;4} have currents
  penetrating through them, but surface \num{3} does not.
\end{frame}



\begin{frame}{Modifying Amp\`{e}re's Law}
  This might give a better view of what the ``soap bubble'' looks like
  \begin{center}
    \pic{.5}{bubble.png}
  \end{center}
  There is no current through the surface $A_2$ (same as surface \num{3} in the
  last slide), but there is definitely a changing \emph{electric flux}
\end{frame}



\begin{frame}{Maxwell's Modification to Amp\`{e}re's Law}
  James Clerk Maxwell, in 1860, proposed a modification to Amp\`{e}re's Law
  to make it work with unsteady current as well

  \eq{-.1in}{
    \boxed{
      \oint\mb{B}\cdot d\boldsymbol{\ell}=\mu_0 I +
      \mu_0\varepsilon_0 \frac{d\Phi_E}{dt}
    }
  }

  Maxwell called the correction term
  $\displaystyle \varepsilon_0\frac{d\Phi_E}{dt}$
  \textbf{displacement current}.
  \begin{itemize}
  \item The word ``displacement'' has historical roots, but no physical meaning
  \item However, ``current'' means that the effect of changing the electric
    flux is indistinguishable from real currents in producing magnetic field
  \end{itemize}
\end{frame}



\begin{frame}{Maxwell's Equations}
  \begin{itemize}
  \item Maxwell recognized the relationship between electricity and
    magnetism in \textbf{Gauss's law}, \textbf{Faraday's law} and
    \textbf{Amp\`{e}re's law}
  \item Combined them into a unified set of equations, now known as
    \textbf{Maxwell's equations} for electrodynamics
  \end{itemize}
\end{frame}


\begin{frame}{Maxwell's Equations in Integral Form}
  Maxwell's equations can be expressed in its integral form, which is how we
  have studied the equations in the first place:

  \vspace{-.2in}{\Large
    \begin{align*}
      \oint\mb{E}\cdot d\mb{A} &=\frac{q}{\varepsilon_0} &
      \text{\normalsize (Gauss, for $\mb{E}$)}\\
      \oint\mb{B}\cdot d\mb{A} &= 0 &
      \text{\normalsize (Gauss, for $\mb{B}$)}\\
      \oint\mb{E}\cdot d\boldsymbol{\ell} &=\frac{d\Phi_B}{dt} &
      \text{\normalsize (Faraday)}\\
      \oint\mb{B}\cdot d\boldsymbol{\ell} &
      =\mu_0I+\mu_0\varepsilon_0\frac{d\Phi_E}{dt} &
      \text{\normalsize (Amp\`{e}re, with Maxwell's mod)}
    \end{align*}
  }
\end{frame}



\begin{frame}{Maxwell's Equations in Vacuum}
  \begin{columns}
    \column{.35\textwidth}
    
    {\Large
      \begin{align*}
        \oint\mb{E}\cdot d\mb{A} &= 0 \\
        \oint\mb{B}\cdot d\mb{A} &= 0 \\
        \oint\mb{E}\cdot d\boldsymbol{\ell}&=\frac{d\Phi_B}{dt} \\
        \oint\mb{B}\cdot d\boldsymbol{\ell}&=\mu_0\epsilon_0\frac{d\Phi_E}{dt}\\
      \end{align*}
    }

    \column{.65\textwidth}
    In vacuum, we can remove all references to matter in the equation, and
    Maxwell's equations simplifies.
    \begin{itemize}
    \item The equations show ``symmetry''
    \item Magnetic and electric fields are on equal footing
    \item In a vacuum where charges are currents are absent, the only source of
      either field is a change in the other field
    \end{itemize}
  \end{columns}
\end{frame}



\begin{frame}{Maxwell's Equations in Differential Form}
  \framesubtitle{For Simplicity, in a Vacuum}
  \begin{columns}

    \column{.25\textwidth}
    {\Large
      \begin{align*}
        \nabla\cdot\mb{E} &= 0\\
        \nabla\cdot\mb{B} &= 0\\
        \nabla\times\mb{E} &=-\frac{\partial\mb{B}}{\partial t}\\
        \nabla\times\mb{B} &=\mu_o\varepsilon_o\frac{\partial\mb{E}}{\partial t}
      \end{align*}
    }

    \column{.75\textwidth}
    \begin{itemize}
    \item Maxwell's equations are usually expressed in \emph{differential} form,
      which is obtained using vector calculus. Follow
      [\underline{\href{https://www.wikihow.com/Convert-Maxwell\%27s-Equations-into-Differential-Form}{this link}}]
      to see how it's done.
    \item The differential form shows how the \emph{time derivatives} of
      $\mb{E}$ and $\mb{B}$ are related to the \emph{spatial derivatives}
      of the other field
    \item The last two equations (Faraday's and Amp\`{e}re's laws) together
      represent two set of second order partial differential equations (one for
      each field), the solution of which represents a traveling wave
    \end{itemize}
  \end{columns}
\end{frame}



\begin{frame}{Electromagnetic (EM) Wave}
  Maxwell's equations show that an ``electromagnetic wave'' must exist. In a
  simple case where electric and magnetic fields only vary in
  $x$ and time $t$ only, i.e.\ $\mb{E}=\mb{E}(x,t)$ and $\mb{B}=\mb{B}(x,t)$,
  Faraday's and Amp\`{e}re's laws reduce to:

  \eq{-.1in}{
    \frac{\partial E}{\partial x}=-\frac{\partial B}{\partial t}
    \quad\quad
    \frac{\partial B}{\partial x}=
    -\mu_0\varepsilon_0\frac{\partial E}{\partial t}
  }

  Taking the spatial derivative of $E$ with respect to $x$ on both
  side of Faraday's law, and switch the order of differentiation, we get:

  \eq{-.2in}{
      \frac{\partial}{\partial x}
      \left(\frac{\partial E}{\partial x}\right)
      =-\frac{\partial}{\partial x}\left(\frac{\partial B}{\partial t}\right)
      \quad\rightarrow\quad
      \frac{\partial^2E}{\partial x^2}=
      -\frac{\partial}{\partial t}\left(\frac{\partial B}{\partial x}\right)
  }
\end{frame}



\begin{frame}{Electromagnetic (EM) Wave}
  But we already have an expression for $\partial B/\partial x$ from
  Amp\`{e}re's law:

  \eq{-.2in}{
    \frac{\partial^2E}{\partial x^2}=
    -\frac{\partial}{\partial t}\left(\frac{\partial B}{\partial x}\right)
    =-\frac{\partial}{\partial t}\left(
    -\mu_0\varepsilon_0\frac{\partial E}{\partial t}
    \right)
  }

  Rearranging the terms on the right hand side, we get

  \eq{-.2in}{
    \frac{\partial^2E}{\partial x^2}=
    \mu_0\varepsilon_0\frac{\partial^2 E}{\partial t^2}
  }
  
  This is the standard form of the 1D wave equation (a 2nd-order partial
  differential equation):

  \eq{-.2in}{
    \frac{\partial^2\Psi}{\partial x^2}=
    \frac{1}{v^2}\frac{\partial^2\Psi}{\partial t^2}
  }
\end{frame}



\begin{frame}{Electromagnetic (EM) Wave}
  \begin{itemize}
  \item ``Second-order'' means that the equation deals with second derivatives,
    in this case, in $x$ and in $t$.
  \item ``Partial'' means the equation involves partial derivatives (i.e.\
    when a function has more than one variables, and you only differentiate
    against one variable)
  \item We can also repeat the exercise by first differentiating Amp\`{e}re's
    law to get

    \eq{-.2in}{
      \frac{\partial^2 B}{\partial x^2}=
      \mu_0\varepsilon_0\frac{\partial^2 B}{\partial t^2}
    }
  \end{itemize}
\end{frame}



\begin{frame}{Electromagnetic (EM) Wave}
  \begin{itemize}
  \item The wave equation shows that disturbances in electric and magnetic
    fields propagate as an electromagnetic wave with a universal speed

    \eq{-.2in}{
      v=c_0=\frac{1}{\sqrt{\mu_0\varepsilon_0}}=\SI{299792458}{m/s}
    }

    generally referred to as the speed of light.

  \item Our simple exercise can't show (because we have effectively ignored
    the cross-product) that $\mb{E}$ and $\mb{B}$ are actually perpendicular
    to each other
  \end{itemize}
\end{frame}



\begin{frame}{Electromagnetic (EM) Wave}
  A EM wave is considered to be \textbf{polarized} if both $\mb{E}$
  (and therefore) $\mb{B}$ of the wave are confined to a single plane. The
  direction of the polarization is the direction of $\mb{E}$.
  \begin{center}
    \pic{.4}{em-20field.png}
  \end{center}
\end{frame}



\begin{frame}{``Failure'' of Maxwell's Equation}
  A peculiar feature of Maxwell's equation:
  \begin{itemize}
  \item When applying \emph{Galilean transformation} (our classical equation for
    \emph{relative motion}) to Maxwell's equations, they seem to ``fail''
  \item Gauss's law for magnetism break down: magnetic field lines appear to
    have beginnings/ends
  \item So does that mean that in \emph{some} inertial frames of reference,
    Maxwell's equations are valid, but in others, they are not?
  \item Physicists theorized that, perhaps, there is/are actually
    \emph{preferred} inertial frame(s) of references
  \item This violate the long-standing
    \emph{principle of relativity}, which says that
    \emph{the laws of physics are equal in all inertial frames of reference}
  \end{itemize}
\end{frame}



\begin{frame}{Making The Equations Work Again}
  Maxwell's equations didn't ``fail''; it was our understanding of space and
  time that needed to change
  \begin{itemize}
  \item Albert Einstein believed in the principle of relativity, and rejected
    the concept of a preferred frame of reference
  \item In Maxwell's equations, the speed of an electromagnetic wave (speed of
    light) is independent of the frame of reference
  \item In order to make the equations to work again, Einstein revisited the
    most basic concepts involved in our understanding of physics: space and
    time
  \end{itemize}
\end{frame}



\begin{frame}{Einstein and the Principle of Relativity}
  Einstein's Postulates of Special Relativity:
  \begin{enumerate}
  \item All laws of physics must apply equally in all inertial frames of
    reference.
  \item As measured in any inertial frame of reference, light always propagated
    empty space with a definite velocity $c$ that is independent of the state
    of motion of the emitting body.
  \end{enumerate}
  Published in 1905 in the article
  \emph{On the Electrodynamics of Moving Bodies} when Einstein was 26 years old
  working as a patent clerk in Switzerland
\end{frame}
\end{document}
