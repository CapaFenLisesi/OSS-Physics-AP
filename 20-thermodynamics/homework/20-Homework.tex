\documentclass{../../oss-apphys}

\begin{document}
\genheader

\gentitle{2}{FLUID MECHANICS \& THERMODYNAMICS}{19 \& 20}

\genmultidirections

\gengravity

\raggedcolumns
\begin{multicols}{2}

  \begin{enumerate}[leftmargin=18pt]

  \item Two blocks of different sizes and masses float in a tray of water. Each
    block is half submerged, as shown in the figure. Water has a density of
    \SI{1000}{kg/m^3}. What can be concluded about the densities of the two
    blocks?
    \begin{enumerate}[noitemsep,topsep=0pt,leftmargin=18pt,label=(\Alph*)]
    \item The two blocks have different densities, both of which are less than
      \SI{1000}{kg/m^3}.
    \item The two blocks have the same density of \SI{500}{kg/m^3}.
    \item The two blocks have the same density, but the density cannot be
      determined with the information given.
    \item The larger block has a greater density than the smaller block, but
      the densities of the blocks cannot be determined with the information
      given.
    \end{enumerate}

  \item The figure shows four cylinders of various diameters filled to different
    heights with water. A hole in the side of each cylinder is plugged by a
    cork. All cylinders are open at the top. The corks are removed. Which
    of the following is the correct ranking of the velocity of the water ($v$)
    as it exits each cylinder?
    \begin{enumerate}[noitemsep,topsep=0pt,leftmargin=18pt,label=(\Alph*)]
    \item $v_A > v_D > v_C > v_B$
    \item $v_A = v_D > v_C > v_B$
    \item $v_B > v_C > v_A = v_D$
    \item $v_C > v_A = v_B = v_D$
    \end{enumerate}
  \end{enumerate}
  
  \textbf{Questions 3 and 4}

  Four differently shaped sealed containers are completely filled with alcohol,
  as shown in the figure. Containers A and B are cylindrical. Containers C and
  D are truncated conical shapes. The top and bottom diameters of the
  containers are shown.
  
  \begin{enumerate}[leftmargin=18pt,start=3]
    
  \item Which of the following is the correct ranking of the pressure ($P$) at
    the bottom of the containers?
    \begin{enumerate}[noitemsep,topsep=0pt,leftmargin=18pt,label=(\Alph*)]
    \item $P_A = P_B = P_C = P_D$
    \item $P_A = P_D > P_C = P_B$
    \item $P_A > P_D > P_C > P_B$
    \item $P_D > P_A > P_C > P_B$
    \end{enumerate}

  \item The force on the bottom of container A due to the fluid inside the
    container is F. What is the force on the bottom of container B due to
    the fluid inside?
    \begin{enumerate}[noitemsep,topsep=0pt,leftmargin=18pt,label=(\Alph*)]
    \item $F$
    \item $F/4$
    \item $F/8$
    \item $F/16$
    \end{enumerate}
    
  \item Two cylinders filled with a fluid are connected by a pipe so that fluid
    can pass between the cylinders, as shown in the figure. The cylinder
    on the right has 4 times the diameter of the cylinder on the left. Both
    cylinders are fitted with a movable piston and a platform on top. A
    person stands on the left platform. Which of the following lists the
    correct number of people that need to stand on the right platform so
    neither platform moves. Assume that the platform and piston have
    negligible mass and that all the people have the same mass.
    \begin{enumerate}[noitemsep,topsep=0pt,leftmargin=18pt,label=(\Alph*)]
    \item \num{16} people
    \item \num{4} people
    \item \num{1} person
    \item It is impossible to balance the system because you need 1/16 of a
      person on the right side.
    \end{enumerate}

  \item A mass ($m$) is suspended in a fluid of density ($\rho$) by a thin
    string, as shown in the figure. The tension in the string is $T$. Which of
    the following is an appropriate equation for the buoyancy force? Select
    two answers.
    \begin{enumerate}[noitemsep,topsep=0pt,leftmargin=18pt,label=(\Alph*)]
    \item $F_b=mg$
    \item $F_b=mg-T$
    \item $F_c=a_2 \rho gh_1$
    \item $F_d=a\rho g(h-h_2)$
    \end{enumerate}
    
  \item Three wooden blocks of different masses and sizes float in a container
    of water, as shown in the figure. Each of the masses has a weight on top.
    Which of the following correctly ranks the buoyancy force on the wooden
    blocks?
    \begin{enumerate}[noitemsep,topsep=0pt,leftmargin=18pt,label=(\Alph*)]
    \item $A > B = C$
    \item $A = B > C$
    \item $B > A = C$
    \item $B > A > C$
    \end{enumerate}
    
  \item Two blocks of the same dimensions are floating in a container of water,
    as shown in the figure. Which of the following is a correct statement about
    the two blocks?
    \begin{enumerate}[noitemsep,topsep=0pt,leftmargin=18pt,label=(\Alph*)]
    \item The net force on both blocks is the same.
    \item The buoyancy force exerted on both blocks is the same.
    \item The density of both blocks is the same.
    \item The pressure exerted on the bottom of each block is the same.
    \end{enumerate}
    
  \item The figure shows four cubes of the same volume at rest in a container
    of water. Cube C is partially submerged. Cubes A, B, and D are fully
    submerged, with B resting on the bottom of the container. Which of the
    following correctly ranks the densities ($\rho$) of the cubes? Assume the
    water to be incompressible.
    \begin{enumerate}[noitemsep,topsep=0pt,leftmargin=18pt,label=(\Alph*)]
    \item $\rho_C >\rho_D >\rho_A >\rho_B$
    \item $\rho_B >\rho_A >\rho_D >\rho_C$
    \item $\rho_B >\rho_A =\rho_D >\rho_C$
    \item $\rho_B >\rho_A =\rho_D =\rho_C$
    \end{enumerate}

  \item A beaker of water sits on a balance. A metal block with a mass of
    \SI{70}{\gram} is held suspended in the water by a spring scale in position
    1, as shown in the figure. In this position, the reading on the balance is
    \SI{1260}{\gram}, and the spring scale reads \SI{120}{g}. When the block is
    lifted from the water to position 2, what are the readings on the balance
    and spring scale?

  \item Blood cells pass through an artery that has a buildup of plaque along
    both walls, as shown in the figure. Which of the following correctly
    describes the behavior of the blood cells as they move from the right
    side of the figure through the area of plaque? Assume the blood cells
    can change volume.
    \begin{enumerate}[noitemsep,topsep=0pt,leftmargin=18pt,label=(\Alph*)]
    \item The blood cells increase in speed and expand in volume.
    \item The blood cells increase in speed and decrease in volume.
    \item The blood cells decrease in speed and expand in volume.
    \item The blood cells decrease in speed and decrease in volume.
    \end{enumerate}
    
  \item Firefighters use a hose with a \SI{2}{cm} exit nozzle connected to a
    hydrant with an \SI{8}{cm} diameter opening to attack a fire on the second
    floor of a building 6 m above the hydrant, as shown in the figure. What
    pressure must be supplied at the hydrant to produce an exit velocity of
    \SI{15}{m/s}? (Assume the density of water is \SI{1000}{kg/m^3}, and the
    exit pressure is \SI{1e5}{\pascal}.)
    \begin{enumerate}[noitemsep,topsep=0pt,leftmargin=18pt,label=(\Alph*)]
    \item\SI{1.7e5}{\pascal}
    \item\SI{2.0e5}{\pascal}
    \item\SI{2.6e5}{\pascal}
    \item\SI{3.2e5}{\pascal}
    \end{enumerate}
    
  \item A \SI{1}{cm} diameter pipe leads to a showerhead with twenty \SI{1}{mm}
    diameter exit holes. The velocity of the water in the pipe is $v$. What is
    the velocity of the water exiting the holes?
    \begin{enumerate}[noitemsep,topsep=0pt,leftmargin=18pt,label=(\Alph*)]
    \item $0.05v$
    \item $0.5v$
    \item $5v$
    \item $100v$
    \end{enumerate}

  \item Air is made up primarily of nitrogen and oxygen. In an enclosed room
    with a constant temperature, which of the following statements is
    correct concerning the nitrogen and oxygen gases?
    \begin{enumerate}[noitemsep,topsep=0pt,leftmargin=18pt,label=(\Alph*)]
    \item The nitrogen gas molecules have a higher average kinetic energy than
      the oxygen gas molecules.
    \item The nitrogen gas molecules have the same average kinetic energy as
      the oxygen gas molecules.
    \item The nitrogen gas molecules have a lower average kinetic energy than
      the oxygen gas molecules.
    \item More information is necessary to compare the average kinetic energies
      of the two gases.
    \end{enumerate}
    
  \item Air is made up primarily of nitrogen and oxygen. In an enclosed room
    with a constant temperature, which of the following statements is correct
    concerning the nitrogen and oxygen gases?
    \begin{enumerate}[noitemsep,topsep=0pt,leftmargin=18pt,label=(\Alph*)]
    \item The nitrogen gas molecules have a higher velocity than the oxygen gas
      molecules.
    \item The nitrogen gas molecules have the same velocity as the oxygen gas
      molecules.
    \item The nitrogen gas molecules have a lower velocity than the oxygen gas
      molecules.
    \item It is impossible to compare the velocity of the two gases without
      knowing the temperature of the air and the percentage of nitrogen and
      oxygen in the room.
    \end{enumerate}
  
  \item In an experiment, a gas is confined in a cylinder with a movable piston.
    Force is applied to the piston to increase the pressure and change the
    volume of the gas. Each time the gas is compressed, it is allowed to
    return to a room temperature of \SI{20}{\celsius}. The data gathered from
    the experiment is shown in the table. What should be plotted on the
    vertical and horizontal axes so the slope of the graph can be used to
    determine the number of moles of gas in the cylinder?
    \begin{enumerate}[noitemsep,topsep=0pt,leftmargin=18pt,label=(\Alph*)]
    \item $P$ and $V_2$
    \item $P$ and $V$
    \item $P$ and $(V) 1⁄2$
    \item $P$ and $1/V$
    \end{enumerate}

  \item In an experiment, a sealed container with a volume of
    \SI{100}{\milli\litre} is filled with hydrogen gas. The container is heated
    to a variety of temperatures, and the pressure is measured. The data from
    the experiment is plotted in the figure. Which of the following methods
    can be used to determine additional information regarding the gas?
    Select two answers.
    \begin{enumerate}[noitemsep,topsep=0pt,leftmargin=18pt,label=(\Alph*)]
    \item The slope can be used to calculate the number of atoms in the gas.
    \item The area under the graph can be used to calculate the work done by
      the gas.
    \item The vertical axis can be used to calculate the force the gas exerts
      on the container.
    \item The x-intercept can be used to estimate the value of absolute zero.
    \end{enumerate}
  
  \item Two identical rooms are connected by an open door. The temperature in
    one room is greater than the temperature in the other. Which room contains
    the most gas molecules?
    \begin{enumerate}[noitemsep,topsep=0pt,leftmargin=18pt,label=(\Alph*)]
    \item The warmer room.
    \item The colder room.
    \item The number of gas molecules will be the same in both rooms.
    \item It is impossible to determine without more information.
    \end{enumerate}
    
  \item On a hiking trip in the mountains, where the air temperature is cool and
    has a lower concentration of oxygen, you seal an empty water bottle.
    You return to your home near sea level where the air temperature is
    warm and has a higher concentration of oxygen. You notice that the
    sealed bottle appears partially crushed. Which of the following would
    contribute to the decrease in volume of the bottle?
    \begin{enumerate}[noitemsep,topsep=0pt,leftmargin=18pt,label=(\Alph*)]
    \item The change in temperature
    \item The change in atmospheric pressure
    \item The change in oxygen concentration
    \item The change in temperature, pressure, and oxygen concentration
    \end{enumerate}

  \item The figure shows the pressure and volume of a gas at four different
    states. Which of the following correctly ranks the temperature of the gas
    at the different states?
    \begin{enumerate}[noitemsep,topsep=0pt,leftmargin=18pt,label=(\Alph*)]
    \item $T_A>T_B>T_C>T_D$
    \item $T_B=T_C>T_A=T_D$
    \item $T_C>T_B=T_D>T_A$
    \item $T_D>T_C>T_B>T_A$
    \end{enumerate}
  
  \item Which of the following is correct concerning the two processes shown
    in the figure?
    \begin{enumerate}[noitemsep,topsep=0pt,leftmargin=18pt,label=(\Alph*)]
    \item $\Delta U_1 = \Delta U_2$ and $W_1= W_2$
    \item $\Delta U_1 = \Delta U_2$ and $W_1>W_2$
    \item $\Delta U_1 > \Delta U_2$ and $W_1=W_2$
    \item $\Delta U_1 > \Delta U_2$ and $W_1\geq W_2$
    \end{enumerate}
  
  \item The figure shows four samples of gas being taken through four
    different processes. Process 1 is adiabatic. In which process is heatbeing
    transferred to the gas sample from the environment?
    \begin{enumerate}[noitemsep,topsep=0pt,leftmargin=18pt,label=(\Alph*)]
    \item\num{1}
    \item\num{2}
    \item\num{3}
    \item\num{4}
    \end{enumerate}

  \item Two sealed cylinders holding different gases are placed one on top of
    the other so heat can flow between them. Cylinder A is filled with
    hydrogen. Cylinder B is filled with helium moving with an average speed
    that is half that of the hydrogen atoms. Helium atoms have four times the
    mass of hydrogen atoms. Which of the following best describes the transfer
    of heat between the two containers by conduction?
    \begin{enumerate}[noitemsep,topsep=0pt,leftmargin=18pt,label=(\Alph*)]
    \item Net heat flows from cylinder A to cylinder B, because heat flows from
      higher kinetic energy atoms to lower kinetic energy atoms.
    \item Net heat flows from cylinder B to cylinder A, because heat flows from
      higher kinetic energy atoms to lower kinetic energy atoms.
    \item There is no net heat transfer between the two cylinders, because both
      gases have the same average atomic kinetic energy.
    \item There is no net heat transfer between the two cylinders, because heat
      conduction requires the movement of atoms between the cylinder, and the
      cylinders are sealed.
    \end{enumerate}
  \end{enumerate}
  
  \textbf{Questions 24 and 25}

  A gas beginning at point O on the graph can be taken along four paths to
  different ending conditions.

  \begin{enumerate}[leftmargin=18pt,start=24]

  \item Which of the following are the same for processes 2 and 3? Select two
    answers.
    \begin{enumerate}[noitemsep,topsep=0pt,leftmargin=18pt,label=(\Alph*)]
    \item $Q$
    \item $\Delta T$
    \item $\Delta U$
    \item $W$
    \end{enumerate}
    
  \item Along which of the paths is the most thermal energy removed from the
    gas?
    \begin{enumerate}[noitemsep,topsep=0pt,leftmargin=18pt,label=(\Alph*)]
      \item\num{1}
      \item\num{2}
      \item\num{3}
      \item\num{4}
    \end{enumerate}
    
  \item The graph shows the distribution of speeds for one mole of hydrogen at
    temperature $T$, pressure $P$, and volume $V$. How would the graph change
    if the sample was changed from one mole hydrogen to one mole of argon at
    the same temperature, pressure, and volume?
    \begin{enumerate}[noitemsep,topsep=0pt,leftmargin=18pt,label=(\Alph*)]
    \item The peak will shift to the left
    \item The peak will shift upward and to the left
    \item The peak will shift to the right
    \item The peak will shift downward and to the right
    \end{enumerate}
    
  \item The graph shows the pressure and volume of a gas being taken from state
    \#1 to state \#2. Which of the following correctly indicates the sign of
    the work done by the gas, and the change in temperature of the gas?

  \end{enumerate}
  

  \textbf{Questions 115 and 116}
  
  A resistor of resistance ($R$) is sealed in a closed container with $n$ moles
  of gas inside. A battery of emf ($\mathcal{E}$) is connected to the resistor

  \begin{enumerate}
  \item Which of the following graphs shows the correct relationship between
    the gas atoms’ average velocity ($v_{\textrm{avg}}$) and electrical energy
    ($E$) supplied to the resistor?

  \item Which of the following graphs shows the correct relationship between
    he gas atoms’ average velocity ($v_{\textrm{avg}}$) and electrical energy
    ($E$) supplied to the resistor?
  \end{enumerate}
\end{multicols}

\newpage
\genanswersheet{2}{Fluid Mechanics \& Thermodynamics}

\begin{center}
  %begin{minipage}{.2\textwidth}
  \vspace{.2in}
  \bgroup
  \begin{tabular}{>{\centering}m{1.3cm} >{\centering}m{1.7cm}}
    No. & Answer
  \end{tabular}\\
  \def\arraystretch{1.5}
  \begin{tabular}{|>{\centering}m{1.3cm}|>{\centering}m{1.7cm}|}
    \hline
    1 & \\ \hline
    2 & \\ \hline
    3 & \\ \hline
    4 & \\ \hline
    5 & \\ \hline
    6 & \\ \hline
    7 & \\ \hline
    8 & \\ \hline
    9 & \\ \hline
    10 & \\ \hline
    11 & \\ \hline
    12 & \\ \hline
    13 & \\ \hline
    14 & \\ \hline
    15 & \\ \hline
    16 & \\ \hline
    17 & \\ \hline
    18 & \\ \hline
    19 & \\ \hline
    20 & \\ \hline
    21 & \\ \hline
    22 & \\ \hline
    23 & \\ \hline
    24 & \\ \hline
  \end{tabular}
  \egroup
  %end{minipage}
\end{center}
\newpage

\genfreetitle{2}{Fluid Mechanics \& Thermodynamics}{4}

\genfreedirections{10}

%\begin{enumerate}[leftmargin=15pt]
%
%\item A lead box containing radioactive materials that emit both electrons and
%  positrons is placed near an apparatus consisting of an evacuated capacitor
%  that is filled with a magnetic field, as shown in the figure. Electrons that
%  enter along the center line of the capacitor plates travel straight through
%  (undeflected) with a velocity of $v=\SI{1.0e7}{m/s}$ and out
%  the hole in the center of the apparatus on the right. The capacitor plates
%  are separated by a distance of $d=\SI{0.020}{\metre}$; each plate has an area
%  of $A=\SI{1.0e-4}{m^2}$ and a potential difference of $\Delta V$. A uniform
%  magnetic field of $B=\SI{0.030}{\tesla}$ is directed out of the page between
%  the plates, as shown in the figure.
%
%  \vspace{-.2in}
%  \begin{center}
%    \begin{tikzpicture}[scale=1.2]
%      \draw(0,0) rectangle(5,0.2);
%      \draw(0,1.8) rectangle(5,2);
%      \draw(-.75,0.2) rectangle(-2.5,1.8)
%      node[midway]{\footnotesize Radioactive};
%      \foreach \x in {.5,1,...,4.5}{
%        \foreach \y in {.5,1,1.5} \draw(\x,\y) circle(0.075);
%      }
%      \draw[dashed](0,1)--(7,1);
%      \draw[very thick,<->](4.3,.2)--(4.3,1.8) node[pos=.7,left]{$d$};
%      \foreach \xx in {-.5,5.5}{
%        \draw[fill=gray!60](\xx,1) circle (0.1);
%        \draw[very thick,->](\xx+.2,1)--(\xx+1.2,1) node[midway,above]{$v$};
%      }
%      \draw(5.4,1.3) rectangle(5.6,2.5);
%      \draw(5.4,0.7) rectangle(5.6,-.5);
%    \end{tikzpicture}
%  \end{center}
%
%  \begin{enumerate}[noitemsep]
%  \item\vspace{-.2in} Explain why it is acceptable to neglect the effects of
%    gravity on the electrons passing through the apparatus.
%  \item
%    \begin{enumerate}
%    \item Explain why the electrons pass through the capacitor plates
%      undeflected. Support your argument with an algebraic equation
%      and an appropriately drawn force diagram.
%
%    \item Use your equation to calculate the potential difference ($\Delta V$)
%      between the capacitor plates.
%
%    \item Which capacitor plate has the highest potential? Justify your 
%      reasoning making reference to the electric field.
%
%    \item Calculate the magnitude of the energy that is stored in the capacitor.
%    \end{enumerate}
%
%  \item A positron enters the apparatus along the same path as the
%    electrons from part (b).
%    \begin{enumerate}
%    \item Explain why the positron, traveling at the same speed as the
%      electrons, will also travel straight through the device undeflected.
%      Support your argument with an equation.
%    \item A second positron enters the apparatus at a speed of $2v$. Sketch
%      the path of the positron through the capacitor plates on the figure.
%    \end{enumerate}
%
%  \item An electron exits the apparatus at a velocity of $v=\SI{1.0e7}{m/s}$
%    parallel to a long wire of a circuit, as shown in the figure. The
%    distance between the electron and the wire is \SI{1}{mm}.
%    \begin{center}
%      \begin{tikzpicture}[american voltages,scale=1.2]
%        \draw(0,-.5)--(0,1)--(4,1)--(4,-.5) to[battery1=$\mathcal{E}$] (2,-.5)
%        to[R=$R$] (0,-.5);
%        \draw[fill=gray!60](1.5,1.3) circle (0.1);
%        \draw[thick,->](1.7,1.3)--(2.7,1.3) node[midway,above]{$v$};
%      \end{tikzpicture}
%    \end{center}
%    \begin{enumerate}
%    \item Calculate the potential difference-to-resistance ratio of the circuit
%      such that the electron will experience a force $F$ of \SI{1.3e-16}{N}.
%    \item Draw a force vector on the figure to show the direction of the force
%      on the electron.
%    \end{enumerate}
%  \end{enumerate}
%\end{enumerate}
%\newpage
%Perform calculations for Question 1 here.
%\newpage
%
%\begin{center}
%  \begin{tikzpicture}[scale=.8]
%    \foreach \x in {0,...,8} {
%      \foreach \y in {0,...,4} \node at (\x,\y) {\textcolor{blue}{$\times$}};
%    }
%    \draw[fill=brown!60](-.5,3.4) rectangle(7.8,3.6);
%    \draw[fill=brown!60](-.5,0.4) rectangle(7.8,0.6);
%    \draw[fill=brown](3.2,-.2) rectangle(3.8,4.2);
%    \draw[<->](-0.7,0.5)--(-0.7,3.5) node[midway,left]{$\ell$};
%    \node at (3.5,4.5) {$\mb{B}$ into page};
%    \node at (8,0.5) {$b$};
%    \node at (8,3.5) {$a$};
%    \node at (4.1,2.5) {$R$};
%  \end{tikzpicture}
%\end{center}
%
%\begin{enumerate}[leftmargin=18pt]
%  \setcounter{enumi}{1}
%\item In the above figure, a rod has a resistance and the rails have negligible
%  resistance. A battery of emf $\mathcal{E}$ and negligible internal resistance
%  is connected between points $a$ and $b$ such that the current in the rod is
%  downward. The rod is placed at rest aqt $t=0$.
%  \begin{enumerate}[noitemsep]
%  \item Find the force on the rod as a function of speed $v$.
%  \item Show that the rod reaches terminal velocity, and find the expression for
%    it.
%  \item What is the current when the rod reaches its terminal velocity?
%  \end{enumerate}
%  \vspace{2in}
%  
%\item In the above figure, the rod has a resistance of $R$ and the rails have
%  negligible resistance. A capacitor with charge $Q_0$ and capacitance $C$ is
%  connected between points $a$ and $b$ such that the current in the road is
%  downward. The rod is places at rest at $t=0$.
%  \begin{enumerate}[noitemsep]
%  \item Write the equation of motion for the rod on the rails.
%  \item Show that the terminal speed of the rod down the rail is related to the
%    final charge on the capacitor.
%  \end{enumerate}
%\end{enumerate}
%\newpage
%\begin{center}
%  \begin{tikzpicture}[scale=.8]
%    \draw[blue!60!red,ultra thick](0,0) rectangle(4,4);
%    \draw[<->](0,4.3)--(4,4.3) node[midway,above]{\SI{20}{cm}};
%    \foreach \x in {0.5,1.5,...,3.5} {
%      \foreach \y in {0.5,1.5,...,3.5} \fill[blue!70](\x,\y) circle(0.05);
%    }
%    \node at (3,3) {$\mb{B}$};
%    \draw[very thick](-1,1.6) rectangle(0.7,2.7);
%    \draw[<->](-1.2,1.6)--(-1.2,2.7) node[midway,left]{\SI{5}{cm}};
%    \draw[<->](-1,2.9)--(.7,2.9) node[midway,above]{\SI{10}{cm}};
%    \draw[green!80!black,ultra thick,->](.7,2.2)--(1.7,2.2)
%    node[pos=1,right]{$v$};
%  \end{tikzpicture} 
%\end{center}
%
%\begin{enumerate}[leftmargin=18pt]
%
%  \setcounter{enumi}{3}
%\item A \SI{10}{cm} by \SI{5}{cm} rectangular loop with resistance
%  \SI{2.5}{\ohm} is pulled through a region of uniform magnetic field
%  $B=\SI{1.7}{\tesla}$ with constant speed $v=\SI{2.4}{cm/s}$. The front of the
%  loop enters the region of the magnetic field at time $t=0$.
%  \begin{enumerate}[noitemsep,leftmargin=18pt]
%  \item Find and graph the flux through the loop as a function of time.
%  \item Find and graph the included emf and the current in the loop as a
%    function of time. Neglect any self inductance of the loop and extend your
%    graph from $t=0$ to $t=\SI{16}{\second}$.
%  \end{enumerate}
%\end{enumerate}
\end{document}
