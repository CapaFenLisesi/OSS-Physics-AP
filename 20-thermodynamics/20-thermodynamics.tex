\documentclass[12pt,aspectratio=169]{beamer}

\mode<presentation>
{
  \usetheme{Singapore}
 %\setbeamersize{text margin left=.6cm,text margin right=.6cm}
%  \setbeamertemplate{navigation symbols}{} % suppress nav bar
%  \setbeamercovered{transparent}
}
\usefonttheme{professionalfonts}
\usepackage{graphicx}
\usepackage{tikz}
\usepackage{amsmath,bm}
\usepackage{mathpazo}
\usepackage[scaled]{helvet}
\usepackage{xcolor,colortbl}
\usepackage{siunitx}
%\usepackage{hyperref}

\sisetup{detect-all}

\title{Class 20: Thermodynamics}
\subtitle{AP Physics}
\author[TML]{Dr.\ Timothy Leung}
\institute{Olympiads School}
\date{April 2018}

\newcommand{\pic}[2]{\includegraphics[width=#1\textwidth]{#2}}
\newcommand{\mb}[1]{\mathbf{#1}}
\newcommand{\eq}[2]{\vspace{#1}{\Large\begin{displaymath}#2\end{displaymath}}}

\begin{document}

\begin{frame}
  \maketitle
\end{frame}



\begin{frame}
  \frametitle{Files for You to Download}
  Download from the school website:
  \begin{enumerate}
  \item\texttt{20-thermodynamics.pdf}---This presentation. If you want to print
    the slides on paper, I recommend printing 4 slides per page.
  \item\texttt{20-Homework.pdf}---Homework assignment for Classes 19 and 20,
    which cover Fluid Mechanics and Thermodynamics
  \end{enumerate}

  \vspace{.2in}Please download/print the PDF file before each class. When you
  are taking notes, pay particular attention to things I say that aren't
  necessarily on the slides.
\end{frame}



\begin{frame}
  \frametitle{Review: Absolute Temperature Scale}
  \framesubtitle{aka the Kelvin Scale}
  Relationship between the absolute temperature scale and the Celsius scale is
  a constant:
  
  \eq{-.4in}{
    \boxed{T = T_C + 273.15}
  }
  
  \vspace{-.1in}Plotting pressure vs.\ temperature at \emph{constant volume}
  for gases gives a straight line that appear to intersect the $x$-axis at
  \SI{-273.15}{\celsius}:
  \begin{center}
    \begin{tikzpicture}[scale=1.1]
      \begin{scope}[rotate=25]
        \draw[thick,red,dashed](0,0)--(1,0);
        \draw[thick,red](1,0)--(5,0);
      \end{scope}
      \draw[->](0,0)--(5,0)
      node[pos=1,below]{$T_C$};
      \draw(0,0)--(0,.15)
      node[pos=0,below]{\footnotesize\SI{-273.15}{\celsius}};
      \draw[->](2,0)--(2,2.5) node[pos=1,left]{$P$};
    \end{tikzpicture}
  \end{center}
\end{frame}



\begin{frame}
  \frametitle{Review: Absolute Temperature Scale}
  \eq{.01in}{
    \boxed{T = T_C + 273.15}
  }
  \begin{center}
    \begin{tabular}{l|c|l}
      \rowcolor{pink}
      \textbf{Quantity}      & \textbf{Symbol} & \textbf{SI Unit} \\ \hline
      Absolute temperature  & $T$             & \si{\kelvin} (Kelvin) \\
      Temperature in degree Celsius & $T_C$  & \si{\celsius} is not an SI unit\\
    \end{tabular}
  \end{center}
  \begin{itemize}
  \item Developed by William Thomson (Lord Kelvin) and James Joule
  \item Until the 1960's, absolutely temperature used the unit ``degree kelvin''
  \item The temperature \emph{change} of \SI{1}{\kelvin} is the same as
    \SI{1}{\celsius}
  \item The two scales differ in where \emph{zero} is (called the
    ``null point'')
  \end{itemize}
\end{frame}
  



\begin{frame}
  \frametitle{Thermal Expansion}
  When temperature $T$ of an object with length $L$ increases, the object
  \emph{usually} expands. The \emph{thermal strain} ($\Delta L/L$) is
  proportional the change in temperature:
  
  \eq{-.2in}{
    \boxed{\frac{\Delta L}{L} =\alpha\Delta T}\quad\text{\normalsize where}\quad
    \boxed{\alpha=\frac{1}{L}\frac{dL}{dT}}
  }
  \begin{center}
    \begin{tabular}{l|c|l}
      \rowcolor{pink}
      \textbf{Quantity} & \textbf{Symbol} & \textbf{SI Unit} \\ \hline
      Length      & $L$  & \si{\metre} (meter) \\
      Temperature & $T$  & \si{\kelvin} (kelvin)\\
      \textbf{Coefficient of linear expansion} & $\alpha$ &
      \si{\per\kelvin} (per kelvin)
    \end{tabular}
  \end{center}

  $\alpha$ is independent of pressure for solids and liquids, but may vary
  with $T$.
\end{frame}


\begin{frame}
  \frametitle{Thermal Expansion}
  There is also a similar expression for
  \textbf{coefficient of volume expansion}:
  
  \eq{-.2in}{
    \boxed{\frac{\Delta V}{V}=\beta\Delta T}
    \quad\text{\normalsize where}\quad
    \boxed{\beta=\frac{1}{V}\frac{dV}{dT}}
  }

  \vspace{-.05in}$\beta$ is also independent of pressure for solids and liquids,
  but may vary with $T$.
  \begin{center}
    \begin{tabular}{l|c|l}
      \rowcolor{pink}
      \textbf{Quantity} & \textbf{Symbol} & \textbf{SI Unit} \\ \hline
      Volume      & $V$  & \si{\m^3} (cube meter) \\
      Temperature & $T$  & \si{\kelvin} (kelvin)\\
      Coefficient of volume expansion & $\beta$ & \si{\per\kelvin} (per kelvin)
    \end{tabular}
  \end{center}

  Careful application of calculus will show that for isotropic material (where
  $\alpha$ is the same in all direction)

  \eq{-.45in}{
    \beta = 3\alpha
  }
\end{frame}



\begin{frame}
  \frametitle{Ideal Gas Law for Low-Density Gases}

  \textbf{Boyle's Law}: Physicist Robert Boyle (1627-1691) discovered that,
  when a gas is allowed to expand or compress at \emph{constant temperature},
  the product of pressure $P$ and $V$ remain constant, i.e.:

  \eq{-.2in}{
    PV=\textrm{constant} %\quad\quad\text{\normalsize constant temperature}
  }

  \vspace{-.1in}We also know that at \emph{constant volume}, temperature is
  proportional to pressure. Our equation now modifies to:

  \eq{-.3in}{
    PV=CT
  }

  \vspace{-.2in}where ``C'' is some constant that is yet to be determined.
\end{frame}


\begin{frame}
  \frametitle{Ideal Gas Law}
  Thought experiment:
  \begin{itemize}
  \item Two identical containers with volume $V$ with same amount of same kind
    of gas at same pressure $P$ and temperature $T$.
  \item When the containers are combined and the molecules are free to
    move, $P$ and $T$ remain the same, but volume is increased by factor of 2.
  \end{itemize}

  $C$ must scale with the number of molecules of gas $N$, which modifies the
  equation to this, the \textbf{ideal gas law}:

  \eq{-.25in}{
    \boxed{PV=NkT}
  }

  \vspace{-.15in}The constant $k=\SI{1.381e-23}{\joule/\kelvin}$ is called
  \textbf{Boltzmann's constant}. It is found experimentally to have the same
  value for any kind or amount of gas.
\end{frame}



\begin{frame}
  \frametitle{Ideal Gas Law}
  The ideal gas law is more often written in terms of the number of
  \emph{moles} of gas $n$ and the \textbf{universal gas constant} $R$:
  
  \eq{-.2in}{
    \boxed{PV=nRT}
  }
\end{frame}


\begin{frame}
  \frametitle{Kinetic Theory of Gases}
  Assumptions for the kinetic theory of gases:
  \begin{enumerate}
  \item The gas consists of a large number of molecules that make
    \emph{elastic} collisions with each other and with the walls of the
    container.
  \item The molecules are separated, on average, by distances that are large
    compared to their diameters, and they exert no force (gravitational,
    electrostatic etc) on each other except when they collide.
  \item In the absence of external forces, there is no preferred position for a
    molecules in the container, and there is no preferred velocity vector.
  \end{enumerate}
\end{frame}


\begin{frame}
  \frametitle{Kinetic Energy}

  Pressure $P=F/A$ is created when gas molecules

  The average kinetic energy of an ensemble of gases is given by:
  
  \eq{-.2in}{
    \big\langle K\big\rangle=\frac{3}{2}NkT
  }
  
  %(Divide the equation by $N$ for a \emph{single} molecule.)
  Often it is
  advantageous to calculate the root mean square of velocity of the molecules,
  which is approximately the most statistically probable speed:

  \eq{-.2in}{
    v_\mathrm{rms}=\sqrt{\frac{3kT}{m}}
  }
\end{frame}



\begin{frame}
  \frametitle{Real Gases}
  Most gases behave like ideal gas at most ordinary pressures, but the
  equation breaks down when the density of gas is high and molecules are not
  far apart:
  \begin{itemize}
  \item pressure is sufficiently high
  \item temperature is low
  \end{itemize}
  In this situation the \textbf{van der Waal equation} provides a more accurate
  description of the behaviour of real gases:

  \eq{-.2in}{
    \boxed{\left(P+\frac{an^2}{V^2}\right)(V-bn)=nRT}
  }
\end{frame}


\begin{frame}
  \frametitle{Phase Diagrams}

  \begin{center}
    \pic{.5}{10-figure-31.png}
  \end{center}
\end{frame}


\begin{frame}
  \frametitle{First Law of Thermodynamics}

%  \begin{block}{First Law of Thermodynamics}
%    The change in the energy of a system is the sum of the work and heat
%    exchanged between a system and its surroundings. 
%  \end{block}
%  
  \eq{-.2in}{
    \boxed{\Delta U=Q+W}
  }
  
\end{frame}



%\begin{frame}
%  \frametitle{Pressure with Different Fluids}
%  \begin{columns}
%
%    \column{.3\textwidth}
%    \begin{tikzpicture}
%      \draw[ultra thick,->](-1,2)--(-1,4.5) node[pos=1,above]{$z$};
%
%      \fill[white!5!blue!80](0,0)   rectangle(3,1);
%      \fill[white!10!blue!60](0,1)   rectangle(3,2.5);
%      \fill[white!30!blue!45](0,2.5) rectangle(3,3);
%      \fill[white!40!blue!30](0,3)   rectangle(3,4.2);
%      \fill[white!50!blue!20](0,4.2) rectangle(3,5);
%      \node at (-.2,5) {$z=z_1$};
%      \node at (-.2,4.2) {$z_2$};
%      \node at (-.2,3) {$z_3$};
%      \node at (-.2,2.5) {$z_4$};
%      \node at (-.2,1) {$z_5$};
%      \node at (-.2,0) {$z_6$};
%
%      \node at (1.5,5.1) {\tiny Known pressure $z$};
%      \node at (1.5,4.3) {\tiny Oil, $\rho_O$};
%      \node at (1.5,3.1) {\tiny Water, $\rho_W$};
%      \node at (1.5,2.6) {\tiny Glycerine, $\rho_G$};
%      \node at (1.5,1.1) {\tiny Mercury, $\rho_M$};
%    \end{tikzpicture}
%    
%    \column{.7\textwidth}
%    For the fluid surface to remain static, the fluid pressure on both side of
%    the interface have to be equal. (Obviously!) In this example:
%
%    \vspace{-.45in}{\Large
%      \begin{align*}
%        p_2-p_1&=\rho_Og(z_2-z_1)\\
%        p_3-p_2&=\rho_Wg(z_3-z_2)\\
%        p_4-p_3&=\rho_Gg(z_4-z_3)\\
%        p_5-p_4&=\rho_Mg(z_5-z_4)\\\hline
%        p_5-p_1&=\sum \Delta p
%      \end{align*}
%    }
%  \end{columns}
%\end{frame}
%
%
%
%\begin{frame}
%  \frametitle{A Simple Example}
%  \begin{columns}
%
%    \column{.7\textwidth}
%    \textbf{Example 1:} An aquarium is filled with water. The lateral wall of
%    the aquarium is \SI{40}{cm} long and \SI{30}{cm} high. Using \SI{10}{m/s^2}
%    for the acceleration due to gravity, and \SI{1}{g/cm^3} for density of
%    water, the force on the lateral wall of the aquarium is:
%    \begin{enumerate}[(a)]
%    \item\SI{36}{N}
%    \item\SI{90}{N}
%    \item\SI{180}{N}
%    \item\SI{1500}{N}
%    \end{enumerate}
%
%    \column{.3\textwidth}
%    \pic{1}{home-fish-tank.jpg}
%  \end{columns}
%\end{frame}
%
%
%
%\begin{frame}
%  \frametitle{Example}
%  \textbf{Example 2:} Consider the hydraulic jack in the diagram. A person
%  stands on a piston that pushes down on a thin cylinder full of water. The
%  cylinder is connected via pipes to a wide platform on top of which rests a
%  1-ton (\SI{1000}{kg}) car. The area of the platform under the car is
%  \SI{25}{m^2}; the person stands on a \SI{0.3}{m^2} piston. What is the
%  lightest weight of a person who could successfully lift the car?
%  \begin{center}
%    \vspace{-.2in}
%    \pic{.4}{jack.png}
%    
%    \vspace{-.2in}{\tiny Believe it or not, there \emph{is} someone who draws
%      worse diagrams than Tim!}
%  \end{center}
%\end{frame}
%
%
%
%\begin{frame}
%  \frametitle{A ``Manometer'' Example}
%  \begin{columns}
%
%    \column{.55\textwidth}
%    \textbf{Example 3:} Pressure gauge $B$ is to measure the pressure at point
%    $A$ in a water flow, as shown in the figure on the right. If the pressure at
%    $B$ is \SI{87}{\kilo\pascal}, estimate the pressure at $A$, in
%    \si{\kilo\pascal}. Assume all fluids are at \SI{20}{\celsius}.
%
%    \vspace{.1in}The densities of water, mercury and SAE 30 oil are,
%    respectively:
%
%    \vspace{-.3in}
%    \begin{align*}
%      \rho_\mathrm{water}&=\SI{1000}{kg/m^3}\\
%      \rho_\mathrm{Hg}&=\SI{13600}{kg/m^3}\\
%      \rho_\mathrm{oil}&=\SI{890}{kg/m^3}
%    \end{align*}
%    
%    \column{.45\textwidth}
%    \pic{1}{mano.jpg}
%  \end{columns}
%\end{frame}
%
%
%
%%\begin{frame}
%%  \frametitle{Hydrostatic Example: Forces on a Hinge}
%%  \begin{columns}
%%
%%    \column{.45\textwidth}
%%    \begin{tikzpicture}
%%      \begin{
%%    \end{tikzpicture}
%%    
%%    \column{.55\textwidth}
%%    The gate in the figure on the left is \SI{5}{m} wide, is hinged at point
%%    $B$, and rests against a smooth wall at point $A$. Compute
%%    \begin{enumerate}[(a)]
%%    \item the force on the gate due to seawater pressure, and
%%    \item the horizontal force $P$ exerted by the wall at point $A$, and
%%    \item the reaction at the hinge $B$
%%    \end{enumerate}
%%  \end{columns}
%%\end{frame}
%
%
%
%\begin{frame}
%  \frametitle{Buoyancy}
%  \framesubtitle{Everything Floats a Little}
%  When an object is submerged inside a fluid (e.g.\ water, air, etc), the fluid
%  exerts a pressure at the surface of the object. We can integrate the pressure
%  over the entire surface area $S$ to find the total force $\mb{B}$ the fluid
%  exerts on the object.
%  \begin{center}
%    \pic{.35}{rock_fbvectors.jpg}
%  \end{center}
%\end{frame}
%
%
%
%\begin{frame}
%  \frametitle{Derivation of Buoyance Force}
%  Although the derivation required a lot of calculus, the expression of
%  buoyance force is straightforward (and \emph{this} is what you need to
%  remember):
%  
%  \eq{-.1in}{
%    \boxed{\mb{B} = \rho_\mathrm{fluid}gV\bm{\hat{k}}=
%      m_\mathrm{fluid}g\bm{\hat{k}}}
%  }
%  
%  where $\rho_\mathrm{fluid}$ is the density of the displaced fluid, and
%  $V$ is the volume displaced. This equation is known as
%  \textbf{Archimedes' principle}.
%  
%  \vspace{.25in}\textbf{Buoyance force has a magnitude that equals to the
%    weight of the fluid displaced by the submerged object, pointing upward.}
%\end{frame}
%
%
%
%\begin{frame}
%  \frametitle{An Easier Explanation of Buoyancy}
%  \framesubtitle{Not Much Calculus}
%  \begin{columns}
%    \column{.7\textwidth}
%    There is a simpler way to find the buoyance force, by taking an
%    infinitesimal ``tube'' of the object, and finding the pressure difference
%    between the top and bottom of the tube:
%
%    \vspace{-.5in}{\Large
%      \begin{align*}
%        \mb{B}&=\int (p_2-p_1)dA\\
%        &= \rho_\mathrm{fluid} g\int(z_2-z_1)dA\\
%        &=\rho_\mathrm{fluid} g V
%      \end{align*}
%    }
%
%    \vspace{-.2in}which is the same expression that we got with calculus.
%
%    \column{.3\textwidth}
%    \pic{1}{buoyancy.jpg}
%  \end{columns}
%\end{frame}
%
%
%
%\begin{frame}
%  \frametitle{Buoyancy}
%%  Buoyancy depends on:
%%  \begin{itemize}
%%  \item the density of the (displaced) fluid $\rho_\mathrm{fluid}$
%%  \item the volume of the fluid displaced $V$, and
%%  \item the local acceleration due to gravity $g$
%%  \end{itemize}
%  Note that buoyancy does not depend on:
%  \begin{itemize}
%  \item the mass of the immersed object, or
%  \item the density of the immersed object
%  \end{itemize}
%%\end{frame}
%%
%%\begin{frame}
%%  \frametitle{Buoyancy}
%  \vspace{.15in}Objects immersed in a fluid have an ``apparent weight''
%  $\mb{W}'$ that is reduced by the buoyance force:
%
%  \eq{-.2in}{
%    \mb{W}' = \mb{W}-\mb{B}=\rho'\mb{g}V
%  }
%  
%  where $\rho'=\rho_{\textrm{obj}}-\rho_{\textrm{fluid}}$ is the relative density
%\end{frame}
%
%%\begin{frame}
%%  \frametitle{Buoyancy}
%%  For a submerged object:
%%  \begin{center}
%%    \begin{tabular}{c|c|c|c}
%%      \rowcolor{pink}
%%      Densities	&
%%      $B>W_{\textrm{obj}}$ &
%%      $B=W_{\textrm{obj}}$ &
%%      $B<W_{\textrm{obj}}$ \\\hline
%%      $\rho_{\textrm{obj}}<\rho_{\textrm{fluid}}$ & object rises & float on surface & \\
%%      $\rho_{\textrm{obj}}=\rho_{\textrm{fluid}}$ & & neutral buoyancy & \\
%%      $\rho_{\textrm{obj}}>\rho_{\textrm{fluid}}$ & & & object sinks
%%    \end{tabular}
%%  \end{center}
%%\end{frame}
%
%
%\begin{frame}
%  \frametitle{How Submarines Work}
%  \framesubtitle{Like this?}
%  \begin{center}
%    \pic{.7}{EbHMOXk.jpg}
%  \end{center}
%\end{frame}
%
%
%
%\begin{frame}
%  \frametitle{How Submarines Work}
%  Like all ships, a submarine does not naturally sink due to buoyancy. When a
%  submarine submerges, water needed to be pumped into the  ``ballast tanks'' in
%  the hull to make the ship heavier.
%  \begin{center}
%    \pic{1}{risinglemur.jpg}
%  \end{center}
%\end{frame}
%
%
%
%\begin{frame}
%  \frametitle{Stable? Or unstable?}
%  Buoyance force $\mb{B}$ acts at the \emph{center of buoyancy} (CB) of a
%  submerged object
%  \begin{itemize}
%  \item The CB is the CG \emph{if the object has constant density} and is
%    fully submerged
%  \item The actual CG of the object may be at a different position
%  \item Sometimes the object is not fully submerged
%  \end{itemize}
%  
%  $\mb{F}_g$ and $\mb{B}$ may act at different points, creating a torque/moment
%  on the object
%  \begin{center}
%    %\vspace{-.15in}
%    \pic{.5}{stable-unstable.jpg}
%  \end{center}
%\end{frame}
%
%
%
%\begin{frame}
%  \frametitle{Example}
%  \begin{columns}
%
%    \column{.7\textwidth}
%    \textbf{Example 4:} An apple is held completely submerged just below the
%    surface of a container of water. The apple is then moved to a deeper point
%    in the water. Compared with the force needed to hold the water just below
%    the surface, what is the force needed to hold it at a deeper point?
%    \begin{enumerate}[(a)]
%    \item Larger
%    \item The same
%    \item Smaller
%    \item Impossible to determine
%    \end{enumerate}
%
%    \column{.3\textwidth}
%    \pic{1}{apple.jpg}
%  \end{columns}
%\end{frame}
%
%
%
%\begin{frame}
%  \frametitle{Example}
%
%  \begin{columns}
%
%    \column{.4\textwidth}
%    \pic{1}{hpa_b.jpg}
%
%    \column{.6\textwidth}
%    \textbf{Example 5:} A salvage ship tries to raise a sunken miniature
%    submarine from the bottom of Lake Superior. The submarine and its contents
%    have a mass of \SI{72000}{kg} and a volume of \SI{18.9}{m^3}. What upward
%    force must be applied to raise the submarine? The density of water is
%    \SI{1000}{kg/m^3}.
%    \begin{enumerate}[(a)]
%    \item\SI{1.8e5}{\newton}
%    \item\SI{2.0e5}{\newton}
%    \item\SI{4.8e5}{\newton}
%    \item\SI{5.2e5}{\newton}
%    \end{enumerate}
%    
%  \end{columns}
%\end{frame}
%
%
%
%\begin{frame}
%  \frametitle{Fluid Flow}
%
%  \begin{center}
%    As important as it is to understand hydrostatics,\\
%    it's way more interesting when the fluid is moving!
%  \end{center}
%\end{frame}
%
%
%\begin{frame}
%  \frametitle{Control Volume and Control Surfaces}
%  A control volume ``CV'' is a fixed volume in which fluid is able to flow in
%  and out of it. The surfaces of the control volume is called the control
%  surface ``CS''.
%  \begin{center}
%    \pic{.5}{CV-CS.jpg}
%  \end{center}
%\end{frame}
%
%\begin{frame}
%  \frametitle{Fluid Flow: Continuity}
%  In a CV, we can quantify how fluid mass changes inside:
%  \begin{center}
%    \textbf{rate of decrease in mass in the CV = mass flux out of the CV}
%  \end{center}
%
%  The fluid mass in the CV is the integral of density over the volume:
%
%  \eq{-.2in}{ \int_{CV}\rho dV }
%  
%  The rate of decrease is therefore the negative of the time derivative:
%  
%  \eq{-.25in}{
%    -\frac{\partial}{\partial t}\int_{CV}\rho dV
%  }
%\end{frame}
%
%
%
%\begin{frame}
%  \frametitle{Fluid Flow: Continuity}
%  The mass flux out of the surfaces of the control volume the volume flux
%  multiplied by the fluid density at the surface:
%
%  %Applying the divergence theorem, we can convert this surface interested
%
%  \eq{-.2in}{ \int_{CS}\rho\mb{v}\cdot d\mb{A} }
%  
%  Combining the LHS and RHS terms, we have the \emph{integral} form of the
%  continuity equation:
%
%  \eq{-.2in}{\boxed{
%      \int_{CV}\frac{\partial\rho}{\partial t}dV +
%      \int_{CS}\rho\mb{v}\cdot d\mb{A}=0
%  }}
%\end{frame}
%
%
%\begin{frame}
%  \frametitle{Fluid Flow: Continuity}
%  With some clever use of vector calculus, we get the \emph{differential form}
%  of the continuity equation:
%
%  \eq{-.2in}{
%    \frac{\partial\rho}{\partial t} + \nabla\cdot(\rho\mb{v})=0
%  }
%
%  \ldots which is still too difficult. So in AP Physics we usually only look at
%  simple cases where
%  \begin{itemize}
%  \item Steady flow (time independent)
%  \item Constant density
%  \item Flow perpendicular to control surfaces
%  \end{itemize}
%\end{frame}
%
%
%
%\begin{frame}
%  \frametitle{Inlet Outlet Flow}
%  \begin{center}
%    \pic{.5}{physicsbook_fluids_graphik_26.png}
%  \end{center}
%  In this example, the mass flowing at the inlet is the same as the flow out of
%  it:
%
%  \eq{-.2in}{\boxed{\rho_1 v_1A_1=\rho_2 v_2A_2}}
%  
%  \vspace{-.15in}And if fluid density is constant (incompressible flow), the
%  $\rho$ terms on both sides of the equation will cancel:
%
%  \eq{-.3in}{\boxed{v_1A_1=v_2A_2}}
%\end{frame}
%
%
%
%\begin{frame}
%  \frametitle{Example: Multiple Inlet \& Outlets}
%  \begin{columns}
%
%    \column{.43\textwidth}
%    \begin{tikzpicture}[scale=.9]
%      \fill[blue!20!gray!30](0,0) rectangle(4,4);
%      \fill[blue!20!gray!30](2.5,0) rectangle(3.25,-1);
%      \fill[blue!20!gray!30](-1,.5) rectangle(0,1.5);
%      \fill[blue!20!gray!30](4,3) rectangle(5,3.75);
%      \draw[very thick](-1,1.5)--(0,1.5)--(0,4)--(4,4)--(4,3.75)--(5,3.75);
%      \draw[very thick](-1,0.5)--(0,0.5)--(0,0)--(2.5,0)--(2.5,-1);
%      \draw[very thick](3.25,-1)--(3.25,0)--(4,0)--(4,3)--(5,3);
%      \node[draw] at (-.5,2.25) (a) {1};
%      \node[draw] at (4.5,4.5) (b) {2};
%      \node[draw] at (4,-.5) (c) {3};
%      \draw(-.5,1.5)--(a);
%      \draw(4.5,3.75)--(b);
%      \draw(3.25,-.5)--(c);
%      \draw[->,very thick,blue](-1.5,1)--(-.5,1);
%      \draw[->,very thick,blue](4.5,3.375)--(5.5,3.375);
%      \draw[<->,very thick,blue](2.875,-.5)--(2.875,-1.5);
%    \end{tikzpicture}
%
%    \column{.57\textwidth}
%    \textbf{Example 6:} Water at \SI{20}{\celsius} flows steadily through a
%    closed tank, as shown in the figure. As section 1, $D_1=\SI{6}{cm}$ and
%    the volume flow is \SI{100}{m^3/h}. At section 2, $D_2=\SI{5}{cm}$ and the
%    average velocity is \SI{8}{m/s}. If $D_3=\SI{4}{cm}$, what is
%    \begin{enumerate}
%    \item the flow rate $Q_3$ in \si{m^3/h}?
%    \item the average $v_3$ in \si{m/s}?
%    \end{enumerate}
%  \end{columns}
%\end{frame}
%
%
%
%\begin{frame}
%  \frametitle{Governing Equations for Fluid Dynamics}
%
%  To properly describe fluid flows, there are three conservation equations:
%  \begin{itemize}
%  \item continuity
%  \item momentum, and
%  \item energy
%  \end{itemize}
%\end{frame}
%
%\begin{frame}
%  \frametitle{Fluid Flow: Momentum \& Energy Equations}
%  In the momentum equation, the rate of decrease of total fluid momentum inside
%  the control volume CV is the net momentum flux of the fluid out of the CV
%  all the forces (pressure, body, shear) acting on the fluid:
%
%  \eq{-.15in}{
%    \frac{\partial(\rho\mb{v})}{\partial t} +
%    \nabla(\rho\mb{v}\otimes\mb{v}) = -\nabla p +\mb{f}+\mu\nabla^2\mb{v}
%  }
%  
%  \vspace{-.1in}(This is even more complicated than the continuity equation, so
%  thankfully you won't need this equation for AP Physics!)
%
%  \vspace{.2in}The energy equation follows a similar thought process as the
%  previous two equations, but the terms are even more complicated.
%
%\end{frame}
%
%
%
%\begin{frame}
%  \frametitle{Navier-Stokes Equations}
%  Together, the three conservation equations are called the
%  \textbf{Navier-Stokes equations}. In differential form, they are usually
%  written as:
%  
%  \vspace{-.4in}{\Large
%    \begin{align*}
%      \frac{\partial\rho}{\partial t} + \nabla\cdotp(\rho\mb{v})&=0\\
%      \frac{\partial(\rho\mb{v})}{\partial t} +
%      \nabla(\rho\mb{v}\otimes\mb{v}) &= -\nabla p +\mb{f}+\mu\nabla^2\mb{v}\\
%      \frac{\partial e}{\partial t} + \nabla\cdotp(e\mb{v})&=
%      -\nabla\cdotp p +\frac{1}{Re\; Pr}\nabla q +
%      \frac{1}{Re}\nabla\cdotp(\tau\cdotp\mb{v})
%    \end{align*}
%
%  }
%
%  Even for a 2nd-year engineering student experienced with calculus, solving
%  the N-S equations is still a daunting task, so let's make some assumptions!
%\end{frame}
%
%
%
%\begin{frame}
%  \frametitle{Let's Make Some Assumptions}
%  \framesubtitle{For an ``ideal fluid flow''}
%  The flow is \textbf{steady}
%  \begin{itemize}
%  \item Flow is ``time independent'', i.e.\ does not change with time
%  \item All derivatives w.r.t.\ time are zero
%  \end{itemize}
%
%  The flow is \textbf{inviscid}
%  \begin{itemize}
%  \item The fluid has no viscosity
%  \item No friction between the fluid and the surrounding, and therefore
%  \item No shear stresses on the fluid
%  \item Only forces are pressure at the surface, and body forces from gravity
%  \end{itemize}
%
%  The flow is \textbf{incompressible}
%  \begin{itemize}
%  \item Density is constant throughout
%  \item The compressibility of fluid usually depends on its flow velocity,
%    at Mach number of $M\approx 0.3$, fluid becomes incompressible
%  \end{itemize}
%\end{frame}
%
%
%
%\begin{frame}
%  \frametitle{Let's Make Some Assumptions}
%  \framesubtitle{For an ``ideal fluid flow''}
%  We will also assume that
%  \begin{itemize}
%  \item there is \textbf{no shaft work} done along the streamline
%  \item there is \textbf{no heat transfer} along the streamline
%  \end{itemize}
%  Then the N-S equations reduces to the
%  \textbf{Bernoulli equation}
%  
%  \eq{-.1in}{\boxed{
%      p_1+\frac{1}{2}\rho v_1^2 + \rho gz_1=
%      p_2+\frac{1}{2}\rho v_2^2 + \rho gz_2
%    }
%  }
%  
%  The term $\displaystyle\frac{1}{2}\rho v^2$ is called ``dynamic pressure'',
%  and $\rho gz$ is the ``hydrostatic pressure''
%\end{frame}
%
%
%
%%\begin{fame}
%%  \frametitle{Flow Continuity}
%%  Usually the continuity equation is written in \emph{differential form}, by
%%  applying the \emph{divergence theorem} to the flux term:
%%
%%  \vspace{-.3in}{
%%    \begin{align*}
%%      \int_{CV}\frac{\partial\rho}{\partial t}dV+
%%      \int_{CS}\rho\mb{v}\cdot d\mb{A}=&0\\
%%    \end{align*}
%%  }
%%\end{frame}
%%  \eq{-.15in}{
%%    \boxed{\Phi_\mathrm{V}=\int\mb{V}\cdot d\mb{A}}
%%  }
%%    
%%  \vspace{-.1in}where $\mb{V}$ is the velocity (vector field) at the surface,
%%  and $d\mb{A}$ is the infinitesimal area pointing \textbf{outwards}. We can
%%  also expressed volume flux using the outward normal unit vector
%%  $\hat{\mb{n}}$:
%%
%%  \eq{-.15in}{
%%    \boxed{\Phi_\mathrm{V}=\int\mb{V}\cdot\hat{\mb{n}}dA}
%%  }
%%\end{frame}
%%
%%\begin{frame}
%%  \frametitle{Bernoulli Equation}
%%
%%  \eq{-.01in}{\boxed{
%%      p_1+\frac{1}{2}\rho v_1^2 + \rho gz_1=
%%      p_2+\frac{1}{2}\rho v_2^2 + \rho gz_2
%%  }}
%%\end{frame}
%%
%%
%%\begin{frame}
%%  \frametitle{Bernoulli Equation}
%%
%%  \eq{-.01in}{\boxed{
%%      p_1+\frac{1}{2}\rho v_1^2 + \rho gz_1=
%%      p_2+\frac{1}{2}\rho v_2^2 + \rho gz_2
%%  }}
%%
%%  Bernoulli's equation is valid when
%%  \begin{itemize}
%%  \item the flow is \textbf{steady} (independent of time)
%%  \item the flow is \textbf{incompressible}--compressibility (i.e. changes in
%%    density of the fluid) effects are negligible for Mach number $M<0.30$
%%  \item the flow \textbf{along a single streamline}
%%  \item there is \textbf{no shaft work} done along the streamline between 1 and
%%    2
%%  \item there is \textbf{no heat transfer} along the streamline between 1 and 2
%%  \end{itemize}
%%\end{frame}
%
%
%
%\begin{frame}
%  \frametitle{Bernoulli Equation}
%
%  Regions where Bernoulli equation is valid:
%  \begin{center}
%    \pic{.8}{bernoulli.jpg}
%  \end{center}
%\end{frame}
%
%
%
%\begin{frame}
%  \frametitle{Example}
%
%  \textbf{Example 7:} Find a relation between the nozzle discharge velocity
%  $V$ and the tank free-surface height $h$. Assume frictionless flow.
%  \begin{center}
%    \pic{.4}{EGL.jpg}
%  \end{center}
%
%  \vspace{-.2in}{
%    \footnotesize (The line labelled ``EGL'' is called the
%    ``energy grade line'', or the ``Bernoulli head'', given by the equation
%    $h_0=z+p/\rho g+v^2/2g$. In the region where Bernoulli equation is valid,
%    EGL is a constant.)\par
%  }
%\end{frame}
%
%
%
%\begin{frame}
%  \frametitle{How Does A Wing Work?}
%
%  When air flows past a wing, a force is generated
%\end{frame}
%
%\end{document}
