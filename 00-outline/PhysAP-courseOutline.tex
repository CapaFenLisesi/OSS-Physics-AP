\documentclass[11pt]{article}
\usepackage[margin=.8in,letterpaper]{geometry}

\usepackage{times}
\usepackage{enumitem}
\usepackage{titlesec}
\renewcommand{\familydefault}{\sfdefault}

\titleformat*{\section}{\large\bfseries}

\title{\vspace{-.2in}\textbf{Advanced Placement Physics}}
\author{Instructor: Dr.\ Timothy Leung (\texttt{tleung@olympiadsmail.ca})}
\date{Fall/Winter 2019}

\begin{document}
\maketitle

\subsection{Class Time}
Saturdays 4:10pm--6:40pm (Starts November 2)
%Saturdays \& Sundays 7:00--10:00pm (July and August)\\
%Class time in September TBA




\subsection{Course Material}
\begin{itemize}[itemsep=0pt,leftmargin=12pt]
\item No textbook required
\item Please check the school website for lecture slides, homework, and other
  resources
\item Students are expected to bring the following to each class:
  \begin{itemize}[noitemsep,topsep=0pt]
  \item A pen/pencil for note-taking
  \item Paper/notebook/binder
  \item A scientific calculator for working in-class example problems
  \end{itemize}
\end{itemize}




\subsection{Classroom Expectations}
Students are expected to:
\begin{itemize}[itemsep=0pt,leftmargin=12pt]
\item Be in your seat and ready to learn and participate during class
\item Stay on task without disturbing or distracting others
\item Raise your hand if you have any questions or comments and wait to be
  called. Don't wait too long before you ask a question
\item If you need to leave the class early, your parent needs to pick you up at
  the classrom door
\item Be respectful for yourself, others, and the facilities; act in
  a responsible manner in everything you do
\end{itemize}




\subsection{Homework Expectation}
\begin{itemize}[itemsep=0pt,leftmargin=12pt]
\item Homework is assigned approximately every \emph{two} weeks, depending on
  the course material
\item Late homework is accepted
\item For free-response questions:
  \begin{itemize}[noitemsep]
  \item Show \emph{all} work by providing complete and organized steps. Answer
    the questions as if the reader is learning the concept from you, not as if
    s/he already understands it.
  \item If a question requires you to \emph{explain}, please do so using
    complete sentences with supporting detail.
  \item Proper math format must be used, e.g.\ proper use of ``='' sign, units,
    etc.
  \item Circle or box all your final answers.
  \end{itemize}
\item Some of the more difficult questions will be taken up during class.
  However, this does \emph{not} mean you don't need to do your homework at
  home. Always do your best.
\end{itemize}


\subsection*{Pre-requisites}
\begin{itemize}[itemsep=0pt,leftmargin=12pt]
\item\textbf{Physics 11 and 12:} Student will need to be competent in all the
  topics covered in the high-school level courses. Many topics from Phyiscs
  11 and 12 are covered more in-depth in this course.
\item\textbf{Calculus:} The two ``C'' exams are calculus based, and students
  are required to perform basic differentiation and integration.
\item\textbf{Vectors:} Students need to have basic understanding of vector
  operations, including addition and subtraction, as well as dot products and
  cross products.
\end{itemize}



\subsection*{Course Outline}
\begin{enumerate}[itemsep=0pt,leftmargin=15pt]
\item Topics in \emph{AP Physics C: Mechanics}
  \begin{enumerate}[itemsep=0pt,leftmargin=18pt]
  \item Kinematics
  \item Dynamics
  \item Work and energy
  \item Momentum, impulse and collisions
  \item Center of mass
  \item General circular motion and angular momentum
  \item Simple harmonic motion %--general equation of oscillatory systems,
    %pendulums and spring-mass systems
  \item Universal gravitation and planetary motion
  \item\textbf{Practice AP Physics C: Mechanics exam}
  \end{enumerate}
\item Topics in \emph{AP Physics C: Electricity and Magnetism} (``E\&M'')
  \begin{enumerate}[itemsep=0pt,leftmargin=18pt]
  \item Electrostatics
  \item Gauss's law
  \item Capacitance
  \item Magnetism
  \item Inductance
  \item Circuit analysis (RC, RL, LC and RLC circuits)
  \item Maxwell's equations and electromagnetic wave
  \item\textbf{Practice AP Physics C: E\&M exam}
  \end{enumerate}
\item Additional topics in \emph{AP Physics 1} and \emph{AP Physics 2}
  \begin{enumerate}[itemsep=0pt,leftmargin=18pt]
  \item Fluid dynamics
  \item Thermal physics
  %\item Mechanical waves
  %\item Light and optics
  \item Special relativity
  \item Quantum mechanics
  \item\textbf{Practice AP Physics 2 Exam}
  \end{enumerate}
\end{enumerate}
\end{document}


