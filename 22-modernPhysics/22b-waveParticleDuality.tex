\documentclass[12pt,compress,aspectratio=169]{beamer}

\mode<presentation>
{
%  \usetheme{Warsaw}
  \usetheme{Singapore}
%  \setbeamertemplate{navigation symbols}{} % suppress nav bar
%  \setbeamercovered{transparent}
}
\usefonttheme{professionalfonts}
\usepackage{graphicx}
\usepackage{tikz}
\usepackage{amsmath}
\usepackage{mathpazo}
\usepackage[scaled]{helvet}
\usepackage{xcolor,colortbl}
\usepackage{hyperref}
\usepackage{siunitx}
\sisetup{detect-all}


\title[Quantum]{Class 22b.\ Wave-Particle Duality}
\subtitle{Advanced Placement Physics}
\author[TML]{Dr.\ Timothy Leung}
\institute{Olympiads School}
\date{Winter/Spring 2018}


\newcommand{\mb}[1]{\mathbf{#1}}
\newcommand{\pic}[2]{\includegraphics[width=#1\textwidth]{#2}}
\newcommand{\eq}[2]{\vspace{#1}{\Large\begin{displaymath}#2\end{displaymath}}}


\begin{document}

\begin{frame}
  \maketitle
\end{frame}


\begin{frame}
  \frametitle{You Have Seen This Before}
  A significant portion of these slides are condensed from Physics 12. For
  some of you this is a review.
\end{frame}



%\section{Kirchkoff}
%\begin{frame}
%  \frametitle{Blackbody Radiation}
%  \begin{columns}
%    \column{0.25\textwidth}
%    \pic{1}{graphics/Black-body_realization.png}
%    \column{0.75\textwidth}
%    \begin{itemize}
%    \item An idealized physical object that absorbs all incident electromagnetic
%      radiation, regardless of frequency or angle of incidence
%    \item A black body is in thermal equilibrium, i.e. all absorbed radiation
%      (energy) is immediately radiated back
%    \item A black body at room temperature appears black, as most of the energy
%      it radiates is infrared and cannot be perceived by the human eye
%    \item Thermal radiation spontaneously emitted by many ordinary objects can
%      be approximated as blackbody radiation 
%    \item The concept was coined by Gustav Kirchkoff in 1860.
%    \end{itemize}
%  \end{columns}
%\end{frame}
%
%\begin{frame}
%  \frametitle{Raleigh-Jeans Law and the Ultraviolet Catastrophe}
%  \begin{columns}
%    \column{0.55\textwidth}
%    \begin{itemize}
%    \item Based on classical thermodynamics
%      
%      \vspace{-0.2in}
%      {\large
%        \begin{displaymath}
%          P(\lambda,T)=8\pi kT\lambda^{-4}
%        \end{displaymath}
%      }
%
%      \vspace{-0.2in}
%      {\footnotesize $T$=temperature, $\lambda$=wavelength,
%        $k$=Boltzmann's constant}
%    \item Agrees with experimental results for long wavelengths
%    \item But disagrees violently for short wavelengths:
%      \begin{itemize}
%      \item Shorter wavelengths (e.g. ultraviolet waves) seem to have infinite
%        intensity
%      \item Known as \textbf{``Ultraviolet catastrophe''}
%      \end{itemize}
%    \end{itemize}
%    \column{0.45\textwidth}
%    \pic{1}{graphics/1280px-Black_body.png}
%  \end{columns}
%\end{frame}
%
%\begin{frame}
%  \frametitle{We're Dead, Aren't We}
%  According to classical thermodynamics, we should all be dead. But yet we
%  are not. Why? How?
%\end{frame}
%
%\section{Planck}
%\begin{frame}
%  \frametitle{``Quantization'' of Energy}
%  \begin{columns}
%    \column{0.3\textwidth}
%    \pic{1}{graphics/20973-050-F6EEBFF1.jpg}\\
%    Max Planck
%    \column{0.67\textwidth}
%    \begin{itemize}
%    \item Made a strange modification in the classical calculations
%    \item Derived a function of $P(\lambda,T)$ that agreed with experimental
%      data for all wavelengths
%    \item First found an empirical function to fit the data
%    \item Then searched for a way to modify the usual calculations
%    \item Energy emitted by black body not continuous but discrete
%    \item When energy is emitted from the harmonic oscillator, it drops to the
%      next lower energy level 
%    \end{itemize}
%  \end{columns}
%\end{frame}
%
%\begin{frame}
%  \frametitle{Quantization of Energy}
%  \begin{columns}
%    \column{0.5\textwidth}
%    {\Huge
%      \begin{displaymath}
%        \boxed{E=hf}
%      \end{displaymath}
%    }
%
%    \vspace{-0.2in}
%    where
%    \begin{itemize}
%    \item $h=6.626\e{-34}J\cdot s$ is ``Planck's constant'', and 
%    \item $f$ is the frequency
%    \end{itemize}
%    When a black body emits radiation, it has to drop down one or more levels
%    and emit a unit of energy equal to the difference between the allowed
%    energy levels of the oscillator. 
%    \column{0.5\textwidth}
%    \pic{1}{graphics/quantum.png}
%%  As for his formula, it's called Planck's Law:
%%
%%  \vspace{-0.3in}
%%  {\Huge
%%    \begin{displaymath}
%%      \boxed{P(\lambda,T)=\frac{2hc^2}{\lambda^5}\frac{1}{e^{\frac{hc}{\lambda k_BT}}-1}}
%%    \end{displaymath}
%%  }
%  \end{columns}
%\end{frame}
%
%\section{Maxwell}
%


\begin{frame}
  \frametitle{Maxwell's Equations in a Vacuum}
  \framesubtitle{Everything Comes Back to This}

  \vspace{-.3in}{\Large
    \begin{align*}
      \nabla\cdot\mb{E} &= 0\\
      \nabla\cdot\mb{B} &= 0\\
      \nabla\times\mb{E} &=-\frac{\partial\mb{B}}{\partial t}\\
      \nabla\times\mb{B} &=\mu_o\varepsilon_o\frac{\partial\mb{E}}{\partial t}
    \end{align*}
  }
  
  \vspace{-.15in}Disturbances in $\mb{E}$ and $\mb{B}$ travel as an
  ``electromagnetic wave'', with a speed:

  \eq{-.25in}{
    c=\frac{1}{\sqrt{\varepsilon_0\mu_0}}=\SI{299792458}{m/s}
  }
\end{frame}


\begin{frame}
  \frametitle{Maxwell's Equations}
  Physicists already have an estimate of the speed of light (within about
  \SI{10}{\percent}, so is light an electromagnetic wave then?
  \begin{itemize}
  \item In order to prove that light is an electromagnetic wave, we must
    generate an alternating current with a frequency of \SI{e14}{\hertz}
  \item Technology of that time can only generate frequencies around
    \SI{e8}{\hertz} (already much higher than the \SI{60}{\hertz} that our
    electrical outlet uses, but still \num{e6} times too low)
  \end{itemize}
\end{frame}



\begin{frame}
  \frametitle{The Spark Gap Experiment}
  German physicist Heinrich Hertz devised a ``spark gap experiment'' to
  generate high frequencies
  \begin{center}
    \pic{.5}{Hertz_exp_2.png}
  \end{center}
  \begin{itemize}
  \item Produced EM waves with frequency \SI{e14}{\hertz}
  \item Also showed that light has the same wavelengths as predicted by
    Maxwell's equations
  \end{itemize}
\end{frame}


\begin{frame}
  \frametitle{Photoelectric Effect}
  \begin{itemize}
  \item Terse remark in Hertz's results:\\
    \emph{It is essential that the pole surfaces of the spark gap should be
      frequently repolished to ensure reliable operation of the spark.}
  \item This is now known as the \textbf{photoelectric effect} caused by
    ultraviolet radiation
  \item Physicist who repeated his experiments did not have an explanation
  \end{itemize}
\end{frame}



\begin{frame}
  \frametitle{Photoelectric Effect}
  When electromagnetic waves (e.g. light) hits certain metals, electrons are
  knocked off the surface

  \pic{1}{73bacc9f2bf571752483a89ef6c61a94f07470f7.png}
  \begin{itemize}
  \item Increasing intensity of light knocked off more electrons, but doesn't
    change their kinetic energy, but
  \item Changing the frequency of the light did change $K$ though, although
  \item Below a certain frequency, \emph{no} electrons were emitted
  \end{itemize}
\end{frame}


\begin{frame}
  \frametitle{The Photon: Packets of Energy}
  \begin{itemize}
  \item Light is not a continuous wave, but
  \item A collection of discrete energy packets (photons)
  \item Each photon has energy $E=hf$
  \end{itemize}

  \eq{-.3in}{
    \boxed{K_\mathrm{max}=
      \begin{cases}
        hf-\varphi & \text{if }hf>\varphi\\
        0          & \text{otherwise}
      \end{cases}
    }
  }
  \begin{center}
    \begin{tabular}{l|c|l}
      \rowcolor{pink}
      \textbf{Quantity} & \textbf{Symbol} & \textbf{SI Unit} \\ \hline
      Maximum kinetic energy of ``photoelectrons'' & $K$ & \si{\joule}\\
      Planck's constant           & $h$   & \si{\joule.\second}\\
      Frequency of the EM wave    & $f$   & \si{\hertz}\\
      Work function of the metal  & $\varphi$ & \si{\joule}
    \end{tabular}
  \end{center}
\end{frame}

\begin{frame}
  \frametitle{Work Function $\varphi$}
  \begin{center}
    \pic{.6}{550px-Photoelectric_effect_diagram.png}
  \end{center}
  Slope is $h$ no matter what metal it is.
\end{frame}

\begin{frame}
  \frametitle{Work Functions of Different Materials}
  The work function $\varphi$ depends on the metal.
  \begin{columns}
    \column{.4\textwidth}
    \pic{1}{work-function.png}
    
    \column{.6\textwidth}
    Work Function for Common Metals:
    \begin{itemize}
    \item The minimum energy required to remove an electron from a solid to a
      point immediately outside the solid surface.
    \end{itemize}
  \end{columns}
\end{frame}


\begin{frame}
  \frametitle{Compton Scattering}
  \begin{itemize}
  \item American physicist Arthur Compton studied x-ray scattering by
    free electrons
  \item Classical theory cannot account for the scattering behaviour
  \item Frequency shift only depends on scattering angle
  \item Prediction possible if treating the x-ray as photons with
    momentum--just like a particle
  \end{itemize}

  \eq{-.3in}{
    \boxed{p=\frac{E}{c}=\frac{hf}{c}=\frac{h}{\lambda}}
  }
\end{frame}

\begin{frame}
  \frametitle{Compton Scattering}
  \begin{center}
    \pic{.5}{compton2.png}
  \end{center}
  If we treat the x-ray as a photon with momentum $p=h/\lambda$ then we can
  use Newton's laws of motion to predict both the recoil electron and scattered
  x-ray!
\end{frame}

\begin{frame}
  \frametitle{Momentum of a Photon}
  The momentum of a photon is prpoortional to Planck's constant and 
  inversely proportional to its wavelength.

  \eq{-.2in}{
    \boxed{p=\frac{h}{\lambda}}
  }

  \begin{center}
    \begin{tabular}{l|c|l}
      \rowcolor{pink}
      \textbf{Quantity} & \textbf{Symbol} & \textbf{SI Unit} \\ \hline
      Momentum          & $p$ & \si{\kilo\gram.\metre/\second}\\
      Planck's constant & $h$ & \si{\joule.\second}\\
      Wavelength        & $\lambda$ & \si{\metre} (meters) \\
    \end{tabular}
  \end{center}

  This is an odd expression, which treats photon both as a particle (with
  momentum) and a wave (with a wavelength $\lambda$).
\end{frame}

%\begin{frame}
%  \frametitle{Example Problem}
%  \textbf{Example 1}: Calculate the momentum of a photon of light that has
%  frequency of $5.09\e{14}\mathrm{Hz}$.
%\end{frame}



\begin{frame}
  \frametitle{Matter Waves}
%  \begin{columns}
%    \column{0.3\textwidth}
%    \pic{1}{graphics/76562-004-66881FD5.jpg}\\
%    Louis De Broglie
%    \column{0.7\textwidth}
%    \begin{itemize}
  \textbf{If electromagnetic waves are really particles of energy, then are
    particles (e.g. electrons) a wave of some sort?}
  \begin{itemize}
  \item The De Broglie hypothesis in 1924: a particle can also have a
    wavelength
  \item Confirmed by the Davisson-Germer Experiment in 1927 (beam of electron
    scattering on nickel crystal surface)
  \end{itemize}
%  \end{columns}
\end{frame}



\begin{frame}
  \frametitle{Electron Interference}
  \begin{columns}
    \column{.82\textwidth}
    If I perform a double-slit experiment with a beam of electrons, will I get
    an interference pattern?
    \begin{center}
      \pic{.7}{CNX_Chem_06_03_Electrnin.jpg}
    \end{center}

    \column{.18\textwidth}
    \pic{1}{206px-Double-slit_experiment_results_Tanamura_2.jpg}
  \end{columns}
\end{frame}



\begin{frame}
  \frametitle{De Broglie Wavelength}
  If matter is also a wave, then what would be its wavelength? Let's solve
  momentum equation for $\lambda$:

  \eq{-.3in}{
    p=\frac{h}{\lambda}\;\;\rightarrow\;\;
    \lambda=\frac{h}{p}\;\;\rightarrow\;\;\boxed{\lambda=\frac{h}{mv}}
  }

  \vspace{-.1in}
  \begin{center}
    \begin{tabular}{l|c|l}
      \rowcolor{pink}
      \textbf{Quantity} & \textbf{Symbol} & \textbf{SI Unit} \\ \hline
      Wavelength of a particle & $\lambda$ & \si{\metre} (meters) \\
      Planck's constant & $h$  & \si{\joule.\second} (joule seconds) \\
      Mass              & $m$  & \si{\kilo\gram} (kilograms) \\
      Velocity          & $v$  & \si{\metre/\second} (meters per second)
    \end{tabular}
  \end{center}
\end{frame}

\begin{frame}
  \frametitle{Heisenberg Uncertainty Principle}
  Because of the wave properties of particles, you can never be completely
  certain of the relationship between an object's momentum $p$ and position
  $x$:

  \eq{-.1in}{
    \boxed{\Delta p\Delta x\leq \frac{1}{2}\hbar}
  }
  
  where $\hbar$ is the \textbf{reduced Plank constant}, or
  \textbf{Dirac constant}:

  \eq{-.1in}{
    \hbar=\frac{h}{2\pi}=\SI{1.054e-34}{\joule.\second}
  }
\end{frame}

\begin{frame}
  \frametitle{Bohr Atomic  Model}
  \begin{itemize}
  \item The ``orbital'' model of electrons does not work, because
  \item As the electron orbits a nucleus, it radiates EM radiation, and lose
    energy
  \item The orbit will eventually collapse
  \item Bohr postulated that electron can move in certain ``non-radiating''
    orbits, corresponding to energy levels:

    \eq{-.1in}{
      \boxed{E_n=-\frac{k^2e^4m}{2\hbar^2}\frac{Z^2}{n^2}}
    }

  \item From the wave-particle duality perspective, the ``orbits'' correspond
    more to a standing wave around the nucleus
    (remember that a standing wave does not lose energy)
  \end{itemize}
\end{frame}



\begin{frame}
  \frametitle{Hydrogen Emission}
  \begin{columns}

    \column{.6\textwidth}
    \begin{itemize}
    \item Lyman series:
      \begin{itemize}
      \item the EM emissions when the electrons drop from a higher energy state
        ($E_n$) to the ground state $n=1$ (i.e.\ $E_1$)
      \item The frequency is given by:
        
        \eq{-.2in}{
          f=\frac{E_1-E_n}{h}
        }
      \item We can apply universal wave equation to get the wavelengths
      \end{itemize}
    \item Balmer series--dropping to $E_2$
    \item Paschen series--dropping to $E_3$
    \end{itemize}

    \column{.4\textwidth}
    \pic{1}{400px-Hydrogen_transitions.png}
  \end{columns}
\end{frame}

\end{document}
