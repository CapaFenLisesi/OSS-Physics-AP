\documentclass[12pt,aspectratio=169]{beamer}

\mode<presentation>
{
  \usetheme{Singapore}
 %\setbeamersize{text margin left=.6cm,text margin right=.6cm}
%  \setbeamertemplate{navigation symbols}{} % suppress nav bar
%  \setbeamercovered{transparent}
}
\usefonttheme{professionalfonts}
\usepackage{graphicx}
\usepackage{tikz}
\usepackage{amsmath}
\usepackage{mathpazo}
\usepackage[scaled]{helvet}
\usepackage{xcolor,colortbl}
\usepackage{siunitx}
\usepackage[siunitx]{circuitikz} % to draw circuits!

\sisetup{number-math-rm=\mathnormal}

\title{Class 13: Magnetism}
\subtitle{AP Physics}
\author[TML]{Dr.\ Timothy Leung}
\institute{Olympiads School}
\date{February 2018}

\newcommand{\pic}[2]{\includegraphics[width=#1\textwidth]{#2}}
\newcommand{\mb}[1]{\mathbf{#1}}
\newcommand{\eq}[2]{\vspace{#1}{\Large\begin{displaymath}#2\end{displaymath}}}


\begin{document}

\begin{frame}
  \maketitle
\end{frame}


\section[Intro]{Introduction}

\begin{frame}
  \frametitle{Files for You to Download}
  Download from the school website:
  \begin{enumerate}
  \item\texttt{12-Magnetism\_print.pdf}---The ``print version'' of this
    presentation. If you want to print on paper, I recommend printing 4 pages
    per side.
  \item\texttt{13-Homework.pdf}---Homework assignment for Class 12 and 13.
    Please note the new formatting style
  \end{enumerate}

  \vspace{.2in}Please download/print the PDF file before each class. There is
  no point copying notes that are already printed out for you. Instead, take
  notes on things I say that aren't necessarily on the slides.
\end{frame}

\end{document}
