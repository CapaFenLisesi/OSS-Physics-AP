\documentclass[12pt,aspectratio=169]{beamer}

\mode<presentation>
{
  \usetheme{Singapore}
 %\setbeamersize{text margin left=.6cm,text margin right=.6cm}
%  \setbeamertemplate{navigation symbols}{} % suppress nav bar
%  \setbeamercovered{transparent}
}
\usefonttheme{professionalfonts}
\usepackage{graphicx}
\usepackage{tikz}
\usepackage{amsmath}
\usepackage{mathpazo}
\usepackage[scaled]{helvet}
\usepackage{xcolor,colortbl}
\usepackage{siunitx}
\usepackage[siunitx]{circuitikz} % to draw circuits!

\sisetup{number-math-rm=\mathnormal}

\title{Class 13: Magnetism}
\subtitle{AP Physics}
\author[TML]{Dr.\ Timothy Leung}
\institute{Olympiads School}
\date{February 2018}

\newcommand{\pic}[2]{\includegraphics[width=#1\textwidth]{#2}}
\newcommand{\mb}[1]{\mathbf{#1}}
\newcommand{\eq}[2]{\vspace{#1}{\Large\begin{displaymath}#2\end{displaymath}}}
\newcommand{\protip}[1]{
  \begin{center}
    \fbox{
      \begin{minipage}{.95\textwidth}
        {\footnotesize
          \textbf{Protip: }#1
        }
      \end{minipage}
    }
  \end{center}
}

\begin{document}

\begin{frame}
  \maketitle
\end{frame}


%\section[Intro]{Introduction}

\begin{frame}
  \frametitle{Files for You to Download}
  Download from the school website:
  \begin{enumerate}
  \item\texttt{12-Magnetism\_print.pdf}---The ``print version'' of this
    presentation. If you want to print the slides on paper, I recommend
    printing 4 slides per page.
  \item\texttt{13-Homework.pdf}---Homework assignment for Class 12 and 13.
    Please note the new formatting style
  \end{enumerate}

  \vspace{.2in}Please download/print the PDF file before each class. When you
  are taking notes, pay particular attention to things I say that aren't
  necessarily on the slides.
\end{frame}

%\section[$\mb{B}$]{Magnetic Field}

\begin{frame}
  \frametitle{Review of Magnetic Field}
  \framesubtitle{Remember Physics 12?}
  \begin{itemize}
  \item A magnetism is generated by moving charged particles, e.g.
    a single charge, or an electric current
  \item It can also be generated by permanent magnets, or Earth
  \end{itemize}
\end{frame}

\begin{frame}
  \frametitle{Review of Magnetic Field}
  \begin{itemize}
  \item Magnetism affects other \emph{moving} charged particles
  \item The vector field is called the \textbf{magnetic field}
  \item Magnetic field has unit \textbf{tesla}
  \item Magnetic field lines have ends---they always run in a loop
  \end{itemize}
\end{frame}



%\begin{frame}
%  \frametitle{We're Also Familiar with Permanent Magnets}
%  \begin{itemize}
%  \item Generally made of iron, nickel, cobalt, some alloys of rare-earth
%    materials, some minerals (e.g.\ lodestone)
%  \item Atoms in these materials can be organized such that the electrons
%    are always creating a small current inside
%  \item We're told that the magnetic field runs from ``north'' to ``south''
%    pole, like the diagram shown below. If this is the case, someone lied to
%    you\ldots
%  \end{itemize}
%  \begin{center}
%    \pic{.35}{graphics/500px-VFPt_cylindrical_magnet_thumb.png}
%  \end{center}
%\end{frame}
%
%
%
%\begin{frame}
%  \frametitle{A Gigantic ``Permanent'' Magnet}
%  \begin{columns}
%    \column{0.7\textwidth}
%    \begin{itemize}
%    \item Earth is also a ``permanent'' magnet
%      \begin{itemize}
%      \item Magnetic field generated by electric currents in the conductive
%        material in its core
%      \item Current created by convection currents due to heat escaping from
%        the core
%      \end{itemize}
%    \item Magnetic field lines run from south to north
%    \item By our convention, our South Pole is actually the ``magnetic north
%      pole'', and our North Pole is the ``magnetic south pole''
%    \end{itemize}
%    \column{0.3\textwidth}
%    \pic{1}{graphics/phys4_2f_2.png}
%  \end{columns}
%\end{frame}
%
%
%
%\begin{frame}
%  \frametitle{What About Permanent Magnet?}
%  \begin{columns}
%    \column{0.7\textwidth}
%    \begin{itemize}
%    \item Magnetic fields don't actually run from a ``North'' pole to a
%      ``South'' pole
%    \item In fact, they run in a loop (see right)
%    \item The magnetic field lines continues inside the bar magnet
%    \end{itemize}
%    \column{0.3\textwidth}
%    \pic{1}{graphics/bar.png}
%  \end{columns}
%\end{frame}
%
\begin{frame}
  \frametitle{Magnetic Field Generated by a Moving Point Charge}
  \begin{center}
    \pic{.35}{pointchargeB.png}
  \end{center}
  
  A point charge generates an electric field $\mb{E}$. When it's moving, it
  also generates a magnetic field $\mb{B}$, given by the equation:

  \eq{-.2in}{
    \boxed{\mb{B}=\frac{\mu_o}{4\pi}\frac{q\mb{v}\times\hat{\mb{r}}}{r^2}}
  }

  The direction of $\mb{B}$ can be obtained by applying the ``right hand rule''
  if you are not confident with cross products.
  
%  \protip{

%  }
\end{frame}


\begin{frame}
  \frametitle{Reminder on the cross product}
  Whenever the ``right hand rule'' is mentioned, it usually means that the
  equation has a cross product in it. Just a reminder on a few properties of
  the cross product:
  \begin{itemize}
  \item If $\mb{C}=\mb{A}\times\mb{B}$, then $\mb{C}$ is perpendicular to both
    $\mb{A}$ and $\mb{B}$.
  \item The length of the cross product of two vectors is:
    
    \eq{-.3in}{
      |\mb{A}\times\mb{B}|=|\mb{A}||\mb{B}|\sin\theta
    }
    where $\theta$ is the angle between $\mb{A}$ and $\mb{B}$
  \item Cross products are anti-commutable:

    \eq{-.35in}{
      \mb{A}\times\mb{B}=-\mb{B}\times\mb{A}
    }
  \end{itemize}
\end{frame}

\begin{frame}
  \frametitle{Magnetic Field Generated by a Moving Point Charge}

  \eq{-.05in}{
    \boxed{\mb{B}=\frac{\mu_o}{4\pi}\frac{q\mb{v}\times\hat{\mb{r}}}{r^2}}
  }
  
  \vspace{-.4in}
    \begin{center}
      \begin{tabular}{l|c|l}
        \rowcolor{pink}
        \textbf{Quantity} & \textbf{Symbol} & \textbf{SI Unit} \\ \hline
        Magnetic field  & $\mb{B}$ & \si{\tesla} (teslas)\\
        Charge          & $q$      & \si{\coulomb} (coulombs)\\
        Velocity of the charge & $\mb{v}$ & \si{m/\second} (metres per second)\\
        Distance from the moving charge & $r$ & \si{\metre} (metres)\\
        Radial unit vector from the charge & $\hat{\mb{r}}$ & no units\\
        Permeability of free space & $\mu_0$ &
        \si{\tesla.\metre\per\ampere} (tesla metres per ampere)
      \end{tabular}
    \end{center}
    Permeability of free space is a constant with a value of
    $\mu_0=\SI{4\pi e-7}{\tesla.\metre\per\ampere}$

\end{frame}



\begin{frame}
  \frametitle{Magnetic Generated By a Current}
  \framesubtitle{Biot-Savart Law}
  \begin{columns}
    \column{.25\textwidth}
    \pic{1}{bsav.png}
    \column{.75\textwidth}
    An electric current is really many charges particles moving along a wire;
    each charge creating its own magnetic field.
    The total magnetic field in the wire is the integral of the contribution
    ($d\mb{B}$) of the current ($I$) from each infinitesimal sections
    ($d\mb{L}$) of the wire, given by the \textbf{Biot-Savart Law}:
  
    \eq{-.2in}{
      \boxed{d\mb{B}=\frac{\mu_o}{4\pi}\frac{Id\mb{L}\times\hat{\mb{r}}}{r^2}}
    }

    The magnetic field in the diagram goes \emph{into} the page
  \end{columns}
\end{frame}


\begin{frame}
  \frametitle{Magnetic Field Generated By an Infinitely Long Wire}
  \begin{columns}
    \column{0.2\textwidth}
    \pic{1}{magcur2.png}
    \column{0.7\textwidth}
    Integrating Biot-Savart law for a point at radial distance $r$ from an
    \emph{infinitely long wire} gives a simple expression:

    \eq{-.35in}{
      \boxed{\mb{B}=\frac{\mu_0(\mb{I}\times\mb{\hat{r})}}{2\pi r}}
      \quad\text{or}\quad
      \boxed{B=\frac{\mu_0I}{2\pi r}}
    }

    \vspace{-.15in}The magnitude and direction current ``vector'' $\mb{I}$ is
    straight forward

    \vspace{-.3in}
    \begin{center}
      \begin{tabular}{l|c|l}
        \rowcolor{pink}
        \textbf{Quantity} & \textbf{Symbol} & \textbf{SI Unit} \\ \hline
        Magnetic field      & $\mb{B}$ & \si{T} (teslas)\\
        Current             & $\mb{I}$ & \si{A} (amperes)\\
        Radial direction from the wire & $\mb{\hat{r}}$ & (no units)\\
        Radial distance from the wire  & $r$            & \si{m} (metres)\\
      \end{tabular}
    \end{center}
  \end{columns}
\end{frame}


\begin{frame}
  \frametitle{Current-Carrying Wire Loop}
  \begin{columns}
    \column{.35\textwidth}
    \pic{1}{curloo.png}
    \column{.65\textwidth}
    When we shape the current-carrying wire into a loop, we can (again) use
    the Biot-Savart law to find the magnetic field away from it.

    \vspace{.2in}
    One loop isn't very interesting (except when you're integrating Biot-Savart
    law) but what if we have many loops
  \end{columns}
\end{frame}


\begin{frame}
  \frametitle{Wounding Wires Into a Coil}
  \begin{itemize}
  \item A \textbf{solenoid} is when you wound a wire into a coil
  \item You create a magnet very similar to a bar magnet, with an effective
    north pole and a south pole
  \item Magnetic field inside the solenoid is uniform
  \item Magnetic field strength can be increased by the addition of an iron core
  \end{itemize}
  \begin{center}
    \pic{.5}{barsol.png}
  \end{center}
\end{frame}

\begin{frame}
  \frametitle{A Practical Solenoid}
  A practical solenoid usually has hundreds or thousands of turns:

  \vspace{-.2in}
  \begin{center}
    \pic{.45}{1020201515330450255.jpg}
  \end{center}

  \vspace{-.2in}
  This ``air core'' coil is used for high school and university experiments. It
  has approximately 600 turns of copper wire wound around a plastic core.
\end{frame}

\begin{frame}
  \frametitle{Magnetic Field Inside a Solenoid}
  \begin{columns}
    \column{.3\textwidth}
    \pic{1.1}{magneticfield4.png}
    \column{.7\textwidth}
    The magnetic field \textbf{inside} the solenoid given by:
    
    \eq{-.3in}{
      \boxed{B=\mu nI}
    }
    Direction of $\mb{B}$ determined by \textbf{right hand rule}
      \vspace{-.1in}
      \begin{center}
        \begin{tabular}{l|c|l}
          \rowcolor{pink}
          \textbf{Quantity} & \textbf{Symbol} & \textbf{SI Unit} \\ \hline
          Magnetic field intensity & $B$ & \si{T} (teslas)\\
          Number of coils          & $n$ & integer, no units\\
          Current                  & $I$ & \si{A} (amperes)\\
          Effective permeability & $\mu$ & \si{T.m/A}\\
        \end{tabular}
      \end{center}
  \end{columns}
\end{frame}


\begin{frame}
  \frametitle{So What Does the Magnetic Field Do?}
  \framesubtitle{In Classical Physics}
  \begin{columns}
    \column[t]{.3\textwidth}
    \begin{center}
      Gravitational Field $\mb{g}$
    \end{center}
    \begin{itemize}
    \item Generated by objects with mass
    \item Affects objects with mass
    \end{itemize}

    \column[t]{.3\textwidth}
    \begin{center}
      Electric Field $\mb{E}$
    \end{center}
    \begin{itemize}
    \item Generated by charged particles
    \item Affects charged particles
    %\item The charged particle can be at rest or moving
    \end{itemize}

    \column[t]{.4\textwidth}
    \begin{center}
      Magnetic Field $\mb{B}$
    \end{center}
    \begin{itemize}
    \item Generated by \emph{moving} charged particles
    \item Affects moving charged particles
    \end{itemize}
  \end{columns}
\end{frame}



\begin{frame}
  \frametitle{Lorentz Force Law}
  Since a moving charge or current create both electric and magnetic fields,
  another moving charge is therefore affected by both $\mb{E}$ and $\mb{B}$.
  The total effect is given by the \textbf{Lorentz Force Law}:

  \eq{-.2in}{
    \boxed{\mb{F}=q(\mb{E}+\mb{v}\times\mb{B})}
  }

  \vspace{-.1in}$\mb{F}_q=q\mb{E}$ is the electrostatic force, and
  $\mb{F}_M=q\mb{v}\times\mb{B}$ is the magnetic force.

  \vspace{-.15in}
  \begin{center}
    \begin{tabular}{l|c|l}
      \rowcolor{pink}
      \textbf{Quantity} & \textbf{Symbol} & \textbf{SI Unit} \\ \hline
      Total force on the moving charge & $\mb{F}$ & \si{N} (newtons) \\
      Charge                 & $q$      & \si{C}   (coulombs) \\
      Velocity of the charge & $\mb{v}$ & \si{m/s} (metres per second)\\
      Magnetic field         & $\mb{B}$ & \si{T}   (teslas) \\
      Electric field         & $\mb{E}$ & \si{N/C} (newtons per coulomb)
    \end{tabular}
  \end{center}

\end{frame}


\begin{frame}
  \frametitle{Force on a Current-Carrying Conductor in a Magnetic Field}

  Likewise, $\mb{B}$ exerts a force on another current-carrying conductor.

  \eq{-.2in}{
    \boxed{F_M=\mb{I}l\times\mb{B}}
  }
  
  \vspace{-.1in}
  \begin{center}
    \begin{tabular}{l|c|l}
      \rowcolor{pink}
      \textbf{Quantity} & \textbf{Symbol} & \textbf{SI Unit} \\ \hline
      Magnetic force on the conductor   & $\mb{F}_M$ & \si{N} (newtons) \\
      Electric current in the conductor & $\mb{I}$   & \si{A} (amperes) \\
      Length of the conductor           & $l$        & \si{m} (metres)\\
      Magnetic field strength           & $\mb{B}$   & \si{T} (teslas)
    \end{tabular}
  \end{center}
\end{frame}


\begin{frame}
  \frametitle{Magnetic Force on Two Current-Carrying Wires}
  \begin{columns}
    \column{.26\textwidth}
    \pic{1.08}{wirefor.png}
    \column{.74\textwidth}
    Two parallel current carrying wires are at a distance $r$ apart. Magnetic
    field at wire 2 from current $I_1$ has strength:

    \eq{-.2in}{
      B_1=\frac{\mu_0I_1}{2\pi r}
    }

    \vspace{-.1in}which is constant everywhere along wire 2. The force of
    $B_1$ on $I_2$ is:

    \eq{-.4in}{
      F=I_2LB_1=\frac{\mu_0I_1I_2L}{2\pi r}
      \;\rightarrow\;
      \boxed{\frac{F}{L}=\frac{\mu_0I_1I_2}{2\pi r}}
    }

    Similarly, $I_1$ exerts the same force on $I_2$, pulling the wires toward
    each other.
  \end{columns}
\end{frame}

\begin{frame}
  \frametitle{Circular Motion Caused by a Magnetic Field}
  When a charged particle enters a magnetic field at right angle\ldots
  \begin{itemize}
  \item Magnetic force $\mb{F}_M$ perpendicular to both velocity $\mb{v}$ and
    magnetic field $\mb{B}$.
  \item Results in circular motion
  \end{itemize}
  Centripetal force $\mb{F}_c$ is provided by the magnetic force $\mb{F}_M$.
  Equating the two expressions:

  \eq{-.4in}{
    \frac{mv^2}{r}=qvB
  }
  
  \vspace{-.1in}We can solve for $r$ get the radius for a charge with a known
  mass, or solve for mass $m$ of a charged particle based on its radius:  

  \eq{-.2in}{
    r = \frac{mv}{qB}\quad\quad\quad m=\frac{qrB}{v}
  }
\end{frame}



\begin{frame}
  \frametitle{Magnetic Flux}

  \textbf{Question:} If a current-carrying wire can generate a magnetic field,
  can a magnetic field affect the current in a wire?

  \vspace{.3in}\textbf{Answer:} Yes, sort of\ldots

  \vspace{.3in}To understand how to \emph{induce} a curent by a magnetic field,
  we need to look at fluxes again.
\end{frame}

\begin{frame}
  \frametitle{Magnetic Flux}
  \begin{center}
    \pic{.4}{flux2.png}
  \end{center}

  

  \vspace{-.15in}Not surprisingly, the magnetic flux is defined similar to
  electric flux:
  
  \eq{-.2in}{
    \boxed{\Phi_\mathrm{magnetic}=\int\mb{B}\cdot d\mb{A}}
  }

  where $\mb{B}$ is the magnetic field, and $d\mb{A}$ is the infinitesimal area
  with its direction point outward.
\end{frame}

\begin{frame}
  \frametitle{Magnetic Flux Over a Closed Surface}

  The magnetic flux over a closed surface is always zero:

  \eq{-.2in}{
    \boxed{\oint\mb{B}\cdot d\mb{A}=0}
  }

  Since magnetic field exists in a loop only, what every flux that leaves the
  surface has to eventually come back.
\end{frame}


\begin{frame}
  \frametitle{Changing Magnetic Flux}
  Changes to magnetic flux can be due to a number of reasons:
  \begin{enumerate}
  \item\textbf{Changing magnetic field}\ldots if the magnetic field is created
    by a time-dependent source (e.g.\ alternating current)
  \item\textbf{Changing orientation of magnetic field} either because the
    surface area is moving relative to the magnetic field.
  \item\textbf{Changing area} the surface area from which the flux is
    calculated is changing.
  \end{enumerate}
\end{frame}


\begin{frame}
  \frametitle{Faraday's Law}
  Faraday's law states that the rate of change of magnetic flux produces an
  electromotive force:

  \eq{-.2in}{
    \boxed{\mathcal{E}=-\frac{d\Phi}{dt}}
  }
\end{frame}

\end{document}
