\documentclass[11pt]{article}

\usepackage[margin=.8in,letterpaper]{geometry}
\usepackage{amsmath,bm}
\usepackage{txfonts} % must be loaded after amsmath?
\usepackage{siunitx}
%\usepackage{enumitem}
\usepackage{graphicx}
%\usepackage{tikz}
%\usepackage{mathpazo}
%\usepackage{xcolor,colortbl}
%\usepackage{hyperref}
\usepackage{cancel}

%\newcommand\vertarrowbox[2]{%
%    \begin{array}[t]{@{}c@{}} #1 \\
%    \rotatebox{90}{$\xrightarrow{\hphantom{abcdefgh}}$} \\[-1ex]
%    \mathclap{\scriptstyle\text{#2}}%
%    \end{array}}

\setlength{\parindent}{0pt}
\setlength{\parskip}{8pt}

%\usetikzlibrary{decorations.pathmorphing,patterns}

\sisetup{
  detect-all,
  per-mode=symbol
}

\title{Vectors and Calculus That You Need to Know}
\author{Timothy M.\ Leung, Ph.D.\\Olympiads School}
\date{\today}

\newcommand{\pic}[2]{\includegraphics[width=#1\textwidth]{#2}}
\newcommand{\mb}[1]{\ensuremath\mathbf{#1}}

\begin{document}

\maketitle

The AP Physics C exams are calculus based, and will use vectors extensively.
Students should be familiar with the material in this handout, however, it is
likely that the calculus and vector operations used in the exams will be much
simpler. If these concepts are difficult, this should be a good time to grab a
calculus textbook and review them.

\section{Vectors}
Vectors are used extensively in physics. They are an integral part of a larger
discipline within mathematics called \textbf{linear algebra}. For the purpose
of AP Physics, it is sufficient to think of vectors as
``a number with a direction''.

\subsection{Notation}
In keeping with the convention used in \emph{most} technical journals and
university-level textbooks, vectors are \emph{printed} (e.g.\ on the slides and
handouts) using a bold face font:
\begin{equation*}
  \mb{v}\quad\mb{F}_g\quad\mb{p}\quad\mb{I}
\end{equation*}
while the ``arrow on top'' notation is used when \emph{writing} (e.g.\ on the
blackboard)\footnote{Although this format is still used in \emph{some}
  introductory level physics textbooks in universities}:
\begin{equation*}
  \vec{v}\quad\vec{F}_g\quad\vec{p}\quad\vec{I}
\end{equation*}
The magnitude of vectors are expressed either with the absolute-value symbol:
\begin{equation*}
  |\mb{v}|\quad|\mb{F}_g|\quad|\mb{p}|\quad|\mb{I}|
\end{equation*}
or as a scalar quantity (afterall, the magnitude of a vector is indeed a scalar
with a positive value):
\begin{equation*}
  v\quad F_g\quad p \quad I
\end{equation*}


\subsection{Writing Vectors}
In Grades 11 and 12 Physics, vectors are usually written by separating the
magnitude from the direction. For example, a velocity vector are usually
written as:
\begin{equation*}
  \mb{v}=\SI{4.5}{\metre\per\second}\text{ [N \ang{55} E]}
\end{equation*}
This approach is based on using the \textbf{polar coordinate system}, which is
the preferred coordinate system for circular motion. In general, polar
coordinate system is very intuitive for describing \emph{one} vector in
two dimensions (that's why it is used extensively in high-school level
physics courses), but it is more complicated when extended into 3D; the
coordinate system needs to be extended to \textbf{spherical coordinate system}
or the \textbf{cylindrical coordinate system}. Moreover, it is difficult to
perform vector arithmetic for \emph{rectilinear} motion.

Intead, for rectilinear motion, vectors in 2D/3D Cartesian space are generally
written in their $x$, $y$ \& $z$ components using the \textbf{IJK notation}:
\begin{equation*}
  \mb{A}=A_x\bm{\hat{\imath}} + A_y\bm{\hat{\jmath}} + A_z\bm{\hat{k}}
\end{equation*}
The vectors $\bm{\hat{\imath}}$, $\bm{\hat{\jmath}}$ and $\bm{\hat{k}}$ are
\textbf{basis vectors} indicating the directions of the $x$, $y$ and $z$ axes.
Basis vectors are \textbf{unit vectors} (i.e.\ length $1$). Note that the
IJK notation does not give the magnitude of the vector, which needs to be
calculated:
\begin{equation*}
  A=|\mb{A}|=\sqrt{A_x^2 + A_y^2 + A_z^2}
\end{equation*}


\subsection{Vector Addition and Subtraction}

Adding and subtracting vectors is straightforward:
\begin{equation*}
  \mb{A}\pm\mb{B}=
  (A_x\pm B_x)\bm{\hat{\imath}} +
  (A_y\pm B_y)\bm{\hat{\jmath}} +
  (A_z\pm B_z)\bm{\hat{k}}
\end{equation*}


\subsection{Dot Product}
The vector \textbf{dot product} (or \textbf{inner product} for general vectors)
is the \emph{scalar} multiplication of two vectors. This is a vector operation
that have been used throughout Grades 11 and 12 Physics courses (although
without explicitly using this notation), for example, when calculating
mechanical work. It is determined by the magnitude of the two vectors and the
cosine of the angle $\theta$ between them:
\begin{equation*}
  C=\mb{A}\cdot\mb{B}=\mb{B}\cdot\mb{A}=|\mb{A}||\mb{B}|\cos\theta
\end{equation*}
In the cross product, $C$ is the \emph{projection} of the vector $\mb{A}$ onto
$\mb{B}$, or the component of $\mb{A}$ along $\mb{B}$.
Note that $\bm{\hat{\imath}}\cdot\bm{\hat{\imath}}=1$,
$\bm{\hat{\jmath}}\cdot\bm{\hat{\jmath}}=1$, and
$\bm{\hat{k}}\cdot\bm{\hat{k}}=1$. For general vectors written in IJK notation,
where the magnitude and direction of vectors are not immediately known, so
intead we sum the product of individual components of $\mb{A}$ and $\mb{B}$:
\begin{equation*}
  C=\mb{A}\cdot\mb{B}=A_xB_x+A_yB_y+A_zB_z
\end{equation*}


\subsection{Cross Products}
The vector \textbf{cross product} is the vector multiplication of two vectors:
\begin{equation*}
  \mb{C}=\mb{A}\times\mb{B}
\end{equation*}
The magnitude of the cross product is determined by the magnitude of $\mb{A}$
and $\mb{B}$ and the angle $\theta$ between them:
\begin{equation*}
  C=AB\sin\theta
\end{equation*}
The cross product $\mb{C}$ is perpendicular to \emph{both} $\mb{A}$ and
$\mb{B}$; its direction given by the right hand rule. Cross products are used
extensively in rotational motion and in electromagnetism
\begin{figure}[ht]
  \centering
  \pic{.3}{cross-product.png}
  \caption{Vector cross product.}
  \label{fig:cross1}
\end{figure}
Note that unlike the cross product, the order of the cross product is
important. (This is why you have to get the right hand rule correctly.)
\begin{equation*}
  \mb{A}\times\mb{B}=-\mb{B}\times\mb{A}
\end{equation*}
In general, the cross product of any two vectors in 3D space is the determinant
of this $3\times 3$ matrix:
\begin{equation*}
  \mb{A}\times\mb{B}=
  \left|
  \begin{matrix}
    \bm{\hat{\imath}} & \bm{\hat{\jmath}} & \bm{\hat{k}}\\
    A_x & A_y & A_z\\
    B_x & B_y & B_z
  \end{matrix}
  \right|
  =(A_yB_z-A_zB_y)\bm{\hat{\imath}} +
  (A_zB_x-A_xB_z)\bm{\hat{\jmath}} +
  (A_zB_y-B_yA_x)\bm{\hat{k}}
\end{equation*}
although it is extremely rare that such notation will ever be used in any
Physics C exams. Most cross product applications in AP Physics C are much
simpler, so we only have to remember the circle shown in
Figure~\ref{fig:cross2}.
\begin{figure}[ht]
  \centering
  \pic{.12}{cross-product-circle.png}
  \caption{Cross product circle that you will likely see in Physics C exams}
  \label{fig:cross2}
\end{figure}

The direction of the arrow gives the index of the cross product (e.g.\
$\bm{\hat{\imath}}\times\bm{\hat{\jmath}}=\bm{\hat{k}}$); going against the
direction of the arrow gives the negative of the next index (e.g.\
$\bm{\hat{k}}\times\bm{\hat{\jmath}}=-\bm{\hat{\imath}}$)


\section{Calculus}
We cannot learn physics properly without calculus (you got away with it for
long enough in grades 11 and 12?!)
%  \item Calculus was ``invented'' so that we can understand motion, especially
%    non-constant velocities and accelerations
%  \item You may have already noticed that a lot of the word problems in
%    calculus are really physics problems
%  \end{itemize}
%\end{frame}
%
\begin{itemize}
\item\textbf{Differential Calculus}
  \begin{itemize}
  \item How quickly something is changing (``rate of change'' of a quantity)
  \item Math: slopes of functions
  \item Physics: how quickly a physical quantity is changing in time and/or
    space
  \item Examples: velocity (how quickly position changes with time),
    acceleration (how quickly velocity changes with time), power (how quickly
    work is done), electric fields (how electric potential changes in space)
  \end{itemize}
\item\textbf{Integral Calculus}
  \begin{itemize}
  \item The opposite of differentiation
  \item We use it to compute the area under a curve, or
  \item Summation of many small terms
  \item Examples: area under the $\mb{v}$-$t$ graph (displacement), area
    under the $\mb{F}$-$t$ graph (impulse), area under the $\mb{F}$-$d$ graph
    (work)
  \end{itemize}
\end{itemize}


\subsection{Derivative}

For any arbitrary function $f(x)$, the derivative with respect to $x$ is:
\begin{equation*}
  f'(x)=\lim_{h\rightarrow 0}\frac{f(x+h)-f(x)}{h}
\end{equation*}
The ``limit as $h$ approaches $0$'' is the mathematical way of making $h$ a
very small non-zero number.


\subsection{Basic Rules for Differentiation}

The derivative of a constant $C$ with respect to any variable is zero. This
should be obvious, since the slope of any function $f(x)=C$ is zero.
\begin{equation*}
  \frac{dC}{dx}=0
\end{equation*}
A constant multiple $a$ of any function $f$ can be factored outside the
derivative:
\begin{equation*}
  \frac{d}{dx}(af)=a\frac{df}{dx}
\end{equation*}
The derivative of a sum of two functions is the sum of the derivatives of the
functions:
\begin{equation*}
  \frac{d}{dt}\left(f(t)+g(t)\right) = \frac{df}{dt}+\frac{dg}{dt}
\end{equation*}
Power Rule:
\begin{equation*}
  \frac{d}{dt}\left(t^n\right) = nt^{n-1}\quad\text{for}\quad n\neq 0
\end{equation*}
Product Rule:
\begin{equation*}
  \frac{d}{dx}\left(f(x)g(x)\right)=f'(x)g(x)+f(x)g'(x)
\end{equation*}
Chain Rule:
\begin{equation*}
  \frac{d}{dx}f\left(g(x)\right)=f'(g(x))g'(x)
\end{equation*}
Quotient Rule is rarely used in physics tests in AP or first-year university,
but you should remember it anyway:
\begin{equation*}
  \frac{d}{dx}\left[\frac{f(x)}{g(x)}\right]=
  \frac{f'(x)g(x)-g'(x)f(x)}{\left(g(x)\right)^2}
\end{equation*}



\subsection{Elementary Derivatives}
When studying \textbf{harmonic motion} and \textbf{circular motion},
trigonometric and exponential functions are often used. We will also find out
the relationship between complex exponential functions and sine/cosine
functions. %The derivatives of sines and cosines are related:
\begin{align*}
    \frac{d}{dt}\sin t &= \cos t \quad\quad\quad\\
    \frac{d}{dt}\cos t &= -\sin t
\end{align*}
And the exponential function:
\begin{equation*}
  \frac{d}{dt}e^{at} = ae^{at}
\end{equation*}

\subsection{Partial Derivatives}
Some functions have many variables (multi-variable function). For example,
gravitational potential energy $U_g$ has three variables: masses $m_1$ and
$m_2$ and the distance $r$ between them:
\begin{equation*}
    U_g(m_1,m_2,r)=-\frac{Gm_1m_2}{r}
\end{equation*}
Differentiating with respect to one variable while holding others constant
gives its \textbf{partial derivative}. (We use the $\partial$ symbol). For
example, the partial derivative of $U_g$ with respect to $r$ is
\begin{equation*}
    \frac{\partial U_g}{\partial r}=\frac{Gm_1m_2}{r^2}
\end{equation*}
In case you have not noticed: the derivative is the is the relationship between
gravitational potential energy $U_g$ and the magnitude of the gravitational
force $F_g$.


\subsection{Integration}

If $F(x)$ is the anti-derivative of $f(x)$, they are related this way:
\begin{equation*}
  \frac{d}{dx}F(x)=f(x)\quad\longrightarrow\quad F(x)=\int f(x)dx
\end{equation*}
The mathematical proof is the \textbf{fundamental theorem of calculus}.


\subsection{Common Integrals in Physics}
Integration, while often necessary, can be very daunting, but integrals in AP
Physics C  are generally straightforward. These rules should help in most cases.

Power rule in reverse:
\begin{equation*}
  \int x^ndx=\frac{1}{n+1}x^{n+1}+C
\end{equation*}
Natural logarithm:
\begin{equation*}
  \int \frac{1}{x}dx =\ln |x|+C 
\end{equation*}
Sines and cosines:
\begin{align*}
  \int\cos xdx&=\sin x+C\\
  \int\sin xdx&=-\cos x+C
\end{align*}
%We can ``ignore'' (i.e.\ cancel) the constant of integration $C$ for definite
%integrals.


\subsection{Definite and Indefinite Integrals}

\begin{itemize}
\item Integrals can be either \textbf{indefinite} or \textbf{definite}
\item An ``indefinite'' integral is another function, e.g.\ position
  $\mb{x}(t)$ as a function of time is found by integrating velocity
  $\mb{v}(t)$:

  \begin{equation*}
    \mb{x}(t)=\int\mb{v}(t)dt=\cdots+\mb{C}
  \end{equation*}
\item A \textbf{constant of integration} $\mb{C}$ is added to the integral
  $\mb{x}(t)$. It is obtained through applying ``initial condition'' to the
  problem.
\end{itemize}


\subsection{Definite Integrals}

A \textbf{definite integral} has lower and upper bounds. e.g.\ given
$\mb{v}(t)$, the displacement between $t_1$ and $t_2$ can be found:
\begin{equation*}
  \Delta\mb{x}=\int_{t_0}^{t_1} \mb{v}(t)dt
\end{equation*}
Once we have computed the integral, we evaluate the limits:
\begin{equation*}
  \Delta\mb{x} =
  \mb{x}(t)\Big|^{t_1}_{t_0}=
  \mb{x}(t_1)-\mb{x}(t_0)=
  \mb{x}_1-\mb{x}_0
\end{equation*}
The constant of integration $\mb{C}$ cancels when we evaluate the upper and
lower bounds.
\end{document}
