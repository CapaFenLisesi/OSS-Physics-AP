\documentclass{../../../Physics.12/ossphys}
\usepackage{bm}
\setlength{\parskip}{1.2em}

\begin{document}
\headfoot{AP Physics}{Symmetric Trajectory}

\begin{center}
  \vspace{.3in}{\Large
    \textbf{Symmetric Projectile Trajectory}
  }
\end{center}

A \textbf{symmetric trajectory} is a special case of projectile motion where an
object is launched at an angle of $\theta$ (between \ang{0} and \ang{90}) above
the horizontal\footnote{This may be obvious, but any angles \emph{below} the 
  horizontal will never have a symmetric trajectory.} with an initial speed
$v_i$, and then lands at the same height, as shown below in Figure~\ref{sym}.
Examples may include hitting a golf ball towards the hole, or shooting a bullet
towards ahorizontal target\footnote{Shooting a bullet towards a horizontal
  target always require an upward angle because of gravity}. The equations for
symmetric trajectory is \emph{not} included in the AP Exam equation sheet.
If you need these equations during the exams, you will need to derive them
during the exam. To derive the equations, we use the $x$-axis for the
horizontal direction and $y$-axis for the vertical.
\begin{figure}[ht]
  \begin{center}
    \begin{tikzpicture}
      \draw[->](0,0)--(7,0) node[pos=1,right]{$x$};
      \draw[->](0,0)--(0,2) node[pos=1,above]{$y$};
      \draw[dotted,domain=0:6,thick] plot (\x, {1.2*\x-.2*\x*\x});
      \draw[ultra thick,->](0,0)--(.75,.9)node[pos=1,above]{$v_i$};
      \draw[very thick,red!80!black,->]
      (0,0)--(0,.9)node[midway,left]{$\mb{v}_{y0}$};
      \draw[very thick,blue!80!black,->]
      (0,0)--(.75,0)node[midway,below]{$\mb{v}_x$};
      \draw[->](.5,0)arc(0:52:.5) node[midway,right]{$\theta$};
    \end{tikzpicture}
  \end{center}
  \vspace{-.2in}
  \caption{Symmetric project trajectory}
  \label{sym}
\end{figure}

The initial velocity $\mb{v}_i$ can be resolved into its components, also shown
in Figure~\ref{sym}:
\begin{align*}
  \mb{v}_x   &=v_i\cos\theta\hat{\bm{\imath}}\\
  \mb{v}_{y0}&=v_i\sin\theta\hat{\bm{\jmath}}
\end{align*}
$v_x$ remains constant during the motion, as there are no forces acting in the
$x$ direction, and therefore no acceleration. In the $y$ direction, there is an
acceleration due to gravity $a_y=-g$.

\textbf{Maximum height} $H$: Apply the kinematic equation in the $y$-direction.
Recognizing that at maximum height $H=y-y_0$, the vertical component of
velocity is zero $v_y=0$:
\begin{align*}
  v_y^2 &= v_{y0}^2 + 2a_y(y-y_0)\\
  0  &= (v_i\sin\theta)^2-2gH
\end{align*}
Solving for $H$, we get the maximum height equation:
\begin{equation}
  \boxed{H=\frac{v_i^2\sin^2\theta}{2g}}
\end{equation}
\newpage

\textbf{Total time of flight} $t_\mathrm{max}$: We apply the kinematic equation
in the $y$ direction. When the object lands at the same height, the final
velocity is the same in magnitude and opposite in direction as the initial
velocity, i.e.\  $v_{y2}=-v_{y1}=-v_i\sin\theta$:
\begin{align*}
  v_y &=v_{y0}+a_yt\\
  -v_i\sin\theta &=v_i\sin\theta -g t_\mathrm{max}
\end{align*}
Solving for $t_\mathrm{max}$ we have:
\begin{equation}
  \boxed{t_\mathrm{max}=\frac{2v_i\sin\theta}{g}}
  \label{tmax}
\end{equation}

\textbf{Range} $R$: We substitute the expression for $t_\mathrm{max}$ from
Eq.~\ref{tmax} into the $t$ term, then apply the kinematic equation in
the $x$-direction to compute $R=x-x_0$ for any given launch angle and initial
speed:
\begin{align*}
  x&=x_0+v_xt\\
  R &=v_i\cos\theta\left(\frac{2v_i\sin\theta}{g}\right)
\end{align*}
Using the trigonometric identity $\sin(2\theta)=2\sin\theta\cos\theta$, we
simplify the equation to:
\begin{equation}
  \boxed{R=\frac{v_i^2\sin(2\theta)}{g}}
\end{equation}
It is obvious that for any given initial speed $v_i$, the maximum range
$R_\mathrm{max}$ occurs at an angle where $\sin(2\theta)=1$
(i.e.\ $\theta=\ang{45}$), with a value of
\begin{equation}
  \boxed{R_\mathrm{max}=\frac{v_i^2}{g}}
\end{equation}
Also, for a known initial speed $v_i$ and range $R$ we can compute the launch
angle $\theta$:
\begin{displaymath}
  \theta_1=\frac{1}{2}\sin^{-1}\left(\frac{gR}{v_i^2}\right)
\end{displaymath}
This angle is labelled $\theta_1$ because it is \emph{not} the only angle that
can reach this range. Recall that for any angle $\ang{0}<\phi<\ang{90}$, there
is also another angle where the $\sin$ are equal:
\begin{displaymath}
  \sin\phi=\sin(\ang{180}-\phi)
\end{displaymath}
Which means that for any $\theta_1$, there is also another angle
$\theta_2$ where $2\theta_2=\ang{180}-2\theta_1$, or quite simply:
\begin{displaymath}
  \theta_2=\ang{90}-\theta_1
\end{displaymath}
\end{document}
