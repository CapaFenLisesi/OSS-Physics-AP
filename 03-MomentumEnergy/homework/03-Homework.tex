\documentclass{../../oss-apphys}
\usepackage{bm}

\begin{document}
\genheader

\gentitle{1 \& C}{MOMENTUM AND ENERGY}{1 \& 2}

\genmultidirections

\gengravity

\raggedcolumns
\begin{multicols}{2}

  \begin{enumerate}[leftmargin=18pt]

%A toy train car of mass 3.0 kg rolls to the left at 2 m/s and collides with
%a 4.0 kg train car rolling to the right at 1 m/s. The two cars stick
%together. The velocity of the cars after the collision is
%(A) 2/7 m/s to the left
%(B) 2/7 m/s to the right
%(C) 4/7 m/s to the left
%(D) 4/7 m/s to the right
%(E) 9/7 m/s to the right

    %MAY BE THIS ONE
%  \item Two steel balls, one of mass m and the other of mass 2m, collide and
%    rebound in a perfectly elastic collision. Which of the following is
%    conserved in this elastic collision?
%(A) velocity only
%(B) momentum only
%(C) momentum and kinetic energy only
%(D) momentum, velocity, and kinetic energy(E) kinetic energy only
%Questions 153–154. A force acts on a 2.0 kg mass during a time interval as
%shown in the graph.
%153. The impulse given to the mass from t = 0 to t = 6 s is
%(A) 4 N s
%(B) 8 N s
%(C) 12 N s
%(D) 16 N s
%(E) 24 N s
%154. If the initial speed of the mass is 2 m/s at t = 0, what is its speed at the
%end of 6 s?
%(A) 4 m/s
%(B) 6 m/s
%(C) 8 m/s
%(D) 10 m/s
    %(E) 16 m/s
  \item A rubber ball of mass $m$ strikes a wall with a speed $v$ at an angle
    $\theta$ below the normal line and rebounds from the wall at the same speed
    and angle above the normal line as shown. The change in momentum
    of the ball is
    \begin{center}
      \pic{.25}{graphics/ball-bounce-wall.png}
    \end{center}
    \begin{enumerate}[noitemsep,topsep=0pt,leftmargin=18pt,label=(\Alph*)]
    \item $mv$
    \item $2mv$
    \item $mv\cos\theta$
    \item $2mv\cos\theta$
    \item  zero
    \end{enumerate}
%    
%156. Two blocks are connected by a compressed spring and rest on a
%frictionless surface. The blocks are released from rest and pushed apart
%by the compressed spring. If one mass is twice the mass of the other,
%which of the following is the same for both blocks?
%(A) magnitude of momentum
%(B) acceleration
%(C) speed
%(D) kinetic energy
    %(E) potential energy
  \item A \SI{1000}{\kilo\gram} railroad car is rolling without friction on a
    horizontal track at a speed of \SI{3.}{\metre\per\second}. Sand is poured
    into the open top of the car for a time of \SI{5.}{\second}. The speed of
    the car after \SI{5.}{\second} is \SI{1.}{\metre\per\second}. The mass of
    the sand added to the car at the end of \SI{5.}{\second} is
    \begin{center}
      \pic{.25}{graphics/railroad-car-sand.png}
    \end{center}
    \begin{enumerate}[noitemsep,topsep=0pt,leftmargin=18pt,label=(\Alph*)]
    \item\SI{500 }{\kilo\gram}
    \item\SI{1000}{\kilo\gram}
    \item\SI{2000}{\kilo\gram}
    \item\SI{3000}{\kilo\gram}
    \item\SI{3500}{\kilo\gram}
    \end{enumerate}
    \columnbreak
    
  \item Two billiard balls are rolling to the right on a table as shown. The
    \SI{.4}{\kilo\gram} ball is moving faster than the \SI{.2}{\kilo\gram}
    ball, so it catches up and strikes it from behind at a slight angle.
    Immediately after the collision, the $y$-component of the
    \SI{.4}{\kilo\gram} ball is \SI{2}{\metre\per\second} downward.
    The $y$-component of the velocity of the \SI{.2}{\kilo\gram} ball must be
    \begin{center}
      \pic{.25}{graphics/big-small-billiard-balls.png}
    \end{center}
    \begin{enumerate}[noitemsep,topsep=0pt,leftmargin=18pt,label=(\Alph*)]
    \item \SI{1}{\metre\per\second} upward
    \item \SI{2}{\metre\per\second} upward
    \item \SI{1}{\metre\per\second} downward
    \item \SI{2}{\metre\per\second} downward
    \item \SI{4}{\metre\per\second} upward
    \end{enumerate}
    
%Questions 159–160. Two balls are on a horizontal billiard table. A 1.0 kg
%billiard ball moves downward along the y-axis with a speed of 16 m/s toward
%a 2.0 kg ball that is at rest. The balls collide at an angle, and move along the
%lines shown. After the collision, the 1.0 kg ball moves at 9 m/s along the +x-
%axis. The table below shows the x and y components of the momentum in kg
%m/s of the two balls before and after the collision.
%159. Which of the following statements is true?
%(A) Momentum is conserved only in the x-direction in this collision.
%(B) Momentum is conserved only in the y-direction in this collision.
%(C) Momentum is conserved in both the x- and y-directions in this
%collision
%(D) The momentum of the 1.0 kg ball increases after the collision.
%(E) The momentum of the 2.0 kg ball decreases after the collision.160. What is the speed of the 2.0 kg ball after the collision?
%(A) 16.0 m/s
%(B) 9.2 m/s
%(C) 7.5 m/s
%(D) 6.0 m/s
%(E) 5.0 m/s
%161. A 0.3 kg baseball at rest on a tee is struck by a bat. The ball leaves the
%bat with a speed of 20 m/s at an angle of 45° above the horizontal. The
%magnitude of the impulse imparted to the baseball by the bat is most
%nearly
%(A) 2 N s
%(B) 6 N s
%(C) 12 N s
%(D) 16 N s
%(E) 20 N s
%162. Two ice skaters, a large man and a small woman, are initially at rest
%and holding each other’s hands. They push away horizontally.
%Afterward, which of the following statements is true?
%(A) They have equal and opposite kinetic energies.
%(B) The have equal and opposite momenta.
%(C) The large man applies a greater force to the small woman.
%(D) The small woman applies a greater force to the large man.
    %(E) They recoil with equal and opposite velocities.

  \end{enumerate}
  \columnbreak
  
  \textbf{Questions 4-5}

  An object has a mass $4m$. The object explodes into three pieces of mass
  $m$, $m$, and $2m$. The two pieces of mass m move off at right
  angles to each other with the same momentum $mv$, as shown.
  \begin{center}
    \pic{.25}{graphics/3-piece-bomb.png}
  \end{center}
  \begin{enumerate}[leftmargin=18pt,resume]
  \item The speed of mass 2m after the explosion is
    \begin{enumerate}[noitemsep,topsep=0pt,leftmargin=18pt,label=(\Alph*)]
    \item $2v$
    \item $\sqrt{2}v$
    \item $\frac{\sqrt{2}}{2}v$
    \item $\frac{\sqrt{2}}{3}v$
    \item $\frac{\sqrt{3}}{2}v$
    \end{enumerate}
    
  \item  The direction of velocity of mass $2m$ is
    \begin{enumerate}[noitemsep,topsep=0pt,leftmargin=18pt,label=(\Alph*)]
    \item $\rightarrow$
    \item $\swarrow$
    \item $\downarrow$
    \item $\nearrow$
    \item $\uparrow$
    \end{enumerate}
  \end{enumerate}
%165. A system consists of two blocks having masses of 2 kg and 1 kg. Theblocks are connected by a string of negligible mass and hung over a
%light pulley, and then released from rest. When the speed of each block
%is v, the momentum of the center of mass of the system is
%(A) (2 kg + 1 kg)v
%(B) (2 kg - 1 kg)v
%(C) 1/3 (2 kg + 1 kg)v
%(D) 1⁄2 (2 kg - 1 kg)v
    %(E) (2 kg)v

  \columnbreak
  \textbf{Questions 6-7}

  A projectile is launched at an angle to the level ground as shown. At the top
  of the trajectory at point $P$, the projectile explodes into two pieces of
  mass $2m$ and $m$.
  \begin{center}
    \pic{.25}{graphics/exploding-projectile.png}
  \end{center}

  \begin{enumerate}[leftmargin=18pt,resume]
  \item Which of the following arrows best represents the direction of the
    velocity of the center of mass of the projectile at point $P$ after the
    explosion?
    \begin{enumerate}[noitemsep,topsep=0pt,leftmargin=18pt,label=(\Alph*)]
    \item $\leftarrow$
    \item $\swarrow$
    \item $\searrow$
    \item $\rightarrow$
    \item $\nearrow$
    \end{enumerate}

  \item Which of the following statements is true of the center of mass of the
    projectile after the explosion?
    \begin{enumerate}[noitemsep,topsep=0pt,leftmargin=18pt,label=(\Alph*)]
    \item The center of mass will continue on a parabolic path and land on
      the ground at the place where it would have landed had it not exploded.
    \item The center of mass will alter its parabolic path and land on the
      ground farther from where it would have landed had it not exploded.
    \item The center of mass will alter its parabolic path and land on the
      ground at a shorter distance than it would have landed had it not
      exploded.
    \item The center of mass will fall straight downward from the point of
      explosion.
    \item The center of mass will travel straight upward from the point of
      explosion.
    \end{enumerate}
    
%Questions 168–169. Two pieces of clay of equal mass m moving with equal
%speeds v o each traveling at an angle of 30° collide and stick together at the
%origin O as shown.
%168. Which of the following arrows represents the direction of the velocity
%of the combined mass after the collision?
%(A)
%(B)
%(C)
%(D)
%(E)
%169. The speed of the combined mass after the collision is
%(A) v o
%(B) 1⁄2 v o
%(C) 1⁄4 v o(D)
    %(E)
    \columnbreak
    
  \item A small mass $m$ is moving with a speed $v$ toward a stationary mass
    $2m$. The speed of the center of mass of the system is
    \begin{enumerate}[noitemsep,topsep=0pt,leftmargin=18pt,label=(\Alph*)]
    \item $\displaystyle\left(\frac{m}{m+2m}\right)v$
    \item $\displaystyle\left(\frac{m+2m}{m}\right)v$
    \item $\displaystyle\left(\frac{m}{2m}\right)v$
    \item $\displaystyle\left(1+\frac{m}{2m}\right)v$
    \item $\displaystyle\left(1+\frac{32m}{m}\right)v$
    \end{enumerate}
  \end{enumerate}
  
  \textbf{Questions 9-10}

  Three identical masses can slide freely on a horizontal surface as shown.
  The first mass moves with a speed of 3.0 m/s toward the second and third
  masses, which are initially at rest. The first and second mass collide
  elastically, and then the second and third masses collide inelastically.
  \begin{center}
    \pic{.4}{graphics/3-masses.png}
  \end{center}

  \begin{enumerate}[leftmargin=18pt,resume]
  \item The speed of the second mass after the collision is
    \begin{enumerate}[noitemsep,topsep=0pt,leftmargin=18pt,label=(\Alph*)]
    \item zero
    \item\SI{1.5}{\metre\per\second}
    \item\SI{3.}{\metre\per\second}
    \item\SI{6.}{\metre\per\second}
    \item\SI{9.}{\metre\per\second}
    \end{enumerate}

  \item The speed of the second and third masses after they collide
    inelastically is
    \begin{enumerate}[noitemsep,topsep=0pt,leftmargin=18pt,label=(\Alph*)]
    \item zero
    \item\SI{1.5}{\metre\per\second}
    \item\SI{3.0}{\metre\per\second}
    \item\SI{6.0}{\metre\per\second}
    \item\SI{9.0}{\metre\per\second}
    \end{enumerate}
    
%173. The diagram in the figure shows the top view of two identical steel
%balls on a horizontal table of negligible friction. The first ball moves
%with a speed of 12 m/s and the second ball is initially at rest. After the
%collision, the first ball moves with a speed of 8 m/s at an angle of 37°
%to the vertical. Which of the following diagrams best represents the
%approximate speed and direction of the second ball after the collision?
%(A)(B)
%(C)
%(D)
%(E)
%174. A known force F acts on an unknown mass for a known time Δt. From
%this information, you could determine the
%(A) change in kinetic energy of the object
%(B) change in velocity of the object
%(C) acceleration of the object
%(D) mass of the object
%(E) change in momentum of the object175. A block of mass m is moving to the right with a speed v o on a
%horizontal surface of negligible friction when it explodes. The
%explosion causes the block to break into two pieces, each of which
%moves in the horizontal direction. One piece of mass m/4 moves to the
%left with a speed of 2v o . What is the velocity of the other piece?
%(A)
%(B)
%(C)
%(D)
%(E)
%2v o to the right
%v o to the right
%3⁄4 v o to the right
%1⁄2 v o to the right
%1⁄4 v o to the left
%Questions 176–177. The graph shown indicates the force acting on a mass of
%2 kg as a function of time.
%176. For the time interval from t = 0 to t = 6 s, the change in momentum of
%the 2 kg mass is
%(A) 48 kg m/s
%(B) 24 kg m/s
%(C) 12 kg m/s
%(D) -12 kg m/s
%(E) zero
%177. If the object starts from rest, the speed at the end of the time interval
%from t = 0 to t = 3 s is
%(A) zero(B)
%(C)
%(D)
%(E)
%12 m/s
%18 m/s
%24 m/s
    %36 m/s
    \columnbreak
    
  \item A \SI{100}{\kilo\gram} cannon sits at rest with a \SI{1}{\kilo\gram}
    cannonball in the barrel. The cannonball is fired with a speed of
    \SI{50}{\metre\per\second} to the right, causing the cannon to recoil with
    a speed of \SI{.5}{\metre\per\second} to the left. The velocity of the
    center of mass of the cannon-cannonball system is
    \begin{enumerate}[noitemsep,topsep=0pt,leftmargin=18pt,label=(\Alph*)]
    \item zero
    \item\SI{5}{\metre\per\second} to the right
    \item\SI{5}{\metre\per\second} to the left
    \item\SI{50}{\metre\per\second} to the right
    \item\SI{50}{\metre\per\second} to the left
    \end{enumerate}
    
%179. The vector shown represents the initial momentum of a moving object.
%The object collides with another object that is initially at rest. Which of
%the diagrams below could represent the momenta of the colliding
%objects after the collision?
%(A)
%(B)
%(C)
%(D)
%(E)
%Questions 180–181. A 20 kg boy runs at a speed of 3.0 m/s and jumps onto a
%40 kg sled on frictionless ice that is initially at rest. The boy and the sled then
%move together for a short time.
%180. The speed of the boy and sled after he jumps on it is
%(A) 0.5 m/s(B)
%(C)
%(D)
%(E)
%0.8 m/s
%1.0 m/s
%1.5 m/s
%2.0 m/s
%181. While the boy and sled are moving, he jumps off the back of the sled in
%such a way the boy is at rest, and the sled continues to move forward.
%The speed of the sled after the boy jumps off is
%(A) 1.5 m/s
%(B) 2.0 m/s
%(C) 3.0 m/s
%(D) 4.5 m/s
%(E) 6.0 m/s
%Questions 182–183
%A cart of mass m 1 is initially moving with a speed of 4.0 m/s on a track
%toward a stationary cart of mass m 2 = 2 kg. After the collision, mass m 1
%moves with a velocity of 1.5 m/s. The force vs. time graph is shown for the
%time during the collision, with the collision beginning at t = 0.
%182. The impulse each cart applies to the other is most nearly
%(A) 40 N s
%(B) 20 N s
%(C) 10 N s
%(D) 5 N s(E) 2 N s
%183. The unknown mass m 1 is equal to
%(A)
%(B)
%(C)
%(D)
%(E)
%0.5 kg
%1.5 kg
%2.5 kg
%4.0 kg
%5.0 kg
%184. A 1.0 kg block is released from rest from a height h at the top of a
%fixed curved ramp of negligible friction. The block slides down the
%ramp and collides with another block of mass 1.5 kg at rest at the
%bottom of the ramp. The two blocks stick together and move with a
%speed of 5 m/s. The height h from which the 1.0 kg block began is
%(A) 0.8 m
%(B) 1.2 m
%(C) 1.8 m
%(D) 2.8 m
%(E) 7.8 m
%185. A dart of mass m is fired into a wooden block of mass 4m that hangs
%from a string. The dart and block then rise to a maximum height h. An
%expression for the initial speed v o of the dart before striking the block is
%(A)
%(B)
%(C)
%(D)
%(E)Questions 186–187. A small block of mass m slides on a horizontal
%frictionless surface toward a ramp of mass 3m which is also free to move on
%the surface. The small block slides up to a height h on the ramp with no
%friction (Figure I), then they move together (Figure II), and the small block
%slides back down the ramp to the horizontal surface (Figure III). Both the
%block and the ramp continue to slide on the horizontal surface after they
%separate.
%186. Which of the following is true regarding the conservation laws
%throughout this process?
%(A) Kinetic energy is conserved from Figure I to Figure II.
%(B) Momentum is conserved from Figure I to Figure III.
%(C) Kinetic energy is conserved from Figure II to Figure III.
%(D) Potential energy is conserved from Figure I to Figure II.
%(E) Potential energy is conserved from Figure II to Figure III.
%187. Which of the following is a true statement regarding Figure III?
%(A) The small block is moving to the left and the ramp is moving to
%the right.
%(B) The small block is moving to the right and the ramp is moving to
%the left.
%(C) The small block is moving to the right and the ramp is moving tothe right.
%(D) The small block is moving to the left and the ramp is moving to
%the left.
%(E) The small block and the large block are moving with the same
%velocity.
%Questions 188–189. A rubber ball of mass m is released from rest from a
%height h onto a fixed inclined plane angled at 45° to the horizontal. The ball
%collides with the surface elastically.
%188. Which of the following diagrams best indicates the direction of the
%impulse vector J as it strikes the plane and the velocity vector v just
%after it strikes the plane?
%(A)
%(B)
%(C)
%(D)
%(E)
%189. The speed of the ball just after striking the surface is(A)
%(B)
%(C)
%(D)
%(E)

  \item A \SI{1000}{\kilo\gram} (empty mass) railroad car is rolling without
    friction on a horizontal track at a speed of \SI{2.}{\metre\per\second}.
    Sand is poured into the open top of the car for the time interval from
    $t=0$ to $t=\SI{4.}{\second}$. The mass of the sand poured into the car as
    a function of time is $m(t)=60t^2$. The velocity of the car at a time of
    \SI{4.}{\second} is most nearly
    \begin{center}
      \pic{.25}{graphics/railroad-car-sand.png}
    \end{center}
    \begin{enumerate}[noitemsep,topsep=0pt,leftmargin=18pt,label=(\Alph*)]
    \item\SI{1}{\metre\per\second}
    \item\SI{2}{\metre\per\second}
    \item\SI{3}{\metre\per\second}
    \item\SI{4}{\metre\per\second}
    \item\SI{5}{\metre\per\second}
    \end{enumerate}
  \end{enumerate}
  \columnbreak
  
  \textbf{Questions 13-14}

  A remote controlled stunt car of mass \SI{800}{\kilo\gram} initially moving at
  \SI{10}{\metre\per\second} is crashed into a rail car of mass $m$ that is
  initially at rest. The cars stick together, and the speed $v$ of both cars
  after the collision is given by $\displaystyle v=\frac{6}{t+1}$.
  \begin{enumerate}[resume,leftmargin=18pt]
  \item By considering the fact that the crash occurs at time $t=0$, determine
    the mass $m$ of the rail car.
    \begin{enumerate}[noitemsep,topsep=0pt,leftmargin=18pt,label=(\Alph*)]
    \item\SI{288}{\kilo\gram}
    \item\SI{445}{\kilo\gram}
    \item\SI{533}{\kilo\gram}
    \item\SI{698}{\kilo\gram}
    \item\SI{800}{\kilo\gram}
    \end{enumerate}
    
  \item The magnitude of the resisting force acting on the cars as a function of
    time after the collision is
    \begin{enumerate}[noitemsep,topsep=0pt,leftmargin=18pt,label=(\Alph*)]
    \item $\displaystyle \frac{6m}{t+1}$
    \item $6m(t+1)$
    \item $6m(t+1)^2$
    \item $\displaystyle\frac{6m}{(t+1)^2}$
    \item $\displaystyle\frac{m(t+1)^2}{6}$
    \end{enumerate}
    
%193. A force acts on a mass m according to the equation F = 12t 3 . If the
%object starts from rest, the velocity of the object as a function of time is
%(A) 36mt 3
%(B)
%(C)
%(D)
%(E)
%194. A dart in a long blow gun starts from rest and gains a momentum
%according to the equation p = 3t 3 + 2t while moving through the barrel
%of the gun. The net force acting on the dart after 0.2 s is
%(A) 1.2 N
%(B) 2.4 N
%(C) 6.0 N(D) 12.2 N
%(E) 16.1 N
%195. A variable force acts on a mass causing it to accelerate. If a graph of
%this force vs. time is plotted, the change in momentum of the mass can
%be determined by finding the
%(A) slope of the graph
%(B) area under the graph
%(C) y-intercept of the graph
%(D) x-intercept of the graph
%(E) change in slope of the graph
%196. A moving object is changing its momentum during a time interval. If a
%graph of momentum vs. time is plotted, the net force acting on the mass
%at any time can be determined by finding the
%(A) slope of line tangent to the graph at that time
%(B) area under the graph
%(C) y-intercept of the graph
%(D) x-intercept of the graph
%(E) change in slope of the graph from beginning to end
%
%
  \item Two masses moving along the coordinates axes as shown collide at the
    origin and stick to each other. What is the angle $\theta$ that the final
    velocity that makes with the $x$-axis?
    \begin{center}
      \pic{.28}{collision1.png}
    \end{center}

    \begin{enumerate}[noitemsep,topsep=0pt,leftmargin=18pt,label=(\Alph*)]
    \item $\tan^{-1}(v_2/v_1)$
    \item $\tan^{-1}[m_1v_1/(m_1+m_2)]$
    \item $\tan^{-1}(m_1v_2/m_2v_1)$
    \item $\tan^{-1}(m_2v_2^2/m_1v_1^1)$
    \item $\tan^{-1}(m_2v_2/m_1v_1)$
    \end{enumerate}
    \columnbreak
    
  \item If a projectile thrown directly upward reaches a maximum height $h$ and
    spends a total time in the air of $T$, the average power of the
    gravitational force during the trajectory is
    \begin{enumerate}[noitemsep,topsep=0pt,leftmargin=18pt,label=(\Alph*)]
    \item $P=2mgh/T$
    \item $P=-2mgh/T$
    \item 0
    \item $P=mgh/T$
    \item $P=-mgh/T$
    \end{enumerate}

  \item Given that the constant net force on an object and the object's 
    displacement, which of the following quantities can be calculated?
    \begin{enumerate}[noitemsep,topsep=0pt,leftmargin=18pt,label=(\Alph*)]
    \item the net change in the object's velocity
    \item the net change in the object's mechanical energy
    \item the average acceleration
    \item the net change in the object's kinetic energy
    \item the net change in the object's potential energy
    \end{enumerate}

  \item If the only force acting on an object is given by the equation
    $F(x)=2-4x$ (where the force is measured in newtons and position in meters),
    what is the change in the object's kinetic energy as it moves from $x=2$ to
    $x=1$?
    \begin{enumerate}[noitemsep,topsep=0pt,leftmargin=18pt,label=(\Alph*)]
    \item +\SI{4}{\joule}
    \item -\SI{4}{\joule}
    \item +\SI{2}{\joule}
    \item -\SI{2}{\joule}
    \item +\SI{8}{\joule}
    \end{enumerate}

  \item A mass traveling in the $+x$ direction collides with a mass at rest.
    Which of the following statements is true?
    \begin{enumerate}[noitemsep,topsep=0pt,leftmargin=18pt,label=(\Alph*)]
    \item After the collision, the two masses will move with parallel velocities
    \item After the collision, the masses will move with antiparallel velocities
    \item After the collision, the masses will both move along the x-axis
    \item After the collision, the $y$-components of the velocities of the two
      particles will sum to zero.
    \item None of the above
    \end{enumerate}
  \end{enumerate}
  \columnbreak
  
  \textbf{Questions 20-21}

  \begin{enumerate}[resume,leftmargin=18pt]
  \item Consider the potential energy function shown below. Assuming that no
    non-conservative forces are present, if a particle of mass $m$ is released
    from position $x_0$, what is the maximum speed it will achieve?
    \begin{center}
      \pic{.28}{potential-well.png}
    \end{center}
    \begin{enumerate}[noitemsep,topsep=0pt,leftmargin=18pt,label=(\Alph*)]
    \item $\sqrt{4U_0/m}$
    \item $\sqrt{2U_0/m}$
    \item $\sqrt{U_0/m}$
    \item $\sqrt{U_0/2m}$
    \item The particle will achieve no maximum speed but instead will continue
      to accelerate indefinitely.
    \end{enumerate}

  \item Which of the following is the most accurate description of the system
    introduced in the previous question?
    \begin{enumerate}[noitemsep,topsep=0pt,leftmargin=18pt,label=(\Alph*)]
    \item stable equilibrium
    \item unstable equilibrium
    \item neutral equilibrium
    \item a bound system
    \item There is a linear restoring force
    \end{enumerate}
    
  \item A mass $m_1$ initially moving at speed $v_0$ collides with and sticks
    to a spring attached to a second, initially stationary mass $m_2$. The two
    masses continue to move to the right on a frictionless surface as the
    length of the spring oscillates. At the instant that the spring is
    maximally extended, the velocity of the first mass is
    \begin{center}
      \pic{.35}{mass-spring-1.png}
    \end{center}
    \begin{enumerate}[noitemsep,topsep=0pt,leftmargin=18pt,label=(\Alph*)]
    \item $v_0$
    \item $m_1^2v_0/(m_1+m_2)^2$
    \item $m_2v_0/m_1$
    \item $m_1v_0/m_2$
    \item $m_1v_0/(m_1+m_2)$
    \end{enumerate}
    \columnbreak
    
  \item A pendulum bob of mass $m$ is released from rest as shown in the figure
    below. What is the tension in the string as the pendulum swings through the
    lowest point of its motion?
    \begin{center}
      \pic{.25}{pendulum1.png}
    \end{center}
    \begin{enumerate}[noitemsep,topsep=0pt,leftmargin=18pt,label=(\Alph*)]
    \item $T=\frac{1}{2}mg$
    \item $T=mg$
    \item $T=\frac{3}{2}mg$
    \item $T=2mg$
    \item None of the above
    \end{enumerate}

  \item Two blocks of mass $m_A$ and $m_B$ are connected by a string that
    passes over a light pulley. The mass of $A$ is larger than the mass of $B$.
    The speed of mass $A$ just before reaching the floor is:
    \begin{center}
      \pic{.2}{graphics/pulley-a-b.png}
    \end{center}
    \begin{enumerate}[noitemsep,topsep=0pt,leftmargin=18pt,label=(\Alph*)]
    \item $\displaystyle\sqrt{\frac{m_A-m_B}{m_A+m_B}gD}$
    \item $\displaystyle\sqrt{\frac{m_A+m_B}{m_A-m_B}gD}$
    \item $\displaystyle\sqrt{\frac{m_A}{m_A+m_B}gD}$
    \item $\displaystyle\sqrt{\frac{m_B}{m_A+m_B}gD}$
    \item $\displaystyle\sqrt{\frac{m_A}{m_B}gD}$
    \end{enumerate}

  \item A particle of mass m moves according to the displacement equation
    $x=2t^{5/2}$. The kinetic energy of the particle as a function of time is
    \begin{enumerate}[noitemsep,topsep=0pt,leftmargin=18pt,label=(\Alph*)]
    \item $10mt^{5/2}$
    \item $10mt^{3/2}$
    \item $\frac{5}{2}mt^3$
    \item $5mt^2$
    \item $2mt^{3/2}$
    \end{enumerate}

  \item The potential energy of an object varies with the equation
    $U(x)=2x^2+x-6$, where force is in newtons and displacement is in meters. A
    force F vs.\ displacement $x$ graph would yield which of the following?
    \begin{enumerate}[noitemsep,topsep=0pt,leftmargin=18pt,label=(\Alph*)]
    \item A straight, horizontal line
    \item A parabola
    \item An exponential decay curve
    \item A straight line with a positive slope
    \item A straight line with a negative slope
    \end{enumerate}

  \item An object is moved from rest at point $P$ to rest at point $Q$ in a
    gravitational field. The net work against the gravitational field depends
    on the
    \begin{enumerate}[noitemsep,topsep=0pt,leftmargin=18pt,label=(\Alph*)]
    \item mass of the object and the positions of $P$ and $Q$
    \item mass of the object only
    \item positions of $P$ and $Q$ only
    \item length moved between points $P$ and $Q$
    \item coefficient of friction
    \end{enumerate}
  \end{enumerate}
  \columnbreak
  
  \textbf{Questions 28-29}

  A force is applied to a block of mass m at a downward angle of $\theta$ to
  the vertical as shown. The block moves with a constant speed across a rough
  floor for a distance $x$.

  \begin{enumerate}[resume,leftmargin=18pt]
  \item The work done by the applied force on the block is
    \begin{enumerate}[noitemsep,topsep=0pt,leftmargin=18pt,label=(\Alph*)]
    \item $Fx\sin\theta$
    \item $Fx\cos\theta$
    \item $Fmx\sin\theta$
    \item $Fmx\cos\theta$
    \item zero
    \end{enumerate}
    
  \item The coefficient of friction between the block and the floor is
    \begin{enumerate}[noitemsep,topsep=0pt,leftmargin=18pt,label=(\Alph*)]
    \item $\displaystyle\frac{F}{mg}$
    \item $\displaystyle\frac{F\cos\theta}{mg}$
    \item $\displaystyle\frac{F\cos\theta}{F\sin\theta+mg}$
    \item $\displaystyle\frac{F\sin\theta}{F\cos\theta+mg}$
    \item $\displaystyle\frac{F\cos\theta}{F\sin\theta}$
    \end{enumerate}

  \item An electron travels in a circle around a hydrogen nucleus at a very high
    speed. The work done by the electrostatic force acting on the electron
    after one complete revolution is
    \begin{enumerate}[noitemsep,topsep=0pt,leftmargin=18pt,label=(\Alph*)]
    \item zero
    \item positive
    \item negative
    \item equal to the kinetic energy of the electron
    \item equal to the potential energy of the electron
    \end{enumerate}
    
  \end{enumerate}
\end{multicols}

\newpage
\begin{center}
  {\Large
    \textbf{AP\textsuperscript{\textregistered} Physics 1 \&C: Momentum and
      Energy\\
      Student Answer Sheet for Multiple-Choice Section}
  }
  
  \begin{minipage}[t]{.3\textwidth}
  \vspace{.2in}
  \bgroup
  \begin{tabular}{>{\centering}m{1.3cm} >{\centering}m{1.7cm}}
    No. & Answer
  \end{tabular}\\
  \def\arraystretch{1.5}
  \begin{tabular}{|>{\centering}m{1.3cm}|>{\centering}m{1.7cm}|}
    \hline
    1 & \\ \hline
    2 & \\ \hline
    3 & \\ \hline
    4 & \\ \hline
    5 & \\ \hline
    6 & \\ \hline
    7 & \\ \hline
    8 & \\ \hline
    9 & \\ \hline
    10 & \\ \hline
    11 & \\ \hline
    12 & \\ \hline
    13 & \\ \hline
    14 & \\ \hline
    15 & \\ \hline
    16 & \\ \hline
    17 & \\ \hline
    18 & \\ \hline
    19 & \\ \hline
    20 & \\ \hline
    21 & \\ \hline
    22 & \\ \hline
    23 & \\ \hline
    24 & \\ \hline
    25 & \\ \hline
  \end{tabular}
  \egroup
  \end{minipage}
  \begin{minipage}[t]{.3\textwidth}
  \vspace{.2in}
  \bgroup
  \begin{tabular}{>{\centering}m{1.3cm} >{\centering}m{1.7cm}}
    No. & Answer
  \end{tabular}\\
  \def\arraystretch{1.5}
  \begin{tabular}{|>{\centering}m{1.3cm}|>{\centering}m{1.7cm}|}
    \hline
    26 & \\ \hline
    27 & \\ \hline
    28 & \\ \hline
    29 & \\ \hline
    30 & \\ \hline
  \end{tabular}
  \egroup
  \end{minipage}
\end{center}
\newpage

\genfreetitle{1 \& C}{MOMENTUM AND ENERGY}{5}

\genfreedirections{10}

\begin{enumerate}[leftmargin=15pt]

\item A mass $m$ is placed on an incline of angle $\theta$ at a distance $d$
  from the end of a spring as shown below. The coefficient of kinetic friction
  between the mass and the plane is $\mu$.
  \begin{center}
    \pic{.3}{ramp1.png}
  \end{center}
  \begin{enumerate}[leftmargin=18pt]
  \item The mass is released from rest at the position shown. Using Newton's
    laws, calculate the block's speed when it reaches the spring.
    \vspace{1.25in}
  \item Using energy conservation, alculate the block's speed when it reaches
    the spring.
    \vspace{1.25in}
  \item The spring has spring constant $k$. At what value $x$ of the compression
    of the spring does the object reach its maximum speed?
  \end{enumerate}
  \newpage
  
\item A mass $m$ attached to a string of length $2r$ swings, starting at rest
  when the string is horizontal, until the string is vertical. At the instant
  the string is vertical, the mass makes contact with the horizontal surface,
  the string is cut, and the mass continues along a frictionless track as shown
  below.
  \begin{center}
    \vspace{-.2in}\pic{.4}{string2.png}
  \end{center}
  \begin{enumerate}[leftmargin=18pt]
  \item What is the speed of the mass attached to the string the instant the
    string is cut?
    \vspace{1in}
  \item Sketch the forces acting on the mass when it is in the position shown
    below.
    \begin{center}
      \pic{.25}{circle1.png}
    \end{center}
  \end{enumerate}
  When the mass is in the position shown above,
  \begin{enumerate}[leftmargin=18pt,resume]
  \item Find the object's speed as a function of $\theta$
    \vspace{.8in}
  \item Find the object's centripetal acceleration as a function of $\theta$
    \vspace{.8in}
  \item Determine at what angle $\theta$ the mass will fall of the track
  \end{enumerate}
  \newpage
  
\item A projectile is fired from the edge of a cliff \SI{100}{\metre} high with
  an initial speed of \SI{60}{\metre\per\second} at an angle of elevation of
  \ang{45}.
  \begin{enumerate}[noitemsep]
  \item Write equation for $x(t)$, $y(t)$, $v_x$ and $v_y$. Choose the origin of
    your coordinate system at the particle's original location.
    \vspace{1.25in}
  \item Calculate the location and velocity of the particle at time
    $t=\SI{5}{s}$.
    \vspace{1.25in}
  \end{enumerate}
  Suppose the projectile experiences an internal explosion at time $t=\SI{4}{s}$
  with an internal force purely in the $y$-direction, causing it to break into
  \SI{2}{\kg} and a \SI{1}{\kg} fragment.
  \begin{enumerate}[noitemsep]
    \setcounter{enumii}{2}

  \item If the \SI{2}{\kg} fragment is \SI{77}{m} above the height of the
    cliff at $t=\SI{5}{s}$, what is the $y$-coordinate of the position of the
    \SI{1}{\kg} piece?
    \vspace{1.25in}
  \item If the speed of the \SI{2}{kg} fragment is \SI{46}{m/s} and the
    fragment is falling at $t=\SI{5}{s}$, what is the $y$-component of the
    velocity of the \SI{1}{kg} fragment?
  \end{enumerate}
  \newpage
  
\item The Ballastic Pendulum. To determine the muzzle speed of a gun, a bullet
  is shot into a mass $M$ from a string as shown below, causing $M$ to swing
  upward through a maximum angle of $\theta$.
  \begin{center}
    \pic{.4}{ballastic.png}
  \end{center}
  \begin{enumerate}[noitemsep]
  \item What is the speed of $M$ the instant after the bullet lodges in it?
    \vspace{1.25in}
  \item What is the speed of the bullet before it hits $M$?
    \vspace{1.25in}
  \item What is the tension in the string at the highest point of the pendulum's
    swing (when the string makes an angle of $\theta$ with the vertical as
    shown)?
  \end{enumerate}
  \newpage
\item Two masses are connected by a spring (spring constant $k$) resting on a
  frictionless horizontal surface as shown. The right mass is initially in
  contact with a wall. A brief blow to the left block leaves it with an initial
  velocity $v_0$ to the right.
  \begin{enumerate}[leftmargin=18pt]
  \item What is the maximum compression of the spring as the left block moves
    to the right?
    \vspace{1in}
  \end{enumerate}
  After the spring is maximally compressed, it eventually moves to the left,
  away from wall. As it moves away from the wall, it continues oscillating.
  \begin{center}
    \pic{.45}{mass-spring-2.png}
  \end{center}
  \begin{enumerate}[leftmargin=18pt,resume]
  \item What is the net momentum of the two masses after they leave the wall?
    \vspace{1in}
  \item What is the total mechanical energy of the oscillating spring system?
    \vspace{1in}
  \item What is the relative velocity of the two masses when the spring is
    maximally compressed?
    \vspace{1in}
  \item What is the maximum compression of the spring after the two masses have
    left the wall? Compare the compression to the maximum compression calculated
    in part (a) and explain any similarities and differences.
  \end{enumerate}




\end{enumerate}
\end{document}
