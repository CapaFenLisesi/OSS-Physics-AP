\documentclass{../../oss-apphys}
\usepackage{bm}

\begin{document}
\genheader

\gentitle{1 \& C}{CIRCULAR MOTION}{5 \& 6}

\genmultidirections

\gengravity

\raggedcolumns
\begin{multicols}{2}
%
%
%\documentclass[12pt]{article}
%
%\usepackage[margin=0.8in,letterpaper]{geometry}
%\usepackage{enumitem}
%\usepackage{graphicx}
%\usepackage{tikz}%,graphicx,wrapfig}
%\usepackage{mathpazo}
%\usepackage[scaled]{helvet}
%\usepackage{siunitx}
%
%\sisetup{number-math-rm=\mathnormal}
%
%\renewcommand{\familydefault}{\sfdefault}
%
%\newcommand{\pic}[2]{\includegraphics[width=#1\textwidth]{#2}}
%\newcommand{\magdir}[2]{$#1\;[\mathrm{#2}]$}
%\newcommand{\mb}[1]{\mathbf{#1}}
%
%\begin{document}
%%\pagestyle{empty}
%\begin{center}
%  Student \#: \underline{\hspace{1in}}\hspace{1.9in}
%  Student Name: \underline{\hspace{2in}}\\
%  \vspace{0.3in}
%  {\LARGE
%    AP Physics \hspace{0.68in} Class 5: Circular and Rotational Motion
%  }
%\end{center}
%
  \begin{enumerate}[leftmargin=18pt]

  \item Linear acceleration is to force as angular acceleration is to
    \begin{enumerate}[noitemsep,topsep=0pt]
    \item kinetic energy
    \item angular velocity
    \item rotational inertia
    \item torque
    \item angular momentum
    \end{enumerate}

  \item A girl stands on a rotating merry-go-round without holding on to a rail.
    The force that keeps her moving in a circle is the
    \begin{enumerate}[noitemsep,topsep=0pt]
    \item frictional force on the girl directed away from the center of the
      merry-go-round
    \item frictional force on the girl directed toward the center of the
      merry-go-round
    \item normal force on the girl directed away from the center of the
      merry-go-round
    \item normal force on the girl directed toward the center of the
      merry-go-round
    \item weight of the girl
    \end{enumerate}

  \item A \SI{.5}{\kg} ball on the end of a \SI{.5}{m} long string is swung in a
    horizontal circle. What would the speed of the ball have to be for the
    tension in the string to be \SI{9.}{\newton}?\\
    \begin{minipage}{0.35\textwidth}
      \begin{enumerate}[noitemsep,topsep=0pt]
      \item 1.0 m/s
      \item 3.0 m/s
      \item 6.0 m/s
      \item 9.0 m/s
      \item 12.0 m/s
      \end{enumerate}
    \end{minipage}
    \begin{minipage}{0.55\textwidth}
      \pic{.75}{ball-horizontal1.png}
    \end{minipage}
    \columnbreak
    
  \item A ball of mass $m$ is swung in a vertical circle of radius $R$. The
    speed of the ball at the bottom of the circle is $v$. The tension in the
    string at the bottom of the circle is
    \begin{enumerate}[noitemsep,topsep=0pt]
    \item $\displaystyle mg$
    \item $\displaystyle mg+\frac{mv^2}{R}$
    \item $\displaystyle mg-\frac{mv^2}{R}$
    \item $\displaystyle \frac{mv^2}{R}$
    \item zero
    \end{enumerate}

  \item A car of mass $m$ drives on a flat circular track of radius $R$. To
    maintain a constant speed $v$ on the track, the coefficient of friction $\mu$
    between the tires and the road must be
    \begin{enumerate}[noitemsep,topsep=0pt]
    \item $\displaystyle mg$
    \item $\displaystyle mg+\frac{mv^2}{R}$
    \item $\displaystyle mg-\frac{mv^2}{R}$
    \item $\displaystyle \frac{v^2}{gR}$
    \item $\displaystyle \sqrt{\frac{v^2}{gR}}$
    \end{enumerate}
    \columnbreak
    
  \item A meter stick of mass \SI{0.1}{\kg} rests on a table as shown. A length
    of \SI{40}{\cm} extends over the edge of the table. How far from the edge of
    the table could a \SI{.05}{kg} mass be placed on the meter stick so that the
    stick just begins to tip?
    \begin{center}
      \vspace{-.2in}
      \pic{.35}{beam1.png}
    \end{center}
    \begin{enumerate}[noitemsep,topsep=0pt]
    \item\vspace{-.2in}\SI{5 }{\cm}
    \item\SI{10}{\cm}
    \item\SI{15}{\cm}
    \item\SI{20}{\cm}
    \item\SI{30}{\cm}
    \end{enumerate}

  \item A meter stick is balanced on a fulcrum at its center, as shown. A mass
    of \SI{5}{\kg} is hung on the left end of the stick, and a mass of
    \SI{2}{kg}
    is hung on the right end. In order to balance the system, a mass $m$ is hung
    at the $25$-\si{cm} mark on the right side. What is the value of the mass
    $m$?
    \begin{center}
      \pic{.4}{beam2.png}
    \end{center}
    \begin{enumerate}[noitemsep,topsep=0pt]
    \item\SI{12}{\kg}
    \item\SI{6 }{\kg}
    \item\SI{4 }{\kg}
    \item\SI{3 }{\kg}
    \item\SI{2 }{\kg}
    \end{enumerate}
    
  \item A ball on the end of a string is swung in a circle of radius \SI{2}{m}
    according to the equation $\theta = 4t^2 + 3t$, where $\theta$ is in radians
    and $t$ is in seconds. The angular acceleration of the ball is
    \begin{enumerate}[noitemsep,topsep=0pt]
    \item\SI{6}{rad/s^2}
    \item $4t^2 + 3t$ \si{rad/s^2}
    \item $8t +3$ \si{rad/s^2}
    \item $\frac{3}{4} t^3 + 3t^2$ \si{rad/s^2}
    \item \SI{8}{rad/s^2}
    \end{enumerate}
    \columnbreak
  
  \item The linear speed $v$ of the ball (in the previous question) at
    $t=\SI{3}{s}$ is
    \begin{enumerate}[noitemsep,topsep=0pt]
    \item\SI{27 }{m/s}
    \item\SI{54 }{m/s}
    \item\SI{108}{m/s}
    \item\SI{135}{m/s}
    \item\SI{210}{m/s}
    \end{enumerate}

  \item A metal bar of constant density and weight $W$ is attached to a pivot on
    the wall at point $P$ and supported by a rope that makes an angle of
    \ang{60} with the vertical wall. The reaction force exerted by the pivot on
    the bar at point P is best represented by which arrow?
    \begin{center}
      \pic{.25}{metal-bar.png}
    \end{center}
    \begin{enumerate}[noitemsep,topsep=0pt]
    \item $\nearrow$
    \item $\uparrow$
    \item $\downarrow$
    \item $\nwarrow$
    \item $\searrow$
    \end{enumerate}

  \item A uniform rod of length $L$ and mass $m$ has a rotational inertia of
    $\displaystyle \frac{1}{12}mL^2$ about its center. A particle, also of mass
    $m$, is attached to one end of the stick. The combined rotational inertia of
    the stick and particle about the center of the rod is
    \begin{center}
      \pic{.4}{I.png}
    \end{center}
    \begin{enumerate}[noitemsep,topsep=0pt]
    \item$\displaystyle \frac{mL^2}{3}$
    \item$\displaystyle \frac{12mL^2}{13}$
    \item$\displaystyle \frac{13mL^2}{12}$
    \item$\displaystyle \frac{mL^2}{156}$
    \item$\displaystyle \frac{13mL^2}{156}$
    \end{enumerate}

  \item A hoop of radius $R$ and mass $m$ has a rotational inertia of $mR^2$. The
    hoop rolls without slipping along a horizontal floor with a constant
    speed $v$ and then rolls up a long incline. The hoop can roll up the
    incline to a maximum vertical height of
    \begin{center}
      \pic{.4}{hoop1.png}
    \end{center}
    \begin{enumerate}[noitemsep,topsep=0pt]
    \item$\displaystyle \frac{v^2}{g}$
    \item$\displaystyle \frac{2v^2}{g}$
    \item$\displaystyle \frac{v^2}{2g}$
    \item$\displaystyle \frac{4v^2}{g}$
      \item$\displaystyle \frac{v^2}{4g}$
    \end{enumerate}

    
  \item Two disks are fixed to a vertical axle that is rotating with a constant
    angular speed $\omega$. The smaller disk has a mass $m$ and a radius $r$,
    and
    the larger disk has a mass $2m$ and radius $2r$. The general equation for the
    rotational inertia of a disk of mass $M$ and radius $R$ is $\frac{1}{2}MR^2$.
    The ratio of the angular momentum of the larger disk to the smaller disk is
    \begin{center}
      \pic{.25}{2disks.png}
    \end{center}
    \begin{enumerate}[noitemsep,topsep=0pt]
    \item$1:4$
    \item$4:1$
    \item$1:2$
    \item$2:1$
    \item$8:1$
    \end{enumerate}
    \columnbreak
    
  \item A light rod has a mass attached at each end. At one end is a \SI{6}{\kg}
    mass, and at the other end is a \SI{3}{\kg} mass. An axis can be placed at
    any of the points shown. Through which point should an axis be placed so that
    the rotational inertia is the greatest about that axis?
    \begin{center}
      \pic{.4}{light-rod.png}
    \end{center}
    \begin{enumerate}[noitemsep,topsep=0pt]
    \item A
    \item B
    \item C
    \item D
    \item E
    \end{enumerate}
    
  \item Two wheels are attached to each other and fixed so that they can only
    turn together. The smaller wheel has a radius of $r$ and the larger wheel
    has a radius of $3r$. The two wheels can rotate together on a frictionless
    axle. Three forces act tangentially on the edge of the wheels as shown.
    The magnitude of the net torque acting on the system of wheels is
    \begin{center}
      \pic{.25}{2wheels.png}
    \end{center}
    \begin{enumerate}[noitemsep,topsep=0pt]
    \item$Fr$
    \item$2Fr$
    \item$3Fr$
    \item$4Fr$
    \item$6Fr$
    \end{enumerate}
    \columnbreak
    
  \item Astronauts are conducting an experiment in a negligible gravity
    environment. Two spheres of mass m are attached to either end of a
    light rod. As the rod and spheres float motionless in space, an astronaut
    launches a piece of sticky clay, also of mass m, toward one of the
    spheres so that the clay strikes and sticks to the sphere perpendicular to
    the rod. Which of the following statements is true of the motion of the
    rod, clay, and spheres after the collision?
    \begin{center}
      \pic{.25}{collision1.png}
    \end{center}
    \begin{enumerate}[topsep=5pt]
    \item Linear momentum is not conserved, but angular momentum is conserved.
    \item Angular momentum is not conserved, but linear momentum is conserved.
    \item Kinetic energy is conserved, but angular momentum is not conserved.
    \item Kinetic energy is conserved, but linear momentum is not conserved.
    \item Both linear momentum and angular momentum are conserved, but kinetic
      energy is not conserved.
    \end{enumerate}
    \columnbreak
    
  \item One end of a stick of length $L$, rotational inertia $I$, and mass $m$ is
    pivoted on an axle with negligible friction at point $P$. The other end is
    tied to a string and held in a horizontal position. When the string is cut,
    the stick rotates counterclockwise. The angular speed $\omega$ of the stick
    when it reaches the bottom of its swing is
    \begin{center}
      \pic{.4}{end-of-stick.png}
    \end{center}
    \begin{enumerate}[noitemsep,topsep=0pt]
    \item$\displaystyle\frac{mgL}{I}$
    \item$\displaystyle\sqrt{\frac{mgL}{I}}$
    \item$\displaystyle\sqrt{\frac{2mgL}{I}}$
    \item$\displaystyle\sqrt{\frac{mgL}{2I}}$
    \item$\displaystyle\sqrt{\frac{4mgL}{I}}$
    \end{enumerate}

  \item A belt is wrapped around two wheels as shown. The smaller wheel has
    a radius $r$, and the larger wheel has a radius $2r$. When the wheels turn,
    the belt does not slip on the wheels, and gives the smaller wheel an
    angular speed $\omega$. The angular speed of the larger wheel is
    \begin{center}
      \pic{.3}{wheels.png}
    \end{center}
  
    \begin{enumerate}[noitemsep,topsep=0pt]
    \item $\displaystyle \omega$
    \item $\displaystyle 2\omega$
    \item $\displaystyle \frac{1}{2}\omega$
    \item $\displaystyle \frac{1}{4}\omega$
    \item $\displaystyle 4\omega$
    \end{enumerate}
    \columnbreak

  \item A disk is mounted on a fixed axle. The rotational inertia of the disk is
    $I$. The angular velocity of the disk is decreased from $\omega_0$ to
    $\omega_f$ during a time $\Delta t$ due to friction in the axle. The
    magnitude of the average net torque acting on the wheel is
    \begin{enumerate}[noitemsep,topsep=0pt]
    \item $\displaystyle\frac{\omega_f-\omega_0}{\Delta t}$
    \item $\displaystyle\frac{(\omega_f-\omega_o)^2}{\Delta t}$
    \item $\displaystyle\frac{I(\omega_f-\omega_o)}{\Delta t}$
    \item $\displaystyle\frac{I(\omega_f-\omega_o)^2}{\Delta t}$
    \item $\displaystyle\frac{I(\omega_f-\omega_o)}{\Delta t^2}$
    \end{enumerate}

  \item The average power developed by the friction in the axle of the disk
    from the previous question to bring it to a complete stop is
    \begin{enumerate}[noitemsep,topsep=0pt]
    \item $\displaystyle\frac{\omega_o}{\Delta t}$
    \item $\displaystyle\frac{(\omega_o)^2}{\Delta t}$
    \item $\displaystyle\frac{I(\omega_f-\omega_o)}{\Delta t}$
    \item $\displaystyle\frac{I\omega_o^2}{\Delta t}$
    \item $\displaystyle\frac{I(\omega_f-\omega_o)}{\Delta t^2}$
    \end{enumerate}
    
  \item A light rod of negligible mass is pivoted at point $P$ a distance $L$ from
    one end as shown. A mass $m$ is attached to the left end of the rod at a
    distance of $3L$ from the pivot, and another mass $4m$ is attached to the
    other end a distance $L$ from the pivot. The system begins from rest in the
    horizontal position. The net torque acting on the system due to gravitational
    forces is
    \begin{center}
      \pic{.4}{light-rod2.png}
    \end{center}
    \begin{enumerate}[noitemsep,topsep=0pt]
    \item $4mgL$ clockwise
    \item $3mgL$ clockwise
    \item $3mgL$ counterclockwise
    \item $mgL$ counterclockwise
    \item $mgL$ clockwise
    \end{enumerate}

  \item The angular acceleration of the system when it is released from rest is
    \begin{enumerate}[noitemsep,topsep=0pt]
    \item zero
    \item $\displaystyle\frac{g}{5L}$
    \item $\displaystyle\frac{g}{4L}$
    \item $\displaystyle\frac{g}{13L}$
    \item  $\displaystyle\frac{g}{L}$
    \end{enumerate}
  \end{enumerate}

\end{multicols}

\newpage
\begin{center}
  {\Large
    \textbf{AP\textsuperscript{\textregistered} Physics 1 \&C: Circular Motion\\
      Student Answer Sheet for Multiple-Choice Section}
  }
  
%  \begin{minipage}[t]{.3\textwidth}
  \vspace{.2in}
  \bgroup
  \begin{tabular}{>{\centering}m{1.3cm} >{\centering}m{1.7cm}}
    No. & Answer
  \end{tabular}\\
  \def\arraystretch{1.5}
  \begin{tabular}{|>{\centering}m{1.3cm}|>{\centering}m{1.7cm}|}
    \hline
    1 & \\ \hline
    2 & \\ \hline
    3 & \\ \hline
    4 & \\ \hline
    5 & \\ \hline
    6 & \\ \hline
    7 & \\ \hline
    8 & \\ \hline
    9 & \\ \hline
    10 & \\ \hline
    11 & \\ \hline
    12 & \\ \hline
    13 & \\ \hline
    14 & \\ \hline
    15 & \\ \hline
    16 & \\ \hline
    17 & \\ \hline
    18 & \\ \hline
    19 & \\ \hline
    20 & \\ \hline
    21 & \\ \hline
    22 & \\ \hline
%    23 & \\ \hline
%    24 & \\ \hline
%    25 & \\ \hline
  \end{tabular}
  \egroup
%  \end{minipage}
%  \begin{minipage}[t]{.3\textwidth}
%  \vspace{.2in}
%  \bgroup
%  \begin{tabular}{>{\centering}m{1.3cm} >{\centering}m{1.7cm}}
%    No. & Answer
%  \end{tabular}\\
%  \def\arraystretch{1.5}
%  \begin{tabular}{|>{\centering}m{1.3cm}|>{\centering}m{1.7cm}|}
%    \hline
%    26 & \\ \hline
%    27 & \\ \hline
%    28 & \\ \hline
%    29 & \\ \hline
%    30 & \\ \hline
%    31 & \\ \hline
%    32 & \\ \hline
%    33 & \\ \hline
%    34 & \\ \hline
%    35 & \\ \hline
%    36 & \\ \hline
%    37 & \\ \hline
%    38 & \\ \hline
%  \end{tabular}
%  \egroup
%  \end{minipage}
\end{center}
\newpage

\genfreetitle{1 \& C}{CIRCULAR MOTION}{2}

\genfreedirections{10}

\begin{enumerate}[leftmargin=15pt]

\item A mass $m$ is hung on a string that is wrapped around a disk of radius
  $R$ and rotational inertia $I$. The mass is released from rest and
  accelerates downward with an acceleration $a$.
  \begin{enumerate}[noitemsep]
  \item Determine the tension in the string as the mass accelerates downward
    in terms of the given quantities.
  \item In terms of the tension $T$ and the other given quantities, determine
    the rate of change of the angular speed of the disk.
  \end{enumerate}
  \pic{.3}{mass-disk.png}
  \vspace{0.5in}
  
\item A disk having a rotational inertia of \SI{2.}{\kilo\gram.\metre^2}
  rotates about a fixed axis through its center. The disk begins from rest at
  $t=0$, and at time $t=\SI{2}{\s}$, its angular velocity is \SI{2}{rad/s}.
  \begin{enumerate}[noitemsep]
  \item Determine the angular momentum of the disk at $t=\SI{2}{s}$.
  \item What is the angular acceleration of the disk between $t=0$ and
    $t=\SI{2}{s}$?
  \item What is the kinetic energy of the disk at $t=\SI{2}{s}$?
  \end{enumerate}
  \pic{0.3}{rotDisk.png}
\end{enumerate}
\end{document}
