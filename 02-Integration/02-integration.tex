\documentclass[12pt,compress,aspectratio=169]{beamer}

\mode<presentation>
{
  \usetheme{Singapore}
  \setbeamersize{text margin left=.5cm,text margin right=.5cm}
%  \setbeamertemplate{navigation symbols}{} % suppress nav bar
%  \setbeamercovered{transparent}
}
\usefonttheme{professionalfonts}
\usepackage{amsmath}
\usepackage{siunitx}
%\usepackage{graphicx}
%\usepackage{tikz}
\usepackage{mathpazo}
\usepackage[scaled]{helvet}
%\usepackage{xcolor,colortbl}
%\usepackage{hyperref}

\sisetup{number-math-rm=\mathnormal}

\title{2.\ Calculus in Physics--Integration}
\subtitle{AP Physics}
\author[TML]{Dr.\ Timothy Leung}
\institute{Olympiads School}
\date{Fall 2017}

\newcommand{\pic}[2]{\includegraphics[width=#1\textwidth]{#2}}
\newcommand{\mb}[1]{\ensuremath\mathbf{#1}}

\begin{document}

\begin{frame}
  \frametitle{Files for You to Download}
  \begin{itemize}
  \item\texttt{00-outline.pdf}--The course outline
  \item\texttt{01-Calculus-print.pdf}--The slides that I used last week
  \item\texttt{01-integration.pdf}--The slides that I am using right now
  \item\texttt{01-Homework.pdf}--Last/this week's homework assignment
  \end{itemize}
  Please download/print the PDF file for the class slides before each class.
\end{frame}

\section{Differentiation}

\begin{frame}
  \frametitle{On Differential Calculus}
  \framesubtitle{A quick review}
  \begin{itemize}
  \item Finding out how quickly a physical quantity is changing (``rate of
    change'' of that quantity)
  \item Math: slopes of functions
  \item Terminology:
    \begin{itemize}
    \item A \textbf{\emph{derivative}}: The slope of a function (noun)
    \item To \textbf{\emph{differentiate}}: Finding the derivative with respect
      to a variable (verb)
    \end{itemize}
  \item Last class: went through the rules and some examples of derivatives
  \end{itemize}
\end{frame}

\begin{frame}
  \frametitle{Examples of Derivatives in Physics}
  \begin{itemize}
  \item Instantaneous velocity $\mb{v}(t)$ is the derivative of position
    $\mb{s}(t)$
    {\Large
      \begin{displaymath}
        \mb{v}(t)=\frac{d\mb{s}}{dt}
      \end{displaymath}
    }
  \item Instantaneous acceleration $\mb{a}(t)$ is the derivative of velocity
    $\mb{v}(t)$. It's also the ``second derivative'' (derivative of a
    derivative) of position $\mb{s}$ with respect to time
    
    \vspace{-.2in}{\Large
      \begin{displaymath}
        \mb{a}(t)=\frac{d\mb{v}}{dt}=\frac{d^2\mb{s}}{dt^2}
      \end{displaymath}
    }
  \item Instantaneous force $\mb{F}(t)$ is the derivative of momentum
    $\mb{p}$ (Newton's second law of motion)

    \vspace{-.3in}{\Large
      \begin{displaymath}
        \mb{F}(t)=\frac{d\mb{p}}{dt}
      \end{displaymath}
    }
  \end{itemize}
\end{frame}

\begin{frame}
  \frametitle{They're All Vectors}
  \framesubtitle{Resolve them into components}
  \begin{itemize}
  \item Notice that position $\mb{s}$, velocity $\mb{v}$, acceleration $\mb{a}$,
    momentum $\mb{p}$, force $\mb{F}$ are all vector quantities with $x$, $y$
    and $z$ components
  \item In this case, we take the derivative separately in each direction.

    \vspace{-.2in}{\Large
      \begin{displaymath}
        \mb{v}(t)=\frac{d\mb{s}}{dt}
        =\frac{d}{dt}\left(s_x(t)\mb{i}+s_y(t)\mb{j}+s_z(t)\mb{k} \right)
      \end{displaymath}
    }
    where $s_x$, $s_y$ and $s_z$ are the $x$-, $y$- and $z$-components of
    $\mb{s}$
  \item In AP or 1st-year physics, $s_x$, $s_y$ and $s_z$ are functions of time
    only, but in practical problems in physics and engineering, they are often
    functions of $x$, $y$ and $z$ coordinates as well. (This is
    \emph{multi-variable calculus}. It's a lot of fun!)
  \end{itemize}
\end{frame}

\begin{frame}
  \frametitle{What the Notation Tell Us}
  \begin{itemize}
  \item When we say we that \textbf{velocity is the time rate of change of
    position}
    {\Large
      \begin{displaymath}
        \mb{v}=\frac{d\mb{s}}{dt}
      \end{displaymath}
    }
  \item We are really asking \textbf{what is the (small) change in position
    $d\mb{s}$ for an infinitesimal (infinitely small) change in time $dt$?}
  \end{itemize}
\end{frame}

\section{Integration}

\begin{frame}
  \frametitle{}
  \begin{center}
    {\LARGE\textbf{NOW ON TO INTEGRATION}}
  \end{center}
\end{frame}

\begin{frame}
  \frametitle{Integration: Area Under the Curve}
  \framesubtitle{Let's start with an example}
  \begin{itemize}
  \item A car is moving with speed $v(t)=5t$. What is its displacement at $t=5$?
  \item We know that if on a $v$-$t$ graph, and the area under that curve is
    the displacement. So how do we find the area?
  \item If we divide $5$ into many small time intervals:
    \begin{displaymath}
      \Delta t_1,\;\Delta t_2,\;\Delta t_3,\;\Delta t_4,\ldots,\;\Delta t_n
    \end{displaymath}
    We can find the displacement in teach of these $\Delta t_i$, and
  \item In this example, the total displacement would be  
    \begin{displaymath}
      d(5)=\lim_{n\rightarrow\infty}\sum_{i=1}^{n}v(t_i)\Delta t_i=\int_{t_1}^{t_2}v(t)dt=
      \int_{t=0}^{5}5t\;dt=\frac{5}{2}t^2\Big|^5_0=\frac{125}{2}
    \end{displaymath}
  \end{itemize}
\end{frame}

%\begin{frame}
%  \frametitle{Integration: Differentiation in Reverse}
%  \begin{displaymath}
%    \frac{d}{dt}\left(t^2\right)=\frac{1}{2}t
%    \quad\quad\longrightarrow\quad\quad
%    \int\frac{1}{2}tdt=t^2
%  \end{displaymath}
%\end{frame}
%
%\begin{frame}
%  \frametitle{Commonly Used Integrals in Physics}
%  Calculating an integral can be a very daunting task. But these few rules
%  should help:
%
%  \begin{align*}
%    \int x^ndx&=\frac{1}{n+1}x^{n+1}+C\\
%    \int \frac{1}{x}&=\ln x+C\\
%    \int\cos xdx&=\sin x+C\\
%    \int\sin xdx&=-\cos x+C
%  \end{align*}
%\end{frame}
%
%\begin{frame}
%  \frametitle{Area Under A Curve}
%  What is the area under the curve
%  \begin{displaymath}
%    f(x)=2x^2+3x+1\quad\textsf{between}\quad x=1\;\textsf{and}\;x=5
%  \end{displaymath}
%  Our integration works like this:
%  \begin{align*}
%    A&=\int_1^5\left(2x^2+3x+1\right)dt\\
%    &=\left(\frac{2}{3}x^3+\frac{3}{2}x^2+x\right)|^5_3\\
%    &=24+\frac{196}{3}
%  \end{align*}
%\end{frame}
%
%\begin{frame}
%  \frametitle{Kinematic Equations}
%  \begin{itemize}
%  \item Remember this equation:
%
%    \vspace{-.2in}{\large
%      \begin{displaymath}
%        s(t)=s_0+v_0t+\frac{1}{2}at^2
%      \end{displaymath}
%    }
%    
%    \vspace{-0.1in}(the notation that you used may be a little bit different,
%    but it's the same equation)
%  \item We actually obtained this by integrating a constant acceleration
%  \end{itemize}
%\end{frame}
%
%\begin{frame}
%  \frametitle{Integration to Find Volume}
%  \begin{itemize}
%  \item Interested in finding the volume when we rotate \emph{any} function
%    about the $x$ axis
%  \item Many applications in physics, e.g.\ finding the centre of
%    mass or centroid of shapes
%  \end{itemize}
%  \begin{columns}
%    \column{.33\textwidth}
%    \pic{1}{cone.png}
%    \column{.64\textwidth}
%    \begin{itemize}
%    \item Each circular disk the yellow has a volume of $\pi r^2dx$,
%      where $r=f(x)$, so the volume of each disk is in fact:
%      
%      \vspace{-0.3in}{\Large
%        \begin{displaymath}
%          dV=\pi f(x)^{2} dx
%        \end{displaymath}
%      }    
%    \item ``summing'' them together gives us the integral:
%      
%      \vspace{-0.2in}{\Large
%        \begin{displaymath}
%          \boxed{V=\int_{x_1}^{x_2} dV=\int_{x_1}^{x_2} \pi f(x)^{2} dx}
%        \end{displaymath}
%      }
%    \end{itemize}
%  \end{columns}
%\end{frame}
%
%\begin{frame}
%  \frametitle{Integration to Find Volume}
%  \textbf{Example:} Find the volume of the following shape:
%  \begin{itemize}
%  \item In this question, $f(x)=3x$, and we are integrating from $x_1=0$ to
%    $x_2=1$
%  \end{itemize}
%  \vspace{.1in}
%  \begin{columns}
%    \column{.37\textwidth}
%    \pic{1}{cone.png}
%    \column{.6\textwidth}
%    We use the formula from before:
%    \begin{align*}
%      V&=\int_{x_1}^{x_2} \pi f(x)^{2} dx\\
%      &=\int_{0}^{1} \pi 9x^2dx\\
%      &=9\pi\int_{0}^{1} x^2dx\\
%      &=3\pi x^3\Big|^1_0\\
%      &=3\pi
%    \end{align*}
%  \end{columns}
%\end{frame}
%
%\begin{frame}
%  \frametitle{One Last Example}
%
%  \textbf{Using Integration to calculate work done by non-constant force}
%
%  A force of $F(t)=5t\si{\N}$ is applied on an object $m=\SI{1}{\kg}$ at
%  rest, there is no friction force. What would be the displacement and work
%  done on this object at $t=\SI{3}{\s}$?
%
%  \begin{enumerate}
%  \item<2-> Apply Newton's second law to find acceleration:
%    $\displaystyle a(t)=\frac{F}{m}=5t$
%  \item<3-> Then we integrate to get velocity:
%    $\displaystyle v(t)=\int a(t)=\frac{5}{2}t^2$
%  \item<4-> And finally, displacement:
%    $\displaystyle s(t)=\int v(t)=\frac{5}{6}t^3\quad\longrightarrow\quad
%    \textsf{at}\;t=3, \boxed{d=\frac{45}{2}\si{m}}$
%  \item<5-> Integrate force with velocity to find work done:
%    $\displaystyle W=\int F(t)v(t)dt =\int\frac{25}{2}t^3dt=\frac{25}{8}t^4
%    \quad\longrightarrow\quad
%    \textsf{at}\;t=3,\;\boxed{W=\frac{2025}{8}\si{J}}$
%  \end{enumerate}
%\end{frame}
%

\begin{frame}
  \frametitle{Remember Our Kinematic Equations?}
  \begin{itemize}
  \item In Physics 11 and 12, you were introduced to a set of 5 kinematic
    equations, which applies to constant acceleration.
  \item Now that we know something about integration, we can understand these
    equations a little bit better
  \item We start with a constant acceleration $a$. The velocity is the integral:
    
    \vspace{-.2in}{\Large
      \begin{displaymath}
        v(t)=\int adt=at+C
      \end{displaymath}
    }
  \item We know that at $t=0$, $v=v_0$ (``initial value''). Substituting those
    allow to find $C=v_0$, and therefore
    
    \vspace{-0.3in}{\Large
      \begin{displaymath}
        \boxed{v(t)=v_0+at}
      \end{displaymath}
    }
  \end{itemize}
\end{frame}

\begin{frame}
  \frametitle{Remember Our Kinematic Equations?}
  \begin{itemize}
  \item Now we integrate $v(t)$ again to get position $s(t)$:
    \vspace{-0.1in}{\Large
      \begin{displaymath}
        s(t)&=\int v(t)dt=\int(v_0+at)dt=v_ot + \frac{1}{2}at^2+C
      \end{displaymath}
    }
  \item Again, we take advantage of know our initial position, so $C=s_o$, and
    we have:
    
    \vspace{-0.2in}{\Large
      \begin{displaymath}
        \boxed{s(t)= s_0 + v_ot + \frac{1}{2}at^2}
      \end{displaymath}
    }
  \item You may be more familiar with this expression, where we use
    \emph{displacement} $\Delta s(t) = s(t)-s_0$ instead of position $s$:

    \vspace{-0.25in}{\Large
      \begin{displaymath}
        \Delta s(t)= v_ot + \frac{1}{2}at^2
      \end{displaymath}
    }
  \end{itemize}
\end{frame}

\begin{frame}
  \frametitle{Remember Our Kinematic Equations?}
  \begin{itemize}
  \item In practical situations, acceleration is \emph{not} constant, and we
    generally have to differentiate or integrate to find your answers.
  \end{itemize}
\end{frame}

\end{document}
